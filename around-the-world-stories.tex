\documentclass[12pt]{book}
\usepackage{fontspec}
\usepackage{hyperref}
\usepackage{microtype}
\setmainfont{Linux Libertine}

\let\cleardoublepage=\clearpage

\begin{document}

\frontmatter

\pagenumbering{gobble}

\begin{titlepage}
\vspace*{\fill}

\center

\LARGE
\sc

Around the World Stories

\vspace{5em}

\large

by
\vspace{1em}

\Large
Olive Risley Seward

\rm \large
\vspace{0.5em}

Editor of\\
William H. Seward’s “Travels Around the World”

\vspace{5em}

\it
The patriot’s boast, where’er we roam,\\
His first, best country ever is at home.\\
\hspace{14em} \rm Goldsmith.

\vspace*{\fill}
\end{titlepage}


\sc
\vspace*{\fill}

\begin{center}
\large
Inscribed to the Memory of
\vspace{0.5em}

\LARGE C. J. \large \\
Born, 5th July, 1864, \\
Died, May 12th, 1886.
\end{center}

\vspace*{\fill}
\rm

\newpage

\begin{center}
{\Large \textbf{\textit{\textsc{To Alfred Rodman, Junior.}}}}
\end{center}

\vspace{0.5em}

\noindent {\sc My dear Alfred: —}

These stories are gathered from recollections of travels your father and mother
and I once enjoyed, together with Mr.~Seward, and embody incidents which have
most amused you, when in response to youthful importunities for “more talk and
stories,” I have called on my memory to gratify you.

In preserving to your fancy some of our novel experiences, I have been led
to hope that the tales which have amused you might not be unwelcome to other
young people.

The world changes so fast it is not impossible that by the time you are old
enough to “put a girdle round about the earth,” yourself, you may be satisfied
with such a view of the Great Wall as can then be had in a half‐hour’s halt at a
Railway station in Northern China, without the hardship which was the price we
elders paid for our knowledge of that strange country and wonderful structure.

Who can tell, indeed, if “Wild West shows,” and baseball tournaments may not,
in these days of new ideas and rash experiments, invade India, to the destruction
of such picturesque pageants of Oriental splendor as were spread before our eyes.

If such were the case, and these “around the world stories” proved useful as
a guide to the beautiful old India we knew, I should feel amply repaid for the
pleasant task of writing them out for you.

\vspace{1em}

Believe me, dear Alfred,

\vspace{1.5em}
\begin{center}
Ever your affectionate aunt,
\end{center}

\vspace{0.5em}

\hfill {\sc Olive Risley Seward.}

\textit{Washington, April 18, 1889.}

\makeatletter
\renewcommand*\l@chapter{\@dottedtocline{1}{0em}{2.5em}}
\makeatother

\textbf{\tableofcontents}

\renewcommand\thechapter{\Roman{chapter}}

\mainmatter

\chapter{A Journey to Peking.}

The reserved and self‐satisfied Chinese people resented the intrusion of foreigners
and their ways, even long before Marco Polo made his voyage to Cathay. And the
first British embassadors who went to Peking were only permitted to land after
sternest threats and most difficult negotiations, ending in a solemn agreement
that if the Chinese authorities would allow them to enter the capital they would
go as “tribute‐bearers” to the “Son of Heaven,” dependents, bearing gifts to the
“Emperor of the Central Flowery Kingdom of the Earth.” And in this manner
Lord MacCartney and his successor went to Peking, the insignia of the humble
characters they assumed being emblazoned on banners floating above their own
royal standard.

Now there is no difficulty in landing, of course, for the Chinese ports are free
to the world; but — it is a very long journey, all must admit, to go from almost
anywhere to China, and when at last, having crossed oceans and continents, one
disembarks at Shanghai or Hong Kong, it seems like coming into a European port
where Chinese servants and laborers predominate, and is rather disappointing.
English is spoken on all sides, and one is taken to a house fitted and furnished with
all familiar American comforts, even the bubbling water‐pitchers filled with Tudor
and Company’s Boston ice, and, notwithstanding the all‐permeating Chinese odor
of straw matting and bamboo, it is difficult realizing that one has reached the land
of the Celestials.

It is soon explained that all the foreign residents live in dwellings within
a “concession,” or “compound,” a section of land set apart for this purpose by
contract with the Chinese government. One hears very little about China or the
Chinese in these foreign communities, which are cut off from all communication
with the natives, while the foreigners are always talking of “home,” and never care
to travel in China after their first experiment and when the novelty has worn off.
Their tales of discomfort and terror, in connection with their Oriental journeys,
seemed rather poor‐spirited to us, and quite unworthy such advantages for travel
and improvement as they possessed, and to attain which we had come so far.

When Mr.~Seward, my sister and I left home, our first points of interest to
reach were Peking, and the Great Wall of China. There were the Pacific Ocean,
Japan, and an endless number of things to be seen on the way; but, to our fancies,
this was the first grand object of our travels.

When we arrived at Shanghai, all sorts of impediments to the realization of
our plan presented themselves. People said it was too late in the season, too
cold, the journey too rough. None of these discouragements affected our purpose,
however.

One consideration only was very serious, and that it daunted us, we could not
deny. There had been, within a few weeks, a horrible massacre at Tien Tsin, a
great town on the Pei‐ho, the port of Peking, and through which we must pass
to get to that city and the Great Wall. This massacre had been too dreadful to
describe, and quite as unlawful, cowardly, and cruel as any of the massacres of
Chinese laborers have been, in Oregon, or on the Pacific Coast of America.

There were grave apprehensions that it was but the beginning of a general
movement on the part of the Chinese against all foreigners, especially at the
North, and it was regarded as unsafe and rash for a small party of Americans
to travel unprotected among the hostile people. There was a common desire
on the part of the foreigners to have a certain exhibition made of the power of
the European governments which were represented there; and all the different
squadrons in Asiatic waters sent their gun boats into the Chinese ports, and
presented an array well calculated to impress the people. The flag ship of the
United States squadron was at Shanghai, and the Admiral in command was an old
friend of Mr.~Seward’s, Admiral John Rodgers, and he now proposed going with
us to Peking, and further decided to take with him his staff officers, a company of
seamen and marines, and the post‐band of the squadron. It was thought that this
demonstration would reassure the foreigners at Tien Tsin, and in the North, and
at the same time inspire respect for American pluck and authority among the
Chinese, which might prevent any repetition of their cruelties and save a great
deal of trouble, if not prevent war.

We younger people had very little concern for the grave reasons for those
preparations, and made a pleasure of the band and the lively companionship of
the young staff officers. We were three ladies, Mr.~Seward, Admiral Rodgers,
eight officers, and two young American civilian friends — a very merry party.

The voyage of five days up the Yellow Sea, the Gulf of Pe‐chee‐lee, and the
Pei‐ho River, in a fine American steamer, brought us to Tien Tsin, and here the
music and dancing and merriment were interrupted, and all comfort ended, for
we began to taste the experience of real Chinese travelling and to appreciate the
feelings of the Shanghai foreign residents, to which we soon found we had shown
small justice.

Our initiation was accomplished by means of the small native scows, for
which we exchanged our good steamer at this point. My sister and I had one boat
to ourselves, which we named the “Monitor,” and our Chinese servant, Ah Moon,
a long‐eyed, pig‐tailed, blue‐bloused boy of sixteen, accompanied us, and we had
four native boatmen.

The boat was the roughest contrivance for navigation ever invented, I believe.
It was about twenty feet long, flat‐bottomed and sported a square cotton sail, the
whole rigging compassed by a reef and one bamboo spar. In this shell we started
on a voyage up the Pei‐ho in November. We had a cabin about eight feet square
in the middle of the boat, resembling the turret of the gallant little iron‐clad for
which our poor Chinese craft was named, and Ah Moon certainly turned the box
into a bower. He spread gray Chinese goat skins on the floor, and over the small
bunk, which was also furnished with a mattress and pillows for us to sleep on,
and then draped the rough boarded sides and the roof of the cabin with scarlet
blankets. Although we had no fire, I cannot deny that we were as cosey as a
couple of little kangaroos in our fur‐lined pocket cabin.

As soon as it was light, Ah Moon brought us water for the toilet, but, ugh!
the ice had to be broken in at the top of the basin, and left a thick jagged edge
all around! Then this good boy brought us hot coffee, as soon as the boats came
near — for all of our friends, Mr.~Seward, the Admiral, Mr.~and Mrs.~Seward, the
young officer, the civilians, the band and marines, were each and all accommodated
as we were, in flat‐bottomed boats, and were also pushing their way up the Pei‐ho.

The river was turbid and very tortuous, winding in and out between flat, yellow
claybanks, after the manner of the Charles, only through what a different country!
No green pastures, nor pleasant groves, nor cheerful homes, nor schoolhouses,
nor churches to be seen here, but only occasionally low stone or adobe houses, or
clusters of houses, with patches of cabbage garden growing around them, and no
trees. Whenever there was a little breeze, the Chinese sailors hoisted the cotton
sail, but the greater part of the time we were hauled along by means of a tow‐rope,
fastened to the top of our masts, and drawn over the shoulders of the sailors.
When they drew up to the bank for the purpose of hauling, we sometimes took
the chance for a run on the soggy tow‐path to get warm, but our boats were often
so far apart that we found it safer to keep in the cabin, or, if it was not too cold,
to sit on the little deck.

We all managed to come together at meal‐times, and we had one boat fitted
up for a kitchen, and another for a dining‐room; and these meetings were the
bright cheerful moments in the dreary and monotonous day.

We travelled at the rate of two miles an hour, often doubling back on our
course by the river turnings, and it was rather dull, especially one day when a
cold rainstorm struck in, and streaked our scarlet blankets. I tried in vain to cheer
my companions by reminding them that “patience and perseverance brought a
snail to Jerusalem.” I never was told where that snail started from, though I have
often in fancy followed his course all the way from my schoolroom window to
the very gates of the sacred city, and “in my mind’s eye” his journey was not very
unlike our voyage up the Pei‐ho.

After three days and nights we disembarked at Tung Chow, to start again
fresh for Peking. Tung Chow looked like all the other mud‐colored settlements
we had seen on the way, except that it was much bigger, a great town indeed,
with hundreds of little narrow, noisome streets, and thousands of inhabitants, all
of the men and boys of whom came down to see the landing of the American
Armada. Here, we ourselves saw our marine guard for the first time drawn up in
line, and starting on their forward march. We had no means of knowing how the
Chinese mind was impressed by this military display, for their faces never betray
an impulse, thought, or emotion; but the spectacle of a company of American
seamen and marines, in full sailor outfit, commanded by officers in full dress
uniform, epaulettes, and belts, swords and plumes, all mounted on donkeys so
tiny that the marines’ feet touched the ground on either side and had to be tied
into the stirrups to prevent the donkeys walking away from under them, was
amazing, and the effect may have been to discourage the Tien Tsiners and Tung
Chowers from attempting any further molestation of a people who could throw
such extraordinary troops into action against them.

The Chinese, a solemn people, are invariably impressed by dignity, while
they hold the ludicrous in high contempt. Hundreds of Chinese men and boys
surrounded us, and no twitching at the corners of the firm mouths, nor lowering
of the stern eyebrows, betrayed which way they were affected by the appearance
of our military; but our own feelings, and almost uncontrollable impulse at sight of
the donkey‐mounted marines, supplied us with an inkling of what their thoughts
might be. The band refused the mount; but they tied their instruments on the
donkeys’ backs, and walked beside them. The bass drum was slung between two
mules, and the trombone and trumpets found suitable resting places among other
bones and brayers.

This was to be the last stage of our journey to Peking, and we left Tung
Chow early in the morning. Some guards, or officers, had been sent down by
the Government to meet Mr.~Seward, the Admiral, and the marines; and these
Chinese captains were dressed in a curious dark uniform, with enormous white
disks painted in the middle of their backs. Our interpreters told us that the disk
inclosed the word “valor” in the Chinese character, which was inscribed in this
way to give courage to the soldiers who marched behind, when they went into
battle, and to give greater triumph over the enemy when they came out. We
could not quite comprehend the philosophy of these tactics, but earnestly hoped
that the Chinese soldiers would keep their sign of “valor” before the eyes of our
marines, perpetually.

When the music and military had fairly got under way we — elders, ladies,
and civilians — started, each in a Sedan chair borne by coolies, who walked with a
strong and energetic jogging step. We had some twenty‐five miles to travel before
reaching Peking, over an indescribably bad road. There are no Government roads,
no highways, no mail routes as we know them even on our wild frontier. The
ways in China have been trodden for hundreds of years, and the very road we
took was built three hundred years ago, on an embankment forty feet wide, and
twenty feet high. The soil — heavy clay — is subject to the overflow of the river,
and the Government determined to have one good, safe road. So this embankment
was built, and paved with hewn granite slabs as big as horse blocks, and when
they were first laid it was doubtless a good smooth surface. But now time, frost,
and incessant use, have dislodged and broken these granite slabs, leaving deep
holes and pitfalls most dangerous. No wheeled vehicle could possibly pass over
the causeway, which has deteriorated from a famous Government highway to a
miserable footpath. All day long we trudged over it, or were carried in jolting
chairs, seeing nothing but the flat, leafless plain, the sordid, clay‐colored houses
destitute of every appearance of comfort or thrift, with occasional glimpses of
the stern, inhospitable‐looking inmates — all under the depressing atmosphere of
a dull November day.

At last, however, as the snail neared his bourne, so we now saw the walls,
towers and domes of Peking in the distance before us. Coming to a cluster
of houses larger than usual, we found messengers who had come to meet us,
bringing words of welcome and fresh, strong men to relieve our tired chair‐bearers.
The American Minister’s wife, Mrs.~Low, had most thoughtfully sent her own
horse, knowing from experience how glad some lady would be of a saddle after
that hard day’s march. This horse fell to my share, and I was soon seated in
a comfortable American saddle, and found my steed to be a beautiful, spirited,
though gentle Arab, who arched her graceful neck, threw back her head, eager to
bound forward the moment she felt my weight. Another horse had been sent, and
a gaily caparisoned donkey decorated with a silver bell, and these were allotted
to our two civilian friends who proposed accompanying me. So we started, with
a guide who seemed there for the purpose — and indeed an English‐speaking, but
dark‐browed Chinese, had told us, with an air of authority, to “follow the mafoo.”

A mafoo, we learned afterwards, is a sort of non‐commissioned dragoon,
whose place it is to precede foreigners, or persons of consequence, and by dint
of shouting and pushing to make way for them through a crowd. There is one
mafoo attached to each legation, and we now followed this one, turning down
from the embankment into the plain below, and soon our animals’ hoofs were
striking quick and fast on the hard clay, the fresh wind blowing in our faces as
we galloped on, feeling all the exhilaration which that inimitable motion inspires.

My donkey‐mounted companion found his little beast fresh and sturdy, and
kept up very well. The other animal proved to be hard‐mouthed and vicious, and
would mind nothing but the curb; and whenever his rider applied that, the horse
would start back on his haunches, or stand stock‐still, while the moment he was
released he would take to his heels and run away. My Arab would have followed
him like a swallow, and passed him like the wind, but I could not leave my other
friend behind, and besides he was riding quite as fast as I cared to ride.

The mafoo seemed bent on some sort of pony‐express expedition, and galloped
ahead, never changing his pace, nor looking back. He rode a fiery little black
horse, and wore a big red tassel in his cap. By the gleam of this tassel I could see
him, through the thin gray air, a long distance, and thought nothing of it, feeling
more than safe in the care of my two American escorts.

After a while we came to the great gate of the wall of the city of Peking. Our
mafoo passed through it, and we lost him in the crowd, for there were many horses
and mules, camels and donkeys, waiting their turn to pass through this “needle’s
eye,” into the great mysterious city. When our turn came we hurried through,
expecting to find the mafoo and our friends beyond, when to our amazement we
found ourselves in the midst of a dense, hurrying crowd, which reeled and swayed
with seemingly no special purpose save to push and yell, in the mud and slime of
that fearful place, but saw no sign of friend, or guide, or any other familiar thing.

Several streets intersected at this point, and one of them rose to a slight
elevation. Straining my eyes, I imagined that I saw the bobbing red tassel of
the mafoo, as he cleared his way through the crowded passage and raced up the
hill. In a moment my friend who rode the fast horse, and who also spoke a few
words in Chinese, started after him, and was soon lost to our sight, while the
donkey‐rider and I more humbly followed.

We had struggled on for perhaps a hundred rods from the gate, when we saw
our friend tearing back, and halted to wait for him. He said there must be some
mistake. He could not make the mafoo stop, and when he shouted to him the man
put spurs to his horse, waved his arm wildly and pointed on. “I am going back,”
said Mr.~Middleton, “to find our friends, and the guard. Stand here, till I return.”
He was soon out of sight again in the opposite direction, and we were alone in
the midst of the swaying mob. Hundreds of strange, stern faces, scowling dislike,
surrounded us. They jostled against our animals’ legs, and pointed with jeering
looks at our helplessness. There was not one woman in the crowd; not one familiar,
sympathetic, human sound, not one smile, nor good‐natured look, greeted us.
My beautiful Arabian bay Fatima held her head high, with nostrils distended,
and allowed me to stroke and pat her smooth neck, but took no notice, save by a
disdainful toss of her dark mane, of all the tumult about her. The intelligent, sleek
little donkey grew nervous and restless. For ourselves, my friend and I spoke no
word to each other, but we had many strange thoughts as we stood waiting there.
The idea of being lost in Peking was too much of a nightmare to be acknowledged
for one moment, and we shuddered in silent and vague apprehension.

After what seemed to us a long time, Mr.~Middleton returned, looking downcast and haggard. He had been unable to find any of our party, or to discover
what had become of them. He said there seemed to be many gates, and he was
not sure that the mafoo had not purposely led us astray.

While we were considering this predicament, the suspected mafoo reappeared,
scowling, from over the hill, and beckoned to us. There seemed nothing else to do,
and we followed, — Mr.~Middleton ahead, Mr.~Rodman behind, my dainty Arab
picking her way, and carrying me, even there, like the gentle thoroughbred she
was. But what a way she had to traverse! The main streets in Peking are wide, but
they are paved like the causeway we had just passed — with great, slabs of granite,
worn out, and here and there displaced. The water, or traffic, has made a deep
muddy ditch on either side of this pavement, which the houses face — and the
houses are low stone or adobe buildings inclosed by walls, so that no doorway, no
window, is visible anywhere; and nothing more forbidding in the way of human
habitations can be imagined. We seemed to be in the commercial part of the town
at first, and merchandise, with nauseous impedimenta of every description, were
piled up in the streets, while buyers and sellers screamed their Chinese “puts,”
and “calls,” on every side.

On and on we went, out of confusion into more bewilderment, for we seemed
gradually to be turning off from the highways and into the by‐ways, up hill, from
which we caught glimpses of the wide plain of closely‐packed houses, relieved
here and there by great overtowering edifices, which we learned afterwards were
temples, and down dale, into some dark passageway, from which not so much as
a roof was visible above the threatening wall.

The mafoo, always riding at full speed, beckoned us onward and we followed
him helplessly. Often he and my well‐mounted countryman were out of sight,
and I stood alone, waiting for my other friend to come in view, while I listened
breathlessly for the sound of his donkey’s bell. Now I found what it was to have
made friends so quickly with a high‐spirited sensitive creature like the peerless
Fatima. She seemed to perfectly understand the situation, never demurred at
waiting for the tired little donkey, but often, when I was too anxious to know
that we were outstripping him, stopped of her own accord and waited until the
lingerer came into view.

Among other dread sights we passed and met long camel trains, heavily laden,
and winding their way through the dingy alleys, literally led by the nose, one
driver to every six or eight camels fastened by rings and cords drawn through their
noses. One does not like to think ill of camels, those patient, long‐suffering beasts
that look so picturesque under the palm‐trees, in pictures of oriental landscape;
but a near acquaintance with them is very disillusioning. They are both sly and
vicious, and become terrible when met in close quarters.

We turned, suddenly, into a narrow sort of alley, repulsive beyond description,
and here Fatima drew back — sniffing, prophetically. I urged her a little, and she
went forward, but presently I saw that we had to meet a long line of camels,
heavily laden with crates of tea, each about the size and shape of the “pressed hay”
packages so common in America. I could not imagine how we could pass them,
and yet I feared to turn about, even had there been space enough, which was
doubtful. Fatima sprang close to the wall, drawing her little hoofs and slender legs
almost under her. Taking my cue from her, I leaned against the dingy adobe mass,
while the long line filed their tea‐crates past us, swaying their heads and long
matted manes from side to side, and grazing against us as they went. Each one
eyed us from small evil‐looking eyes, with a malicious glance which suggested a
longing to strike out a ferocious blow from one of those powerful, noiseless feet.
But their glances were met by looks of scorn and defiance, on the part of Fatima,
mingled perhaps with a little fear, for she evidently knew our danger. With her
body fairly flattened against the wall — and mindful not to hurt me — she laid her
small ears, which were never quiet, close back, and turned her head toward the
camels. Her nostrils dilated and reddened, her lips parted and the fine squarely
set teeth showed between, while her enormous vigilant eyes were fixed on the
camels, and flashed an “at your peril” look at each one as the interminable train
slowly wended its uncouth way past us, leaving us both quivering together as
one poor aspen leaf in the autumn wind.

How far we went in that distracted ride I cannot tell. Peking is divided into two
parts; the Chinese city and the Tartar. The latter is closed to foreigners, without
special permit to pass the wall. The Chinese city is laid out in a parallelogram of
about fifteen square miles in extent; the Tartar town has an area of twelve square
miles. Scattered about, in all directions, but each one isolated and alone, stand
the temples, pagodas, palaces, and edifices of learning, the latter deserted, and all
looking lifeless and unused. An ugly stream, too, winds through the murky plain
which this incomparable capital covers, and commemorative bridges have been
built at intervals across it. There are said to be about two million inhabitants. I am
sure that in our apparently purposeless gyrations — we followed every point of
the compass and returned on our tracks at each different angle of it — we passed
each one of the structures which the guide‐book describes, and looked the entire
male population of Peking in the face; but no one attracted us, and the “Temple
of Great Happiness,” and that of “Extensive Peace,” were equally indifferent. The
“Hall of Intense Thought” could never have held a more concentrated company of
puzzled inquirers than that little band of Americans now rushing past it, without
one glance. The “Palace of Earthly Repose,” with all its delights, could not lure
us, since it did not hold our friends under its closely guarded walls, nor furnish
us tidings of their safety, while the “Gate of Heavenly Rest” was only a cruel
mockery to us, since we followed a flying mafoo.

Benumbed with cold, and stupefied with terror, I had long since ceased to guide
Fatima, though the rein still hung in my nerveless hand, and as if enlightened by
some magnetic current, the gentle creature seemed to know that she must not
only guide herself, but that she must keep the difficult equilibrium for us both in
that frantic giro, over grimy, slippery, dangerous roads, in the fast failing light of
a dim November day.

The animals all began showing signs of fatigue; even the mad mafoo slackened
his pace. My companions drew nearer to me, but we had long since ceased
speaking, not venturing utterance of our discomfort, nor of the apprehensions
which were threatening us with such unspeakable anxiety and gloom. Suddenly,
and with no sign of warning, there floated high above our heads that banner,
all blue, and starry, and “streaked with morning light,” which is the symbol of
freedom’s own “chosen land,” and ours. Yes, there were the “stars and stripes,” and
we were soon passing under the gateway of the United States Legation. Friendly
arms lifted me out of the saddle, and carried me into the house, when it was found
that I could not walk; gentle home‐like voices gave kindliest welcome, and asked
to have tea brought. But before a moment of rest or a comforting draught could
be taken, I was assured that my sister, Mr.~Seward, Admiral Rodgers, and the rest
were coming, and very soon they arrived, one by one, as tired and as anxious as I
had been.

It was explained the next day, that the authorities, having heard of our extraordinary array at Tung Chow, decided it would be too great a risk to allow
the whole company, band and marines, to pass through the city in mass, as the
excited masses might construe such a martial display into a menace, and therefore trustworthy guides had been dispatched to break our little company up into
detachments, and to bring each through a different gate, and by winding ways, to
the Legation. The mafoo proved himself one of those rare servants who perform
their duty too well.

If we had been advised of these precautions for our safety, our nerves would
have been spared a shock, which days of rest, cheerful society, and wonderful
sights could not at once restore; but it was agreed by all that the graceful Fatima
had earned a gold bit and silken reins and what I am sure she valued more — endless
petting, and great praise, for the intelligence and fidelity with which she had
brought her mistress’ guest safe through the perils of Peking, and to the American
compound.

\chapter{The Great Wall of China.}

There are said to be something like fifty thousand characters in the written
language of the Chinese. I am sure it would take them all to fully describe the
queer sights and strange customs we witnessed in Peking during the few days we
rested there, at the cheerful United States Legation, before making our final start
for the Great Wall.

The anomalous impression I received of the exterior of the town in my memorable ride, was intensified as I came to know something of the interior life of
Peking. My sister and I felt like two Chinese Alices in Oriental Wonderland when
we came to visit some of the people who live in those strange, inhospitable‐looking houses, their own homes, for it seemed as if all the pictures we had ever seen
on Chinese porcelain had come to life and the figures were now stepping out of
their slippery state to greet us.

I had never known before that the twisted trees, contorted objects and queer
architecture painted on Chinese punch‐bowls and platters are not droll caricatures,
but the Chinese representations of Chinese art‐ideas in the actual everyday scenes
of Chinese life. The grotesque figures which they paint on fans, or screens,
are all well‐known historical characters; heroes of fiction, or deified saints and
philosophers, and each one carries to the Chinese mind its peculiar traditional or
romantic association.

There is very little picturesque scenery in China, and the few hills, streams
and valleys which lovers of natural beauty have discovered, have done duty in
decoration for hundreds, perhaps thousands, of years. But these outlines, made
familiar by repetition, have a different meaning when the fact is explained that
the skillful Chinese landscape gardeners have made innumerable miniature copies
of these few bits of scenery in the court yards which are enclosed by the inner
walls of all the houses of the better sort. These courts, a few feet in extent, oblong
or square, are laid out in little mountain ranges, showing caverns and lakes, trails
and ravines, on every side.

These tiny rockeries, outlined again on teacups, for example, seem to us
pictures of the real peaks and torrents, and, in the same way, a toy bridge, arching
a toy stream, carved on an ivory card‐case, appears the representation of a huge
aqueduct, so exactly is the Chinese eye adjusted to their precise rules of perspective
and proportion.

Again, at Peking afternoon‐tea, we could easily have fancied ourselves a
painted part of the crooked pageant “toddling away,” to the seven‐storied pagoda
in the garden, which was not as high as our heads, though perched picturesquely
on the edge of a carved soapstone precipice; while the main “ancestral hall” on
the first landing of it would not have been big enough for two penny jointed dolls
to dance a “Highland fling” in.

Our illusion was complete one day when we were seated in state on the only
bench in a little artificial coral grotto. Our hostesses, sweet, bright‐eyed little
ladies, dressed in gay‐colored silks, and with many jewels, great bunches of fresh
chrysanthemums fastened in their smooth black hair, having “litty littly feet,” and
their finger nails protected by wrought gold cases, two or three inches long, were
arranged, also with great stateliness, in a circle round us in the court outside
the grotto. After some moments of silence, during which they curiously eyed
our sombre travelling gowns and our ornaments, which were only the useful
appendages of watch and seal, two of the Chinese ladies took each from her pocket
a round covered box made from a dried gourd, elaborately carved and fastened
with silken cords, from which boxes, again, each gentle lady produced a sprightly,
large black beetle and proposed a fight between the fierce game creatures for our
amusement. The scene of our vision seemed to change from “Wonderland” to
“Lilliputia,” when the interpreter explained that these black, glossy insects were
of a rare and special breed and trained for the “ring;” that their successful combat
was a matter of pride to the owners of the victors, while the sport was a favorite
diversion among the Pekingese ladies of rank.

Dismissing all fancies, we naturally shrank from witnessing a pitched battle
between these ravenous little scavengers, and begged the interpreter to explain
to the ladies that as active members of the “Society for the Prevention of Cruelty
to Animals” we should not enjoy such an entertainment.

The interpreter, a very clever and earnest Chinese youth who had been educated at a missionary school and knew English well, answered that he could
not translate what we had said so as to give the Chinese ladies any notion of our
meaning.

“But \emph{you} understand?” said I.

“O, yes,” replied the interpreter, “and I have heard of your Society; it is an
effort on the part of the more intelligent classes of the West to do away with
the monstrous practices of bull‐fighting in Spain, fox‐hunting in England, and
horse‐racing in America, which are the remains of the more heroic gladiatorial
sports of your ancients with wild beasts. There is nothing analogous to such
customs in the whole history of our civilization, and these ladies would only be
horrified to hear of the cruelties your Society protests against, without being
able to comprehend the great need of humane influence in your barbarous social
conditions.”

After this elaborate, and somewhat humiliating explanation, we declined the
pleasure of the proposed coleopteratic performance, and Madame Yang‐Fang and
her friends doubtless thought us two very prejudiced ladies “afraid of beetles.”

But our visits to temples and to teas in Mandarins’ gardens were cut short by
our purpose to see the Great Wall, which was not altered notwithstanding our
experience of Chinese travel gained in the journey to Peking.

“Every schoolboy knows” and every school girl too, in America, that this
great structure was built by the Chinese two hundred and fifteen years B. C., for a
defence against their enemies. It served that purpose perfectly for many hundred
years, until the Northerners grew stronger than the Chinese themselves, crossed
the wall, overthrew the reigning dynasty, and put a Mantchoo on the throne at
Peking, whose descendant is the ruling monarch to‐day of China.

The wall is left, a perpetually decaying monument to the past force and
grandeur of Chinese and Mantchoorian power alike, and the most stupendous
line of defence that has ever been constructed by the hand of man.

The Great Wall is one thousand five hundred miles long; and the point at
which we purposed seeing it, Nan Kow pass, is forty miles northwest from Peking,
across the first spur of mountains which divide China from Mongolia, and two
thousand feet higher than the sea. Having therefore to prepare as for a journey
to the forty‐fifth parallel of north latitude, some forethought was necessary to
provide suitable equipment.

Our numbers dwindled perceptibly as the privations sure to be encountered
became known. We naturally did not invite the band and marines again, and not
one of the gallant young navy officers could be induced to face the discomforts of
the exploit.

Mr.~Seward and Admiral Rodgers did not fail, of course, and the two civilians
were on hand again. We were, as before, three ladies, and, all told, a party of ten
and some half‐dozen servants, for we took food and cooking utensils.

The roads are so bad in Northern China at that season that travel on horseback
is impossible, and Fatima was not to be thought of. Instead of horses and saddles
therefore we were provided with donkeys, donkey carts and mule litters for the
journey. The Peking cart, their only variety of carriage, is a small covered box
without a seat in which one person can half recline in a cramped position. This
box of torture is fastened without springs to two low heavy wheels and the driver
sits on one of the long awkward shafts, his legs dangling at the side close to the
donkey’s heels.

The mule‐litter is a shade more comfortable, for the occupant can lie down
in it. It is also an oblong covered box, but is larger than the cart, and is hung by
heavy solid shafts to the backs of two mules, one in front, one behind.

Each one of these litters was furnished with a heated brazier, a mattress and
pillows, also with fur robes, blankets and rugs. We wore short warm skirts and
gay woollen blouses, a neat and comfortable costume suitable for the work before
us, but never seen again after we first put it on at the Legation, for, like Arctic
explorers, once having donned our outer garments we did not take them off again
until our return. Long fox‐skin coats, astrakhan caps, high white felting mandarin
boots, and large fur mittens constituted our daily and nightly attire for many
days.

Our progress was a labored walk of at best two miles an hour, during which
the jogging mules, fore and aft, were never known to keep step, nor either one of
them to step with any system of regularity.

The severe and incessant jolting made reading an impossibility. The scenery
was bleak and forbidding, and but for the fact of its being “a strange country for
to see” it would have been intolerably dull.

We sometimes met tall sullen‐looking men and boys dressed in blue nankeen
and sheep skins, the woolly side turned out, driving sheep, or black pigs, but
never by any chance saw a woman or a girl.

Trudge, trudge we went, across the hard rough uncultivated plain, rising
toward the distant mountains seen in gray outline against the horizon before and
around us. We stopped occasionally to cheer one another, or to change the weary
motion of the litter for a not less fatiguing ride on slow mule‐back, or to walk on
the hard stubbly ground.

So serious was our mood, that we were willing to be diverted by the disagreeable experience of one of our party, when in an unguarded moment he was
beguiled by the fruit of the only tree we saw, to bite into the rind of a pernicious
and puckery persimmon. And so we passed the day — only stopping at long
intervals to examine some curious old bridge fallen into disuse, or the ruins of
palaces or temples whose names and histories were past finding out.

Long after the sun had sunk to rest, giving us the yellow, violet and umber
tints of a northern sunset, we reached the end of our first day’s journey and
entered a native inn.

Our guides assured us that this low, thatched, adobe building had accommodations for one hundred travellers. If so they must put from fifteen to twenty in
each room, for there were only six rooms here, and the innkeeper was speechless
with astonishment when we engaged them all for our party of ten. These rooms
were entered from an open court which they surround, and which is the largest
part of the establishment. This court, which is also the stable, has stone mangers
stationed at convenient distances, and is always crowded with donkeys, mules,
and camels.

The Eastern idea of hotel accommodations is somewhat remarkable. At an
inn for instance, when the court is overcrowded and the rooms not full, the
animals are turned into the sleeping apartments; and when, on the other hand,
the chambers are all full and there is still space in the court, the native travellers
do not hesitate to camp out among the cattle in the mangers.

The window in our room had been originally glazed with greased paper, but
that had nearly all disappeared. The floor was a rough pavement which had
evidently never known a broom, much less mop or scrubbing brush. The walls
and ceiling were black and oily from smoke. The solitary piece of furniture in the
room was the kiang or bed, and that was stationary and built against the wall.
It was three feet high and covered the width of half the room — a platform built
of brick and plastered with adobe, six inches deep. An oven is built underneath
this bed of bricks, having flues which convey the heat to every part of the surface.
Here the native travellers spread a bamboo mat and pack themselves, sardine
fashion, across the warm surface of the kiang to sleep.

Our mattresses, furs and blankets were brought from the litters and laid on
the kiang, and with stars glimmering through our gaping window sashes and the
wind playing havoc with our covers, we buttoned our fox skin coats about us and
“laid down to pleasant dreams,” while the camels, cows and donkeys peacefully
chewed their cuds and rested at our very door.

The forms and customs of the Eastern caravansérai or inn, have remained
identical and unchanged throughout the records of all ages. It was not strange
that in our dreams that night in Northern China my sister and I were wafted back
to our distant home, where the voices of children whom we loved would so soon
be chanting the most beautiful of all Christmas hymns:

\begin{quote}
“Cold on His cradle the dew drops are shining, \\
Low lies His head, with the beasts of the stall,”
\end{quote}
for the homely scene of the wondrous story was made vividly real to us now,
by our ancient oriental surroundings.

We had four such days and nights before reaching Nan Kow, the entrance to
the pass, where we rested at an inn not nearly so good as the one I have described
fifteen miles from the Great Wall.

One could hardly look for any home reminder in this desolate place, but the
traveller’s unfailing surprise is in learning how small the world is.

A poor little man, or boy, we could not say which, wizened and deformed,
made his way to us and begged by signs and gestures to be allowed to go with
us as a guide, importuning us to read some “references” which he carried in his
boot‐leg. We read his carefully preserved slips of paper, and discovered by them
that some young American navy officers, more adventurous than those we had
left behind at Peking, had penetrated Nan Kow pass before us and furthermore
left there some words of encouragement and instruction to any countrymen who
might follow their example. They had written careful directions about the road,
and warnings against certain coolies, whom however we could not identify; but
the little hunchback, who was unmistakable by the description, was very proud
of one of these scraps which gave him high praise for honesty, intelligence and
fidelity and commended him warmly to the favor of American travellers.

This was all in the familiar handwriting of my old playfellow, and almost
kinsman, Lieut. Cushing, and it was like his kind and generous heart to leave this
testimony of the unfortunate coolie’s good service.

When little Ping, for that was his name, learned that we knew his gay patron
and good friend, he attached himself to us closely. We ordered food for him and
allowed him to join the guides although we fancied they tried to prevent it.

Leaving the litters at Nan Kow we took instead Chinese mountain chairs, very
unlike the safe chairs known to European mountain travellers. These were small
bamboo seats, having low arms, and a swinging stirrup for the feet. The back
was so straight that one felt in danger of falling forward at every step, as the
chair was lifted by bamboo shafts and borne on the shoulders of four strong and
nimble‐footed coolies.

We found the pass a wide dark ravine through which mountain torrents have
stormed and ploughed their way for centuries hurling down the unresisting peaks
and loosened bowlders and strewing the only path with a debris of rock and
timber.

Ice had formed, thick and fast, in every depression, which added peril to our
progress up the lonely way where bleak, grim mountains towered on every side.
These were all barren and tenantless except for the grotesque and peering effigies
of Chinese deities fastened in the crevasses and on every available pinnacle,
whence they seemed now to grin or scowl at us mockingly.

Long swaying caravans wound silently along at intervals, the camels choosing
their steps sedately over the roughest path. Their fierce‐looking Mongolian riders
were dressed in furs and wore tall yellow caps. Great silver rings dangled from
their ears and noses, and curious ornaments fastened their riding cloaks.

These camel trains with their riders were on the march to their far Northern
homes, heavily laden with merchandise from Peking. A woman, as a rule, rode the
leader, the men following meekly and carrying the children; for the Mongolians
are the only people in the East who allow women any active part in the affairs
of life. These strange people go even to the other extreme, for among them all
the family labor and responsibility is borne by women alone. They not only
drive and feed the camels, but milk the cows, make butter and cheese, draw
water, gather fuel, tan skins, and till the soil. The men are a stalwart race, but are
trained exclusively to deeds of daring in warfare, and to wonderful horsemanship.
Their haughty bearing, flashing eyes, and high color, like carmine on copper, are
distinctive features which do not belong to the Chinese Mongolians.

Amid such scenes we reached the end of our journey.

The Great Wall of China was to us the precise realization of the picture of
it in the old primary geography. A solid granite structure, twenty to fifty feet
high, and really wide enough for “six horses to traverse it abreast,” stretching and
winding in either direction over the crests of the mountains and down the slopes
of the plains, as far as the eye can follow and for leagues and leagues beyond, the
distance broken at every mile by high square watchtowers of defense.

It was four o’clock when we reached the Wall, and we were not long in
ascending the well‐preserved granite stairway leading to the parapet, and in
walking up and down, across and around, through every accessible part of the
great fortification; but, like many objects in life long looked forward to and hard
to be attained, our visit to the Great Wall was quickly over, and had become an
episode of the past.

Turning from our last view of the apparently trackless wilds of Mantchooria,
we faced China, and prepared to retrace the precarious pathway down the pass.

It was soon dark and there was no moon; the air was sharp and cold; we
were too tired and apprehensive too, to talk, and one by one our chairs became
separated beyond speaking distance. Before I was aware of any change however,
I found myself quite alone with my four coolie chair‐bearers, who skipped and
tripped along at a rate which dismayed me. I gradually realized that instead of
following the path downward, by which we had come, they now disdained any
path and were bearing at right angles over the hills, toward the mountains, as
fast as their sturdy legs could carry them.

I knew no word of command, and when I motioned to them emphatically to
put me down, or to turn back, they laughed derisively and hurried on the faster. I
tried to cheer myself with the hope that they were taking a shorter route, and
looked anxiously for some sign of my friends, but with no success. We dashed
on, recklessly it seemed to me, and higher and higher at every step, and faster
and faster, in the deepening darkness, grim mountains looming up on every side
crowned by the ever‐present line of the gloomy defiant Wall.

At last we seemed coming to a settlement of four squalid huts huddled together
which we certainly had not passed in our upward journey, and which were
surrounded by a crowd of Chinese men and boys, their figures standing out black
and savage against the blazing red light of a cornstalk bonfire.

All the distracting bewilderment I had felt at Peking in presence of that crowd
of strange, unfriendly faces and labyrinth of lanes and ditches, was trifling in
comparison with the terror which now seemed freezing me to a block of ice. It
was not possible to imagine that among that crowd of thousands there were not
some kind human hearts; besides I had had my two friends with me then, and the
dear Fatima who reassured me at every step with a sense of her willingness and
power to carry me safely through any difficulty.

But here I was alone, at night, in a desolate and hostile country, surrounded
by fierce natures who hated my very race with relentless vindictiveness. The
horrors of the Tien Tsin massacre had become realities by the many descriptions
heard since our arrival, and I was indeed completely at the mercy of these coolies,
who knew it only too well. They no longer showed the deference assumed in
the presence of the American party, and now, springing forward, dumped my
chair unceremoniously on the ground, far away from even the poor comfort of
the blaze.

This action gave me time to take in the situation, from which I failed however to
gather one ray of hope. That they had captured me for hope of ransom money was
the best fate that I could picture. The coolies seemed to be enjoying themselves,
unconcernedly lying on the ground and talking at leisure, although it had been a
part of their bargain when we engaged them to get us back to Nan Kow, from the
Wall, in two hours — that is at eight o’clock, and it was long past that hour now.

Remembering that the Chinese of all classes are only impressed by dignity,
reserve and self‐control, I summoned all the nerve I possessed, drew my eyebrows
together sternly, and awaited my fate in silence.

This indifference seemed to have the effect of amusing the coolies hugely,
and of puzzling them too, as they came nearer — powerful, dark‐skinned men,
ragged and unclean beyond description. They pointed to my white boots and
fur cap, signs of rank, and insisted with wild gesticulation upon my leaving the
chair, which I steadily refused to do. A boy then drew off one of my mandarin
overboots, and my own buttoned boot under it seemed an object of great curiosity.
Another brought a nauseous pipe, urging me to smoke it, while another twitched
off my great fur mitten. If I had been less terrified, I could have smiled at the
consternation which followed their discovery of what they supposed to be the
color of my hand, encased in a dark kid glove. They held it against my face, and
when they found that the brown skin peeled off readily, they began a capering
sort of war‐dance with shouts, which to my overwrought imagination seemed a
possible preparation for my final destruction.

One of the most active ones came close to my chair and, snatching off my
astrakhan cap, ran his long dark fingers through my hair, teasingly exultant when I
shrank involuntarily from the rude contact. My eyes grew dim, mountain barriers
all disappeared before me, the coolie figures and the fire seemed shrinking away,
when I suddenly became conscious of a shrill and well‐known sound, which came
from the interior of one of the dark and hitherto silent hovels, and which for the
moment also diverted my Tartar tormentors.

Quite naturally, for it was the sound at once the most imperious and appealing
that can assail the ears of man — the voice in which all hungry babies, the world
over, “continually do cry,” and by which irresistible appeal they as continually
receive some expression of the tenderness they inspire. This particularly lusty
outcry of the Mantchoorian mountain baby was followed by a soft crooning sound,
the perpetual lullaby‐note of a mother’s loving care.

All the coolies harkened, and one youngish‐looking fellow, not one of my
chair‐bearers, gave the new comers to understand that the cry came from the
lungs of his new little son. Just after that moment of respite I heard a familiar
voice shouting my name, and then calling to the coolies with authority. It was
one of the friends who rode with me at Peking.

It seems that as soon as our party came to a halt at the only resting‐place
on the way, my chair, its occupant, and the coolies were missed, and my friends
started in pursuit in different directions. Mr.~Rodman took the Admiral’s good
man Andy and little Ping, who begged to go and who led the way straight to the
dismal clump of huts where they found me.

We never really knew what the coolies’ motive was in going to that out‐of‐the‐way
place. They were meek and humble enough after Mr.~Rodman came, but they
scowled at poor Ping who kept close to us and seemed afraid of them.

The remainder of the tramp down the pass was made quickly and in silence,
though we kept our chairs in close ranks now, and I did not soon forget the ordeal
of those few hours.

Another night at Nan Kow inn, another start at daybreak, saw us winding our
way down the mountain slope, past the Ming Tombs with their avenues of giant
stone monsters, and the beautiful ruined porcelain roofs and marble bridges of
the Yuen Min Yuen palaces, looted and destroyed not many years ago by French
and British troops.

Ping went with us as far as a place called Champing‐Chow, for he did not live
at Nan Kow, and only happened there by chance the day we came. It is needless
to say that we supplemented Lieut. Cushing’s praises with our own thanks, and
that poor Ping went home with a hatfull of “cash,” Chinese coppers, which made
him rich Ping in the eyes of his coolie neighbors.

After three more days and nights we found ourselves again at the gates of
Peking.

Our last day was passed in some anxiety, for by the municipal regulations of
the Flowery capital the gates of Peking are locked and barred at five o’clock and
no wayfarer is allowed to pass them. We implored the drivers to “walk a little
faster,” unable to endure the thought of another night at a Chinese inn.

As we trudged along, the drivers goading the mules with a sharp stick and
urging them with a reproachful syllable sounding like “tut, tut,” our eyes on the
ground and our spirits dull with hunger and fatigue, all thoughts were suddenly
lifted by a musical whirr in the air above our heads as a flock of carrier pigeons
flew over. They were near enough to show the thin rose‐colored packages tied
under their outstretched wings, but we could not account for a harmonious chord
of music and a succession of soft notes, unlike any bird’s, which seemed floating
with them as they flew. The explanation of this mysterious sound gave us a new
appreciation of the æsthetic feeling of the beetle‐fighting Pekingese. It seems they
have the custom of fastening little carefully‐toned reed whistles to the pigeons’
tails, which as the birds fly through the air produce a sound as soft as that of
an elfin’s flute. These pretty letter‐carriers are used in all practical affairs and
commercial correspondence, but it pleased us to fancy that some of the flock we
heard in their musical flight were “bearing the sighs of a lover, or bringing him
news of his fair.”

We cleared the dusky gates as the gongs were striking five o’clock, and entered
the Legation at the hour when guests were arriving for a diplomatic dinner.

It was a scene never to be forgotten as our caravan slowly entered the court,
one dusty, lagging, old Chinese mule‐litter after another, and deposited its tired
occupant on the pavement.

The band was playing “Hail, Columbia,” and a line of handsome young navy
officers stood ready to welcome us back, arrayed in full uniform and looking like
beings of another realm by contrast with us, in their spotless linen and dainty
gloves. Quick to help us to alight, these fairy‐like princes led us into the hall of
the Legation, when all dazed and dingy as we were we did not fail to express to
them the very deep satisfaction we felt in having seen the Great Wall.

Mr.~Seward was so well pleased with this achievement, at his age of seventy
years, that he at once caused a telegram to be sent to his sons, and to my father,
at home, to say that we had seen the Great Wall and were back again safe and
sound. The message was carried by hand nearly two hundred miles across the
mountains to Kiakhta on the Russian frontier, and sent by way of Russia and the
Atlantic cable to America. It was ten days reaching our friends at home.

\chapter{A Coaching Party in Java.}

The change from bleak stretches and gloomy towns of Northern Asia to the radiant
verdure and palm‐thatched villages of Java, is one which any traveller might be
glad to make.

It is like a pantomimic transformation‐scene to turn from the turbid streams,
trackless wilds, and taciturn, inhospitable people of the mis‐called “Flowery
Kingdom” to the crystal fountains, beaming vistas and gentle human sympathy
which smile welcomes on every side in the most enchanting island of the tropical
Pacific.

Our journey from Peking to Batavia was like going from New York to Lima, a
voyage, if made without stops, of about twenty days, by seas and gulfs and straits,
through the temperate zone and across the equator.

Sailors, the world over, make a great ado over a shipmate’s first trip across
the torrid zone, and celebrate the event by all manner of ludicrous practical jokes
which often cause the “green horns” discomfort and chagrin. It is a sailor’s pride,
however, to take all jokes in good part, and at such times the spirit of fun and
frolic seems to take possession of the ship.

We crossed the line in the middle of the night, and my sister and I were sound
asleep when the reckoning changed. We had been forewarned by old travellers
of what might happen, and like wise birds were on the lookout for any snare laid
for the unwary.

All went well during the early morning hours and nothing unusual happened.
But at breakfast everything we tasted seemed smothered in cayenne; when we
asked at last for some ginger marmalade, a specially inoffensive Chinese preserve,
we found it flavored with some new‐fangled and concentrated essence of Tobasco
sauce! Politeness had prevented us from remarking on the peppery quality of the
food at the Captain’s table that morning, but now, choking and sneezing we cried
out against the revolution in sugared ginger, when the Captain, a sober Dutch
wag, said, “O! do you not know that everything is much hotter on this side the
equator?” and we gulped down some iced water and played at thinking it a good
Dutch joke.

A little later, when off Borneo, our ship struck a monsoon, and for several
hours we were severely tossed about; but once passed the straits of Sunda we
sailed into the stormless sea which sweeps the shores of Java. We gazed from our
steamer’s deck across the waters which broke in dazzling surf upon the golden
sanded beach, and soon we too had reached the shore and fallen under the island
charm.

About one hundred and eighty years ago, and eighty years after the Dutch
had settled New York, that same energetic and frugal people came into possession
of the Island of Java, which they have governed ever since.

The first characteristics therefore of Batavia are comfort, order and cleanliness,
like those of Amsterdam, Rotterdam, and the Hague. The streets, straight and
broad, are shaded by double rows of oaks and tropical trees on either side, while a
clear stream of living water flows down the centre of every street and thoroughfare
for irrigating. These streams are mountain torrents which have been diverted
from their natural courses, gathered at intervals into deep pools, and directed to
a lake above the town, whence they take their rapid current through the city and
out to sea.

The houses stand far back from the streets, with gorgeous flower‐gardens
in front, and fountains, vases and statuary gleam among the flowers and foliage everywhere. These houses, built of white stone, or stucco, have stately
pillared verandas across the entire front which seem a promise of repose and
sober hospitality within.

The rooms are lofty and are arranged with a keen eye to comfort; floored and
ceiled in marble, they are cool as tropical rooms can be.

There is very little furniture in them, but what there is is suited to the climate.
Plenty of bamboo chairs and lounges, but no drapery, stuffs, or bric‐à‐brac, to
obstruct the air; the “cosy effects” and warmth not being counted desirable there.

The bed is often inclosed in a fine wire screen, a wire room within a room, into
which you enter from the chamber surrounding it by a little doat which closes
quickly with a spring. The uses of this big wire cage are the reverse of those of a
bird‐cage, this being made to keep little flying creatures out, an invention which
might be useful sometimes in that chosen home of the American mosquito, our
own New Jersey coast.

It is a cherished tradition as well as a recorded fact, that in the days of Dutch
colonization in America, “a passion for cleanliness was a leading principle in
domestic economy, and the universal test of an able housewife.” The same good
practice prevails with the Netherlanders who have made their homes just here
under the equator’s rim. The days begin before dawn, and the Dutch ladies in
the short‐gowns and petticoats of their native country, with big bunches of keys
hanging at their girdles from native wrought silver châtelaines, pass the early
hours in household duties and directing scores of docile Javanese servants who
march, and counter‐march up and down the houses and “around the yards,” armed
with scrubbing‐brushes and dusters, mops, soap and brooms, making war upon
any chance speck of dust or cobweb that may have escaped them the day before.

The children’s lesson hours are all ended at nine o’clock a. m. in this happy
land; they begin, however, in all well‐regulated families at four in the morning.
The hours from nine to twelve are given to children’s indoor playtime, while in
those hours the Dutch gentlemen go out to look after business or government
affairs.

After midday breakfast, the household retires for a three or four hours siesta,
and after this good nap, as the day cools, comes the universal swimming‐bath.
Quite half the space in every house is given to this necessary department of
comfort and health.

As soon as the sun goes down the whole town sallies forth bareheaded; the
gentlemen without exception dressed in white duck trousers and black evening
coats, the ladies and children in the lighter materials, and big and little have their
hands covered with white cotton gloves.

They go to a great open park where for two or three hours friends meet to
chat, children to romp and dance, while all enjoy the refreshing sea‐breezes, have
ices, and sherbets, and the gay music of a government military band. By ten
o’clock absolute, quiet reigns in Batavia, the lights are put out in the streets and
houses, and the town goes to sleep.

The bland manners and peaceful ways of this placid sea‐port were very pleasant after our cheerless experiences in the north; but notwithstanding the rest and
tranquillity of Dutch life and hospitality, we were so fascinated by the appearance
of the Javanese themselves that we longed to leave the foreign town and to be in
the country where we could know better the people’s national life and ways.

The Javanese are Malays, and the smallest in stature of all the Asiatic races. A
man much above five feet high is tall among them, and the women are proportionally small.

They have fine golden‐brown skins, soft dark eyes, regular features, and
straight black hair. Their voices and accent in speaking are soft and musical. They
have peculiarly pleasant smiles, and their manners are cheerful and refined.

Their social customs approach the heroic in a standard of dignity and considerateness. Rudeness is regarded a serious offence, and malicious expressions, or
slander, are punished by instant death.

The Dutch conquerors of this docile, straightforward, and high‐minded people
have by just dealing gained their confidence and respect, and the natives having
learned in this way to regard foreigners as friends, their feelings toward strangers
are confiding, and their ways are winning and sweet.

Before many days it was arranged that we should go to Buitenzorg, to pay
our respects to the Viceroy, whose summer residence is there, and afterwards to
proceed on a visit to a native prince and governor of a province, to whose palace
and domain Mr.~Seward and his companions were bidden as guests.

There being no railroads in Java we made our beautiful journey in the most
delightful way of travelling; that is, by private post‐coach. Our carriage was lent
us by a very kind Dutch gentleman, who was also the American consul at Batavia.
The vehicle was very like a drag, only not so high. It had seats outside and in, for
eight or ten passengers, and we were six without the coachman. Our baggage
was stored away in roomy boxes which fitted under the seats so as to be removed
with all their contents without the least inconvenience at the end of each day’s
journey.

The big wheels of our carriage were profusely greased and the coachman was
provided with a whip and lash, yards and yards long. We soon found to our great
satisfaction that this whip was used only for ornamental flourishes, and that our
Malay coachman was highly accomplished in the art of cracking and snapping
it. He furthermore had an astonishing way of whirling the lash in the air and
throwing it forward so exactly as to tickle the right ear of the off‐leader, to make
him shake his head and kick up.

Our heavy rumbling travelling carriage was drawn by six slender fleet‐footed
ponies, natives of Java, and as gentle and winning as the human inhabitants themselves. The ponies were gray and sorrel, carefully clipped and groomed, and thin
as racers. They stood about ten hands high, and their diminutiveness detracted in
no way from their perfect proportions, and they showed no disposition for pony
tricks. They were shod with only thin narrow iron rims without corks, and they
have no knowledge of the check‐rein torture. The little creatures came up soberly,
as if they understood the work cut out for them and were prepared to do it

Starting on a dead gallop at the crack of the whip and the word go, these
slender ponies never slackened pace nor broke step for five miles. They were
stimulated to the top of their speed by four native boys, of about twelve, who
wore short calico trousers, belts and handkerchiefs tied round their heads and
waists, who ran just in front of the carriage, cracking whips and shouting, their
nimble bare legs keeping pace with the ponies, or as often leading them.

We were sentimentally inclined to think this a cruel and unnecessary practice,
but we found it was the boys’ pleasure and ambition, as well as their trade, to be
champion runners. Occasionally, while the ponies were racing at full speed, the
boys, to please us, would spring to the step of the carriage, resting on one foot
and bracing by a handle made for the purpose on the side of the carriage. After
balancing in this position, like flies, for a few minutes, they would hop off again
and resume the race, the ponies’ gallop unbroken all the time.

At the end of each stage of five miles we changed both ponies and boys,
though none of the small creatures of either race showed signs of fatigue.

The road, to be sure, was perfect; a wide macadamized bed graded to an
exact line, while the scenery of tropical mountain, lake and plain, seemed to our
untrained eyes, accustomed only to the features of temperate regions, a succession
of enchanted visions of bliss and beauty.

We were driving directly inland, and therefore ascending the mountain slope,
though often our road followed some winding way each one more pleasing and
sublime than the last. Sometimes, across valleys smiling in the brilliant verdure
of cereal culture, or along slopes shaded with coffee orchards whose dark glossy
foliage and the different phases of bloom showing snowy blossoms, or fruit in
bright berries, purple, or scarlet, as the stage of maturity might be — threw into
softened shadows the delicate emerald of the little rice fields beyond. Or again
passing through a forest of foliage, never seen save in island tropics or in dreams
of fairy‐land, whose cool shades invited us to alight and explore. There were
laurels and oaks, a few and stunted, and added to these familiar varieties, were
the tenderer forms and hues of the bamboo‐tree, the acacia, mimosa, pepper,
palm‐trees and tree‐ferns.

The humid air promotes vegetation to such a degree that it leaves no uncovered
soil. In the twilight of these shaded forests the feet sink deep at every step into
fine fragrant mosses, reviving one’s early faith in elfdom and loyalty to King
Oberon in this the true land of his royal rule. Not only is that mossy carpet fitted
to the tread of fairy footsteps alone; who can look at the generous, smooth circles
of fungi, or “toad stools,” scarlet or yellow or glossy white, and doubt that they
have served the merry purpose of the “broad board,” of “Queen Mab’s” midnight
banquets?

Overhead, as if the perfection of leafage and blossoms were not enough of
beauty to have bestowed upon this favorite child, nature has here bedecked her
island‐treasure with an orchid‐bloom which partakes of all the charm of bird
and flower‐life too. Sparkling through the thick, dark branches these myriad
blossoms on their invisible stems sway in the gentle breezes, and live on the
scented air they breathe, shining out brilliant in colors, as they are various and
beautiful in form. And if the flowers here look like birds, surely the birds outshine
them in the brilliant colors of their plumage. Not only did we see the speaking
mina, raven‐black, with scarlet bill, and cuckoos of sweetest note, but creamy
cockatoos and stately peacocks in their native wilds, and dazzling nutmeg eating
birds‐of‐paradise, —
\begin{quote}
“Those golden birds, that in the spice‐time drop, \\
About the gardens, drunk with the sweet food, \\
Whose scent hath lured them o’er the summer flood,”
\end{quote}
while gaudy insect creatures on iridescent wings carry perfume in their pinions, which exhudes as they float, mingling with the sweet scent of moss and
herb and flower.

Out of a palm and tree‐fern forest across a plain, our road was lined everywhere
with hedges of heliotrope, lantania, orange flowers, and a filmy eglantine almost
tree‐high, bordering the splendid bloom of field and bower beyond. Then we
skirted the shores of a mountain lake, its dimpling surface bearing the fragrant
burden of white and pink and crimson lotus‐flowers.

Even the rugged mountain sides were made delectable by tangled masses of
rododendrons and sweet ferns, while everywhere little crystal torrents trickled
refreshing streams, rested in shady pools, or hurried in sparkling cascades, lily
fed and foaming, on their way to the sea.

The people who swarm the districts which one passes in this enchanting
tour only add contentment to the scene — they are all so gentle, so genial, so
sweet Husbandmen till the soil and cultivate the natural products, camphor and
dates, coffee, spices, sugar and rice. They are industrious and thrifty in their way.
Though very poor in goods, they have all they need or want: a bamboo cottage
thatched with fresh palms for a home, their provision‐house in the fig‐tree which
shades the doorway and in the banana‐trees beyond. At earliest dawn, boys climb
the tall, straight cocoanut‐trees, and test by some rule known to them the milkiest
fruit, which they bring down for the family supply. When the bananas are cut,
the marketing and cooking, the brewing and the baking, all are finished and done
for the day.

Twice in the year an important family outlay is made, and the winter and
summer outfits are provided; these being the two seasons of the tropics, the wet
and the dry, with very little variation in the temperature. When the forthcoming
change appears, the father and mother call in a pedler, for there are no other
merchants. After long and arduous “dickering” they buy a piece of calico, twelve,
fifteen, twenty yards long, as the case may be, at five cents a yard. They are
particular as to color. It must be, to suit, a good yellow, pink or green, in a large
stamped pattern of palmleaf or scroll. The next day, the father and mother cut
out, and sew, from this gay calico, an ample coat and trousers for the husband,
and a skimpy gown for the wife; and then such straight, serviceable lengths as
are left are made into the simple scarfs, or wraps which, worn singly, are the only
garments boys and girls have until they are grown. One or two big hats serve the
purpose of the whole family — the children never wear them.

Our few unavoidable mishaps on the way, like a sprung tire, or cast shoe,
brought kindest offers of assistance from the peasantry, who crowd the waysides.
The people always came smiling forward, and never begging, when we stopped,
offering hospitality, in cups of cool spring water and gifts of fruit, or bunches of
flowers.

A few hours of such travel brought us to Buitenzorg where the white marble
summer palace of the Viceroy of the Netherlands Indies stands embowered in
palm forests and bamboo groves, and reflected in the waters of a lotus‐blooming
lake.

We found the interior of this palace as cool and inviting as the outside was
impressive and quaint.

The Dutch Viceroy of Java was an Amsterdam gentleman. He had lived many
years in Java and, like all European residents of the tropics, had been forced to
send his children to their native country as they grew older to be educated. His
daughters, now about sixteen and eighteen years old, lost their mother when they
were very young, and they had been educated as English or American girls might
have been — speaking the French, Dutch, German and English languages equally
well; they were good musicians and accomplished in all the manners and ways
which belong to well‐bred young ladyhood from our standpoint

They now received us with the sincerity and gentle dignity which mark the
manners of “Manor house” life along the Hudson, and is guarded there as a relic
of the days of the “old patroons.”

Our Cosmopolitan New Yorkers cherish with jealous pride names which recall
men and localities famous in the history of New Amsterdam; but we do not find in
New York grandeur and display any trace of the frugal ways and homely customs
of the early Dutch settlers. Manhattan Island children have however preserved a
Dutch tradition, and Kris Kringle is as well known, wherever they go, in America,
as in Holland his native land, and it is believed also that he manages his queer
little team as cleverly along the flat roofs of New York houses as he does up and
down the gables of Amsterdam. The dear, jolly old saint goes to Java too and
fills the stockings of the small “good” Diedrichs and Barbaras there just as he did
those of their progenitors, who bequeathed the custom in our country from the
Knickerbocker time of long ago.

They say that coming so fast and hurrying through Java by moonlight Kris
Kringle and the reindeer are completely deceived by the locust blooms and jasmine
petals which at Christmas season strew the ground so thickly as to look like a
covering of softly fallen snow.

Our young hostesses told us this Colonial Dutch story about Kris Kringle,
and satisfied our curiosity about many customs of the Javanese, while the Governor‐General talked to Mr.~Seward of the difficult problems of colonial government
and politics.

They were untiring in plans for our enjoyment, and the young ladies took my
sister and me to drive in their own pony carriage, explored the palace with us,
showed us the great botanical gardens, and finally their swimming‐bath. This
was a deep basin, lined with marble, the size of a small pond, and deeply sheltered
in a bamboo grove where they could swim and dive like mermaids.

Our new friends were dressed in the native costume, which they found more
comfortable and convenient than their own, in day‐time; a sort of graceful skirt
folded straight round the figure, with a wide‐sleeved overdress, made of finest
mull — their hair light and free, their feet slippered, but stockingless.

As our first day at Buitenzorg was drawing to a close we listened to some
native Javanese music, which is more like a dream of sound than a reality. The
oddly‐shaped stringed instruments, the drums and gongs all soft‐toned and melodious, are not tuned, apparently, to a fixed pitch, and the gamut and compass of
our system seems quite unknown to the Javanese musicians.

Mr.~Seward was sorely puzzled when he heard this music.

The day before we set sail on our journey around the world, he had asked
me in San Francisco to buy a music book to take on the voyage. I was at a loss
to know what sort of a music book he meant — whether a collection of Strauss
waltzes to divert the young people, or of nursery‐songs for children, or a psalm
book for Sundays, or “Plaidy’s exercises,” to bring up arrears in five‐finger practice
for my sister and me. He explained that he wanted none of these, but that he had
all his life been longing to know music; that is, he said, he had wanted to read the
notes and to know the meaning and character of a score, as one knows at a glance
the character and purpose of a newspaper without reading it through. Music was
not taught in the schools as we have it now, when Mr.~Seward was a boy, and he
had felt his want of knowledge so great a disadvantage that long years ago he
had determined to use his first leisure hours in overcoming it. Those hours were
very few in his great and busy life. The long voyage before us, of twenty‐five days
across the Pacific, at last brought the opportunity; and I bought the “music book.”

He was then nearly seventy years old, but so well did he apply himself that by
the time we reached Yokohama his wish was gratified. He knew all the notes and
could understand any score. He had felt great satisfaction in this acquirement, and
he now had a corresponding disappointment at finding music practised which had
none of the signs, and followed none of the rules he had just studied so diligently
to know.

In the evening the marble halls were illuminated, and the Governor‐General’s
young daughters, now in gauzy French ball‐dresses, stood beside their father to
welcome the guests bidden to a banquet and ball at the palace. It was easy to see
that their tact and amiability aided him in this part of his official duty, even as
their society brightened his home and made the true happiness of his life.

The hospitality of Government House was heartfelt and delightful, and memory now lays stress upon the courteous customs of that home life, which although
passed in a palace, and in a strange far land, was not different from the home life
of our own country and all the countries of our common religion.

We would gladly have lingered notwithstanding the fascinating prospect
before us of a visit to the palace of a native prince or sovereign.

Before many days we had started again with fresh ponies and runner boys on
our way to Bandong where a Javanese ruler and his son were to make plain to us
the melancholy difference between a Christian and a Mohammedan home.

I will try to show you next how a Mohammedan boy‐prince, who seemed
neither brave, nor cheerful, intelligent nor free, in his father’s presence and his
own ancestral home proved himself to have some of those qualities and principles
which the Crusaders sometimes found among the Saracens, and which made them
worthy the privileges and duties of knighthood.

\chapter{The Prince of Bandong and His Son.}

The Prince of Bandong lived about fifty miles inland from Buitenzorg, on one of
the plateaus which break the ascent of the mountains of Java, where we were
going to visit him.

Up to this time we had met the Dutch conquerors in their tropical setting, and
had seen the effects of their gentle sway among the Javanese. We now expected
to see quite another phase of the life of those millions of human beings whose
knowledge of the great world is bounded by the sea‐washed shores of their island
home.

Eight ponies now drew our big carriage over the steep ascents, often aided by
men and oxen.

The increasing altitude carried us through climatic changes which worked
like magic in the vegetation. In one day we passed through all the various stages
of rice culture, which was like seeing the whole procession of a wheat crop in
twenty‐four hours, from the first turn of the ploughshare in autumn to the last
sheaf gathered at harvest time. A small array of boy‐lopers followed us, glad to
earn their regular stipend of one cent a mile, and the ponies raced and the boys
chased faster than ever.

This odd sort of coaching party was exciting and enchanting, and vague
anticipations of the approaching visit added zest to our enjoyment. We knew
that we were going to see a Mohammedan prince, descended from a line of rulers
whose origin was lost in the distance of unrecorded history, and that this prince
was said to be an awful tyrant among his own people, or at least as much of one
as he could be under the limitations imposed upon his power by the Dutch rule.
Such a visit of course promised something extraordinary. The native princes in
Java are still monarchs in their own palaces, and among their own people, but
they pay tribute for their splendor, and they are themselves subject to the Dutch
dominion and the laws of civilization. Blue Beards cannot now exist in Java, and
“Snicker Snees,” though worn for ornament and display, are never drawn for other
purposes.

The farther we went into the country the more thickly the people swarmed, and
we seemed perpetually passing through successions of densely‐packed villages.
We had various accidents and annoyances, and even some balky bad ponies to
trouble us, but the people were always gentle and smiling, and shoulders were
never wanting to push against any wheel of misfortune that threatened us with
danger or delay.

On the second day of our staging an officer met us, as we entered the domain
of Bandong, bearing courtly greeting from the Prince, and in the Prince’s name
joined our party. There was no change in our surroundings — the same perfect
road, blue sky and brilliant verdure — and, the same crowd everywhere, with the
difference that the gentle little people, instead of flocking out and smiling to greet
us, now cast their eyes down and sank on their knees to the ground. As the great
coach rumbled past these kneeling ranks of human beings of all ages, we thought
of a mowing‐machine stalking across a meadow, with the clover, buttercups and
daisies bowing before it on either side. This was the common way for the people
to salute the guests of the Prince, and their spontaneous expression of homage to
the Bandong uniform which the Javanese officer wore, and this explanation was
all very well for the Dutch inhabitants who were themselves loyal to their own
king. But when we saw an old man with a heavy load on his back kneel and bow
his head, a woman with a little baby in her arms throw herself on the ground,
and two boys with their lame mother, whom they were helping to hobble along,
all drop into the attitude of culprits, we expostulated and begged the officer to
sign to the people to go their way. It was of no use. They could not comprehend.
Their action was involuntary, grown out of the practice of immemorable time, and
the remainder of the drive was made painful to us by this vision of the cringing
servitude which human beings may be reduced to under the long‐continued sway
of arbitrary and despotic power.

Later in the day we drove within the high inclosure which surrounds the
palace at Bandong. The palace is a low rambling pile, very pretentious, with
pillars, verandas, and cupolas stuck everywhere, in place and out. It stands in
neither a park nor grove, but is encircled by a wide pavement of white pebbles.
This pavement forms an enchanted circle, as it were, against the approach of the
reptiles, gigantic insects, and even wild beasts, which live in the jungle hard by;
but the circle is not magic, only practical; for the creatures can be quickly seen if
they attempt to cross the white pavement, and men and boys are always on the
lookout to catch and destroy them.

The Prince of Bandong with his special retinue of personal attendants saluted
us from the door, and walked down to the carriage when it stopped under a
covered gateway. He was old and very cross‐looking, and not scrupulously dainty
in his dress.

Two small boys preceded him carrying a big closed golden umbrella, one boy
walked by his left side, and another on the right, each carrying a box of tobacco,
flint and tow. Directly behind came another boy, holding another golden umbrella,
this one open and carried over the Prince’s royal head, while still two more tiny
fellows brought up the rear.

The Prince’s costume was mixed European and Javanese. He wore a cloth coat
and trousers with a cotton skirt or sarong folded over them, and also a cotton
turban negligently tied. The little boys were dressed like him, excepting that
their trousers and jackets were made of “turkey red.” They seemed of ages from
four to ten years, and some of them were grotesquely deformed. They kept their
eyes fastened on the Prince, and their efforts to stride and walk as he did were
ludicrous, for he was a tall man, and evidently gouty.

The Prince made an obeisance to Mr.~Seward, as the villagers had done, and
when the latter alighted from the carriage the first boys opened their gold umbrella
and held it over his head; the Prince tucked Mr.~Seward’s hand under his arm,
and the little train turned and moved back through the gateway, up the steps, and
into the hall, while we watched the procession from the carriage.

Presently back came the Prince with the umbrella‐bearers and the rest of the
train, when I was helped out of the carriage. My sister and I had found the most
comfortable travelling costumes in hot climates to be simple linen or cambric
gowns which could always be made fresh, and we wore long circular cloaks, or
“dusters,” over them, of the same cool, light material. Our hats were the thick pith
“topees,” invented in India, to keep the sun off, and we tied these on with gauze
veils. On the day we drove to Bandong our dresses were of the plainest white
cambric. As soon as I stepped from the carriage boy number one opened the
smallest golden umbrella and held it over my head, and preceded by the Prince
under his enormous sun‐shade, this time I was the centre of the solemn procession
of umbrella‐boys. Prince, mace‐bearers, and officers, marching along under the
porte‐cochère, and a bandy‐legged boy on either side of me bore aloft a bit of my
muslin circular in lieu of a court train.

My sister was left alone to witness my stately progress, but the Prince, having
accompanied me to a standing‐place just outside the entrance‐door of the palace,
returned for her, and now it was my time for suppressed merriment. Our veils
luckily concealed our faces when my dear sister came slowly toward me. Her
thin white cloak, spreading out like wings, was borne on either side by the small
henchmen of this ancient Oriental king. If her manner were a little lofty and her
head slightly tossed, it was only perceptible to me, and I managed to whisper to
her, unobserved, that I wished our friends at home might see this superb ceremony.
The thought of such a \emph{bizarre} exhibition in America, did not make it easier for us
to appear unmoved.

There was no Princess, wife or daughter of our host, to welcome us to Bandong.
The Prince lived alone in the palace, in state, and his family was scattered about
in the different little houses, or huts, which though within the same inclosure,
showed meagre signs of comfort.

For all its grand appearance the palace was a rickety sort of building, and the
floors and walls tottered so much when we walked from one room to another
that we feared the whole building might tumble down over our heads.

Later came Mr.~Seward’s formal reception by the Prince, which we saw from a
distance, but took no part in. All talking was done of course through interpreters.

The Prince’s name was Wiranarta‐Kalsoema‐Radhe‐Adepathe. He was now
arrayed in a manner which, had it been in the days of his undisputed power,
might have been truthfully described as “dressed to kill.” He wore an enormous
sword, with a scabbard thickly encrusted with precious stones, and big diamonds
even to button his coat.

There were about twenty‐five officers with him, five of whom were Netherlanders, and each sat under a gilded umbrella, big or little according to his rank.
The Javanese all sat on the floor and bowed their heads to the ground whenever
the Prince deigned a glance toward them.

We now learned that the queer, wizened‐faced children, standing about the
Prince, were dwarfs. They were his personal attendants, and also his friends;
the only ones among his subjects who were permitted to stand in his presence.
One distorted little fellow was born on the same day as the Prince, and had been
tutored and trained from babyhood to be his shadow and his slave. We had read
of Paul Du Chaillu finding the race of dwarfs in Africa, and we had seen Tom
Thumb’s wedding reception, when he and Mrs.~Thumb with their bridesmaid and
best man all stood on a pianoforte to receive their guests. We had loved many
dwarfs of fiction, though sweet Ting‐a‐Ling had never been heard of then. But
here we actually saw dwarfs serving in the position they have held, in the courts
of monarchs, since the age of Hippocrates. Poor pygmies! outranked by fools
and jesters, they have always been objects of distrust and jealousy, to all about
them, for they are nearest to the despot by reason of their very helplessness. They
cannot wear honors, nor compete with the humblest full‐grown man. They have
this only recompense; that they can speak to their master as no other would dare
to do, and it is believed, even at Bandong, that the monarch sometimes hears the
unflattering truth from the dwarfs.

The seven little men, who never by night or day leave his side, surrounded
the Prince at the reception, and in a crouching attitude watched his every motion
with lynx eyes. He never spoke to, nor looked at them, but ordered them about
with a careless and disdainful clap of his hands. Once for cigars, twice for a light,
and soon — a sort of sign‐language understood instantly by the small minions and
breathlessly obeyed.

This scene was far from agreeable, and it was not improved by the entrance
of a human being, a man, or dwarf, on all‐fours, who crept up like a boy playing
“tame bear” in the nursery, and laid his head on the Prince of Bandong’s foot.

This creature was barefooted, wore calico trousers with a sort of apron fastened
over them, a cloth coat and a square of black and yellow calico wrapped about
his head.

After saluting the Prince, in the manner I have described, he crawled across
the large room to a chair near the line of barefooted natives sitting on the floor.
When the extraordinary being stood erect we saw a tall, slender Javanese youth
of perhaps eighteen. He seemed very miserable, and held his head down, and as
his face was hidden by the calico wrapping we thought the poor serf might be
suffering from toothache. To our utter amazement, we were told that this was
the son of our host, his eldest boy, and the heir to his title and estates.

Whenever the old Prince spoke to his son, the youth scrambled down to his
knees; his father addressed him with averted head and made no acknowledgment
of the son’s answers, which were made by mute signs and nods. The arrogant
old Prince seemed to fairly exult in the humiliation of his boy, and said that this
was the traditional discipline of his family! that he, the mighty Adepathe, had
served his father in domestic duties and military attendance, even as the young
Kalsoema served him now, and his manner further implied: “And see what a great
Mogul I am!”

The Prince announced to Mr.~Seward through an interpreter, that he had
directed his son to prepare an expedition for our amusement the next day, and
that “the dog” had now come for his final instructions; and then followed oriental protestations of hospitality and regard. The Prince declared himself to be
Mr.~Seward’s slave, and to own nothing in the world which Mr.~Seward might
condescend to desire, from the palace, to the meanest object in his possession.

“My son,” he said at the end, “is your son, and your slave as he is mine.
Command him, and detain him, at your will.”

This was rather more than even Mr.~Seward’s courtesy could endure of
Eastern hyperbole, and he frankly told his Highness that before he could accept
the pleasure of the expedition he must seriously ask one boon from all these
proffered favors.

“If I am to have the pleasure of your son’s companionship,” said Mr.~Seward,
“I ask that he may adopt the customs of my country and my religion, and be my
comrade and my friend, as my own sons are.”

The old Prince looked very stern, and made a ceremony of granting this
request, but finally said it should be as Mr.~Seward wished, and gave some orders
in Javanese, upon which his son dropped on his knees, and kissed his father’s
foot, and was spumed in return.

“You will see,” said the Prince, “that my son is trained in the principle common
to all religions, that ‘obedience is better than sacrifice.’” And the court seemed
highly delighted.

For ourselves, the prospect of a pleasure excursion under the leadership of
the heir of Bandong was not bright. The boy‐prince had presented such an abject
picture of degradation in our eyes, that we did not care even to see him, nor his
haughty father, again.

The expedition which the Prince had commanded his son to organize was
to a certain famous waterfall. The difficulty of getting to it cannot be imagined
without some notion of the country, and the modes of travel there.

The waterfall was eighteen miles distant from the palace. The face of the
country is one succession of serried slopes, precipices and chasms, interspersed
by unbroken jungle. The pulverized volcanic soil of these cañons and jungles, the
latter always well‐stocked with wild beasts, is kept wet by the mountain streams,
and warmed by the perpetual summer’s heat. Wherever you go, therefore, there
is the appearance underfoot, and the atmosphere overhead, of a greenhouse just
after sprinkling time. There is only one road; the great government highway
along which we had travelled. We were to reach the waterfall, half the way by
this road, and the remaining half across the country through the jungle by a path
which had been but now constructed under the direction of the prince’s son.

We arose before dawn, to start on the expedition, and when we came to the
carriage in the court we found ourselves surrounded by a troop of horsemen,
commanded by a young officer who was dashing about on a fine‐spirited horse,
and giving many orders. He was dressed in a well‐fitting, European‐looking
military uniform, and was plumed and belted in the most correct style, while his
sword and epaulettes were a shining battery in themselves. He raised his shako
to salute Mr.~Seward, and the gallant cavalry officer proved to be none other than
the young Prince of Bandong.

Our amazement was greater than when we had first seen him creeping on
hands and knees to kiss his father’s foot.

Visitors go rarely to Bandong palace — the Prince and his family never leave it.
The path which the young prince had now made for his father’s American guests
would be wiped out of existence in a few days by the quick, luxuriant growth of
untrodden vegetation, but we found it easy to travel, and also that our safety and
comfort at every step had been most generously thought of.

Our princely young host had provided twenty‐five horses for our party of
five, relays of peasants to carry the sedan‐chairs, and a posse of armed soldiers to
clear and to guard the way, and to help us over the difficult and dangerous places.
His august father had commanded him to be as Mr.~Seward had asked, his friend
as well as his host, and the young prince proved himself not only willing, but able
to obey. In the light of his submission to his father, his manly demeanor was now
very touching, for he was evidently proud and sensitive.

Walking close by Mr.~Seward’s side, the Prince’s arm and hand were ever
ready when his elderly guest needed a support. With the help of the Dutch
interpreter he talked intelligently of the birds and beasts and flowers and sports
of the country, and said that it was owing to the peasants’ great knowledge of
the principles of grades and levels, constantly used in the problems of irrigation,
that he had been enabled to make this difficult path so quickly and so well.

He was carefully grounded in Mahommedan learning, and the caliph at Constantinople was to his mind the best‐known man on earth, and the one to whom
all the world must bow. A spirit of independent knightliness however seemed to
have arisen in him, which may be attributed to certain noble and heroic principles
of the poetical Javanese people, among whom he had passed his whole life.

The Prince had heard incidents from the life of Benjamin Franklin, translated
from the Dutch into Javanese, which had excited his wonder and admiration as
much as stories of Haroun Al Raschid fascinate American boys, and indeed in
many ways there was a far greater difference between the boy‐life of the young
Prince of Bandong and that of our great Franklin, than between the lives of any
Western or Christian lad of to‐day and the splendid Abbasside caliphs in old
Arabian Nights’ history.

The sound of falling water told us that we were nearing the cascade, and
presently a deep, clear rivulet murmured at our feet, running swiftly through
banks of tangled and matted greenery. When we heard the name of the cascade,
and tried to spell it, “T‐j‐o‐e‐r‐o‐c‐k T‐j‐i‐k‐a‐p‐o‐e‐n‐d‐o‐e‐r‐g,” my sister and I
felt that our childish efforts to learn “Aldiborontephoscophomio,” by heart, in the
old‐fashioned spelling‐book, had been time wasted. We never \emph{could} pronounce
“Tjoerock Tjikapoendoerg.” “It plunges down a chasm of seventy feet,” said the
Prince with pride, “and is the most wonderful waterfall in the world; but you
can only see it from below.” We smiled an incredulous ascent from the superior
heights of our American consciousness, and looked for the nearest way down
the slippery green banks of the lower stream, which sloped at an angle of ninety
degrees, apparently! We could not see even so much as the spray from the
waterfall. The young Prince now wore a look of importance and reserve. What
rash feat of daring might he not be meditating to get us to the bottom of that
deep, impenetrable ravine?

He began examining our animals with the greatest care. They were sturdy
island horses not larger than small bronchos, with many of the same desperate
traits of character which our horses of the plains have, and as easily taught all
manner of tricks. The Prince also looked minutely at the bridles and buckles, the
straps, stirrups and shoes, made some little changes here and there, and then said
to Mr.~Seward: —

“I will take Your Excellency on horseback straight down to see the waterfall!”

The dense foliage partly concealed his figure as he now walked along the very
edge of what seemed to us a sea‐green precipice overhanging a bottomless chasm.
He apparently found some particular place, or thing, which he was looking for,
and then quietly led Mr.~Seward’s horse to this spot, on the edge, facing the chasm.
Three or four men stood near them, when, suddenly, without a sign of warning,
the Prince and the natives, Mr.~Seward and the horse, disappeared out of sight

For a second we heard a kind of swooping sound, most ominous! Had the edge
suddenly crumbled, or the earth caved in? Reassuring echoes of Mr.~Seward’s
well‐known loud ringing laughter answered from the depths below, and shouts
of “Well done! Well done!”

After some minutes the young Prince reappeared, calm and confident as
before; but his high patent leather cavalry boots were muddy, and his spurs
were somewhat awry. He came to my horse’s head, and without one word of
explanation led him, also, to that perilous edge. Four soldiers joined the Prince;
two of these men stood behind me, two by my side, and all four threw their arms
across the pony; those behind seemed ready to push him forward, those at the
shoulders strong to hold him back. They motioned to me not to be afraid, and I
put on at least a brave face.

The Prince now planted himself at the horse’s head, with his back to me; both
his hands were twisted in the bridle, which was also bound round his wrists,
while he drew the horse’s head down hard with the curb. For one instant, the
Prince and the horse, the four soldiers and I, were motionless (I am sure I neither
winked nor breathed).

One syllable of command from the Prince, and the horse sat on his haunches,
brought all four hoofs close together, and stiffened his fore legs. Simultaneously,
the Prince and the two men at my side seemed to brace themselves back to the
same angle the horse made, and I instinctively did the same. Another order, as
quick as a pistol‐shot, and we were all sliding in one compact mass down the
leafy chasm. There was no hitch, no break, and in about three seconds I found
myself at the bottom of the gorge, sitting in my saddle, safe and sound.

Mr.~Seward clapped his hands, and we both shouted this time: “Well done!
well done!”

My sister also came down, with the prince, as swiftly as I had done; and soon
we were all in the glen, admiring the waterfall which broke over tree‐ferns and
through wreaths of orchids, and which was so marvelous in the young Prince’s
eyes; but we naturally were filled with more wonder at our own unexpected
exploit of tobogganing in the tropics.

The young Prince seemed well satisfied, but was also very modest about his
own part in that exploit A splendid horseman, he had trained the ponies to this
trick of sliding, and had himself prepared the coast, seeing that every root was
dug out, and every pebble cast away, which might catch the firm hoofs as they
ploughed down through the slippery loam. We found their furrows, and also
those made by the men, nearly identical, and straight as if cut by a die, black lines,
in the slippery green moss.

In the midst of our tramp, for we followed the Stream for several miles, carried
in sedan‐chairs now, a pretty kiosk came into view. Here the young Prince had
prepared a feast, and he acted the part of host at this Javanese picnic as graciously
and much more gracefully than his royal father could have done.

In the evening we went to the house of one of the old Prince’s wives, to listen
to some music, and here we saw two little girls, his daughters. The Mohammedan
women hang their heads, hide their faces and tremble in the presence of their
husbands, fathers and sons, and the princesses of the house of Bandong were no
exception to the rule. It would have been an easy matter for the young Prince to
act the tyrant among his little sisters, but he was evidently their hero, trusted and
well beloved.

The two princesses were timid little things, of three and four, dressed precisely
like grown women, and they behaved with the same obsequiousness which their
mother had shown in her shrinking though ceremonious reception. Later, when
one of the gentle little girls grew tired, the young Prince lifted her in his arms,
and carried her away as tenderly as any kind big brother might have done. We
had grown to admire the young Prince’s truly noble nature as much as we did his
handsome manly face. We had seen him so subservient in his father’s presence
that to our first impression he caricatured the very dignity of obedience into
worse than a farce. But, in that attitude of servility, he was only conscientiously
acting out the ideal imposed upon him by the tradition of his race, his own
father’s example and command. And Prince Kalsoema had done better than his
father Adepathe. This young Mohammedan Prince was both valiant and gentle in
the spirit of true chivalry, where humility strengthens courage, and heroism is
chastened by tenderness.

We have never heard of him again, but it cannot be imagined for a moment
that our young Javanese prince, this modern Saladin, would ever make his son a
slave.

\chapter{A Typhoon.}

When my sister and I were children, our home was on Lake Erie, near Chautauqua,
and we grew up sharing the common opinion of the people of that region — that
we knew rather more about wind‐storms than those of less‐favored experience.

More is known about storms now, than when we were children; for then there
was no weather‐bureau at Washington, no Signal‐Service anywhere.

A cold wave, or a tempest, sweeps across the continent to‐day, and pays flying
visits to many out‐of‐the‐way places, taking no one by surprise; for the moment
it leaves Manitoba, or even a more distant point, its progress is spoken and made
known everywhere by the telegraphic storm signals which our government has
organized and maintains for the benefit of farmers and seamen.

But when we were children we never heard of tornado‐traps, and storm‐signals
were unknown. The wind on Lake Erie seemed to blow capriciously, and just for
fun, and we never knew at what moment it might come. We understood that as a
rule it began blowing among the great guns of the fort at Detroit, and skipping
down the lake, stopped for frolics at Sandusky, Cleveland, Erie, and Dunkirk, and
finally ended in a double‐banked cotillion grand‐chain at Buffalo.

The maple and apple‐trees in our country, by the lake, grew with their branches
turned southeast — that is, away from the breeze; and a Chautauqua boy whom
we knew, named George — who rivaled the west wind in whistling as it soared
round the church steeple — used to sing in high soprano, while we all battled with
a north‐wester, on the way home from school: —
\begin{quote}
“‘A life on the ocean wave’ — \\
The man who wrote it was green; \\
He never had sailed on the lake \\
And a gale he never had seen.”
\end{quote}
And we never doubted George’s knowledge or authority.

We learned later, from the seamen who chanced to come to our country, and
to sail on our lake, that they really dreaded the winds there, and made haste to
put into the nearest port whenever a cloud or a flaw betokened the prospect of a
squall.

Now Lake Erie is a long shallow sheet of water, narrow, and full of dangerous
channels. It lies in the track of prevailing winds which sweep it easily from end
to end. The “old salts” said this was why the gales were so disastrous, and stoutly
averred that there was nothing to be dreaded from mere wind if one had a good
ship and plenty of sea room.

When crossing the Pacific, my sister and I heard tales of the fierce hurricanes
or typhoons which sweep that great sea, they had no terrors for us, for we
remembered what the sailors said. We rather hoped to meet a typhoon in the
middle of the boundless ocean, and to compare it with a Lake Erie gale. Our wish
was grafted unexpectedly, as wishes often are, but not exactly in the way we had
chosen, which also may happen.

We landed at Yokohama, in Japan, and for many days wiseacres foretold that
something unusual was about to happen there; signs appeared, meaningless to
us, but unmistakable to them. The air was clear, and the barometer higher than
usual, cattle were dull and restless, and storks flew languidly.

We were going to Yeddo by sea, a little voyage, of not more than fifty miles,
along the coast, in a United States man‐of‐war.

Our ship, the {\it Monocacy}, an iron‐clad double‐ender, carrying four guns, was
built in Baltimore and sent out to the special duty of waiting in Asiatic waters for
any stirring events which might happen there, and she proved worthy the trust
before we left her.

The {\it Monocacy} had been listlessly riding at anchor for many months, and when
she now weighed anchor, and got under way, every one on board was glad of the
change to a cheerful little outing.

A war vessel always gives a perfect picture of order and neatness and discipline,
and our ship now outshone the highest standard. Freshly painted, and burnished
at every point, her decks newly “holy‐stoned,” and her sails just bleached, she
was as fair, as crisp and fragrant as a pure pond lily, floating under the July sun.

We were a party of six, and the captain’s guests. His cabin was quite large
and, used for a dining‐room and salon, was cosey, almost luxurious.

The younger officers presided over the wardroom, their special cabin, “aft,” and
they had now converted it into a sort of boudoir, and had brought out photographs
and keepsakes, and adorned it furthermore with nosegays and flowering plants.

Nothing could be calmer than the sea that day, and we steamed slowly along
near shore, the graceful coast‐line fringed with palm‐trees and Fusiama’s fair
cone, resting among the clouds, in full view. Our flag scarcely stirred in the gentle
breeze as we sat on deck under the ship’s ample awnings. The young officers
were untiring and eager hosts. There was banjo playing, and some good ringing
choruses, and as the day cooled, even a little dancing on deck; but later came
the best sport, when the officers, each in turn, told splendid stories — regular
“yarns,” about battles, and storms, and pirates, and cannibals, and all manner of
hair‐breadth escapes at sea, which gave us a pleasant sense of past or distant
danger, and admiration for the gallant young narrators who seemed to have
endured many of the hardships they now described. No matter how dull those
stories threatened to become, they always ended well, and sailors seemed to have
been spared from the most perilous disasters to tell the tale cheerfully.

It was sundown when we dropped anchor in the shallow bay of Yeddo, about
ten miles from shore, and we expected to row across this stretch in the ship’s
open boats.

A heavy rain set in, and we decided to stay on board until morning, to escape
a soaking.

The {\it Monocacy}, moored all summer long off shore, had been taken possession
of by a large colony of gnats, who selected the stateroom cabins for permanent
quarters; when we went to our berths, tired out at last by the day’s amusements,
we found that we were to be imprisoned in folds of the strongest netting, nailed
above, and tucked in on all sides to defy the attacks of this Japanese mosquito — a
truly formidable foe.

When we entered these filmy entrenchments, the rain had ceased and the sea
was calm, clear stars, and a soft moon were shining through our open ports.

I was awakened suddenly in the middle of the night by the flash of what
seemed a shining flood of molten silver pouring through the port, but which
proved to be a wave of cold salt water, bright with the phosphorescence which
often makes that sea so luminous.

This glittering, foamy stream soon spread over my little bed, and threatened
to swamp me before I could tear down the mosquito barricade. My shoes were
floating in the briny wave when I seized a big wrap hanging near, and, drawing
it around me, stepped into the deserted ward‐room to call for help. No answer
came to my summons, though piercing calls from the boatswain’s whistle, and a
running order all along the ship to “close the ports,” showed that the sailors were
on the alert.

Driven again from the drenched stateroom, I seated myself in the cabin. A
strange silence pervaded, the air was heavy, the sea evidently rising.

The ship’s barometer was near the companionway, not far from where I sat.
Leaning against the table, my head in my hands, I drowsily saw two officers come
down to examine the glass, and heard their unguarded exclamations of wonder
and alarm: —

“It is dropping like lead.” “It was 30° at three o’clock — now it is almost down
to 29°.” “It is a typhoon.”

Wide awake now, I called my sister. Her stateroom was on the side untouched
by the first wave sweeping against the ship, precursor of the wind which was fast
making its way toward us and driving the sea before it.

We dressed quickly, thrilled with the prospect of a typhoon at last, and shortly
were on deck, where officers and crew, all haste and bustle, were making ready
for the storm.

My sister and I had our sea‐chairs lashed to the mainmast where we could
watch the sailors and the coming storm in safety.

It was four o’clock — day was dawning in a grayish green light, a few dim
stars still hung in the sky, like beryl stones.

After a ringing call to “all hands down light yards and masts,” no special orders
were given. Each officer and man knew his part in preparing a ship at anchor
for a storm, and all, even the merriest ensign, now wore the stem expression of
conscious responsibility. They were “on duty” to work to‐day, not to play as they
had been yesterday.

The battery was now “secured for sea,” the boats and guns made fast by gripes
and extra lashings. The next move was to look to the ground tackle. Already one
anchor, a bower, was out, but the second, and two sheet‐anchors and coils and
coils of heavy chain, were still in their places on board, ready to follow. Steam
was getting up to be in readiness to help the ground tackle, should that prove
inadequate to hold the ship.

At intervals of fifteen minutes, or more, sweeping gusts of wind, from one
quarter and from another, whirled past us, each followed by a lull. The greenish
dawn passed into a copper‐hued day; as the sun rose, a dull metallic round, the
sea rose too, while the sky lowered, lowered.

We were surrounded by shipping: Chinese junks, queer Japanese boats, European vessels, and craft of every sort, anchored as we were, and now preparing for
the storm.

A fine Portuguese merchantman lay between us and the sharp rocky ledge
of a little island hard by, where a tall lighthouse stood like a silent sentinel. The
merchantman was so near indeed that her people on board had listened to our
music in the evening, and now we could scan her decks and count her crew.

She seemed to have taken in passengers and cargo, intending to weigh anchor
with the morning tide, and be off on her voyage to the Mediterranean.

Her preparations for the coming storm were made with more animation
and ado than ours were. Long after every rope and screw were in place on the
{\it Monocacy}, agile seamen were scaling the masts and rigging of the Portuguese ship,
clearing the decks, and shifting the cargo. We heard the sailors’ cadenced voices,
calling “heave to!” and “pull, my lads, together!” and by the help of our glasses
could see that the brave fellows, gay perhaps with the hope of the homeward
voyage, made light of the coming storm.

It was now six o’clock. The sky had darkened to the color of umber, and the
air was loaded with brine. A suffocating brown mist by degrees shut out the
shipping, the island, and the lighthouse from our view. We no longer saw the
ship’s length, and her breadth was lost in the mist. Finally we could not see each
other, though we sat close to the mainmast still.

The blow was now fairly upon us. The {\it Monocacy} had swung grandly around
to the wind, and the remaining bower was “let go;” one sheet‐anchor followed,
while all the chain was “veered” out to them; the second “sheet” and our last, was
held in reserve to use if one of the chains should part.

Both engines, under full steam, were working for all they were worth, while
four men at the wheel held the {\it Monocacy} to her moorings.

Even now there was no steady wind, but only great blasts, whirling faster and
faster, which lashed the waves with fierce fury until the sea seemed a seething
caldron of foam, held down and pressed smooth by the wind, and bursting forth
in the intermitting lulls to wash over oar decks, sweeping the hatchways and
guns.

Everyone obeyed the captain’s final order for all hands to go below, and the
hatches were battened down.

The cabin was dark. Every movable thing had been put out of the way, the
hanging lamps and mirrors taken down, and nothing left save bare tables and
benches screwed to the floor.

My sister and I sitting near to each other, in the dark, could not hear our own
voices above the din of the storm now raging furiously.

Ah! we had never seen anything like this Asiatic hurricane in the lake winds
of America. There the wind, no matter how boisterous, is straightforward; it blows
one way. You know where to find it But the typhoon, a sudden and eccentric
storm, is in form a spiral curve, which describes a huge circle in its course. The
center of this circle is said to be quiet and calm, but all the seafaring people I have
met, who have witnessed the phenomenon, have been, as we were, in the vortex
of the great gyrating disk, and have never seen the calm center of a typhoon.

Boom! came the wind striking the iron‐clad gunboat, first on one side, then
on the other.

Boom! boom! beating the seas from beneath the great ship and lifting her
into the air, only to bang her down again, grating on the sandy bottom of the bay
with a cruel, crashing sound.

At nine o’clock the storm was at its height. Everything breakable on board
was smashed to atoms, the glass and crockery ground to powder.

At each wailing blast we knew the {\it Monocacy} stood the chance of parting
chains, and we of being dashed to pieces with her — that the storm might drive
against us any one of those strange ships which we had seen riding at anchor
in the Bay, or that we might ourselves be drifting toward the island where the
lighthouse stood.

We were all speechless, and calm enough, too overawed in the presence of
such awful power to realize any distinct thought or emotion. The close air, the
howling din, combined to stupefy us, and all, even the captain, fell under a torpor,
as of a narcotic.

We were clinging to the tables and benches in the ward‐room, and had fallen
in the lethargy near the place where I first heard the report of the coming storm.
Here again the young officers came to examine, and report the movements of
the glass. It rose, faster than it had fallen. The air lightened, and reached us
refreshingly through the ventilating shafts.

At mid‐day we were on deck again. The wind was going down, the sky lifting,
and sunlight was fast making its way to earth through the dark, brassy mist.
When this, in turn, cleared the Bay of Yeddo stretched out before us; but, alas,
no longer proud and gay with the masts of many ships. No craft of any sort was
visible. All had been scattered, driven out to sea, or wrecked in the storm, and
the waters were strewn with their timbers and débris.

Our good ship was unhurt, no rope had broken, nor iron given way, and we
had no parted chain. But her paint was gone, and bruised and beaten by the storm
she bore the shrunken look of age, which a tempest of grief or misery may bring
suddenly to the fairest face of youth.

Notwithstanding the force of the {\it Monocacy}’s well‐tried strength, we had
drifted a mile and more. Farther away from us than when the storm enveloped it,
the tall, white lighthouse now kept sentinel as before, but across the rocky ledge
of the island where the beacon stood, stretched the wreck of the Portuguese ship,
dismembered and desolate. She had struck the rock in one of those fearful blasts,
and lay broken half in two.

The dark, fast‐flying clouds soon turned their gold and silver linings outward,
and the day beamed calm and beautiful, as days will often beam after the fiercest
storms; but as this storm had exceeded in fury all other storms, so the day
surpassed in beauty all the days that we had known. Sky, air and ocean so lately
shrouded in gloomy mists and tempest, were now united in a glow of prismatic
splendor; the dancing sunbeams flashed in countless rainbow hues, while the
billows threw back their radiance from the shining sea below.

\chapter{A Monsoon.}

The typhoon is only an occasional visitor, to particular localities at special seasons,
in Asiatic waters, and never appears on the equator.

The monsoon is a different kind of wind altogether. It is a feature of the system
of regularly alternating air‐currents which blow over India and the Indian Ocean,
and to which the sailors long ago gave the name of “trade‐winds.”

In the winter season dense, dry air flows from the interior of Asia, southwestward, over the Indian Ocean, and pours the cool breath of temperate regions into
the vacuum at the equator made there by the powerful action of the sun’s rays,
which draws the moist heated air away from the surface of the earth into the
upper regions of clouds. These benign currents are the northeast monsoons.

But in summer, when the air‐currents have passed over the long stretches of
heated sands on the plains of Asia, they become warmer than the air over the
Indian Ocean, and the latter flows northward, to carry its freight of moisture over
the parched plains, a southwest monsoon.

At the seasons when these air‐currents are shifting from one monsoon to the
other, seafaring men are on the alert, for then severe local storms occur, and the
blows become dangerous.

We began our voyage to Java just as the northwest monsoon was setting in. It
blew pretty hard, but our ship, a Dutch steam jacket, built of teakwood, gave us a
sense of security which was fully maintained by the good order and discipline on
board. She was the {\it Stoom Schepen Konigen der Nederlanden}, of the Dutch Steam
Navigation Company, one of the oldest commercial lines in the world.

The return voyage from Java, the delicious island, was destined to be a less
easy one. We were again on board a Dutch steamer, the {\it Singapore}, bound for the
port of Singapore, and carrying a cargo of cattle.

Our first look at the {\it Singapore} gave us no good impression. She was cramped
and narrow, and was two days late in sailing; our confidence was utterly shaken
in her, soon after leaving port, by a somewhat novel experience.

A few hours out, and while we were steaming rather inland than toward the
open sea, a rebellious cow among the cargo tossed her head in the air, jumped
overboard, and struck out boldly for the shore.

The {\it Singapore} gave chase, and the chase lasted for several hours. It was often
hard to tell which had the best of it, the ship or the cow; but at last the poor cow
was overtaken, lassoed, and hauled on board. The plucky swimmer had, however,
so nearly outstripped the {\it Singapore}, that thereafter we felt no pride and but little
faith in the power or the speed of our ship.

Not long after this excitement we ran into the monsoon, now fairly set, and
blowing hard and steady from the northwest. Drenching rain fell in spouts like
condensing steam. The {\it Singapore} labored in a dull heavy sea. When the sun
set, lightning began playing mad freaks in all the wonderment of its dazzling
equatorial beauty among the fast‐flying and thirsty‐looking clouds.

The stuffy staterooms were close and intolerable, and we improvised cabins
on deck, out of benches with sails hung round, where we tried to rest.

Unable to bear a suffocating odor of mingled fog and tar any longer, I folded
back my flapping sail‐screen, at midnight, and sat upright. The absence of all
sound or movement near me proved that my sister and friends slept undisturbed
behind their shelter, which was more breezy than mine.

The night was black. Inky waves swept now and then across the bows;
thunders from the deep seemed rising to meet the sharp resounding clashes in
the sky, while, at intervals, lightning flashes made near objects as distinct as day.
A stationary table, draped now, and covered over with tarpaulin sheets, stood
midships on the deck of the {\it Singapore}. In the day we had seen this table divided by
racks into partitions, spread with a few sea biscuits, and many decanters filled with
spirits, “schnaps” and liqueurs, which were often tested by the Dutch passengers,
but never by the ship’s officers, while the crew conspicuously avoided so much as
a whiff of the odious fumes. This peculiar custom on board the Dutch steamers
plying in Indian waters, is never, so far as I have noticed, copied elsewhere at sea.

The captain, a serious, gruff‐voiced old seaman, was still on deck, giving the
night’s commands in Dutch, all hands speaking that uncouth tongue which is so
strangely unintelligible to all unfamiliar as I was, with its sound. He passed near
our cabin encampment, where I sat alone on the dismal bench, while making his
last round of the ship. His keen far‐seeing eyes beamed kindly, and he seemed
inclined to speak, but there was no interpreter at hand and he passed by. An
unusually heavy lurch of the ship, a vivid streak of lightning and a peal of thunder
which seemed resounding to the poles, caused him to turn back, and coming
nearer to me he said, with great earnestness, and deliberation: —

“{\it Er is {\sc geen gevaar}, mejufvrouw, ga gerust slapen en heb {\sc geen vrees}.}”

I knew no word of what he said, nor how to answer, and only bowed my head,
while his prophetic gravity kindled all my smouldering alarms.

The air of importance which belongs to any ship’s commander worthy the
responsibility of such absolute authority as he must exercise at sea, in enforcing
that discipline and obedience upon which many lives may depend, was emphasized
in our Dutch captain. His impressive words now conveyed no meaning to my
ears, but they carried a warning to my sharpened senses which determined me
not to slumber nor rest until I knew what danger he feared enough to cause him
to put me on my guard so sternly.

The second officer came on the watch when the captain went off, and there
was much clatter among the crew, and walking to and fro on deck.

A remittent sound of hammering, coming from the interior of the ship, had
made itself heard, for some time, above the din. At two o’clock this sound had
become an incessant tap, tap, tap, hammer, hammer, hammer, on a hollow, metallic
substance. What could it mean?

Suddenly the hammering ceased, and some one came up the gangway. A flash
of lightning revealed a tall, gaunt man, bareheaded, his sleeves rolled up to the
shoulders, turning face to the wind. His brawny arms were lifted above a shock of
red hair, which crowned his redder face. His red flannel shirt was in keeping with
his general fiery appearance, but the dark line on his gray trousers‐legs, showed
that he had been standing ankle deep in water.

After breathing in some deep draughts of fresh air, the ruddy apparition
cautiously proceeded to uncover the decanters on the stationary table, then
hurriedly poured out and swallowed a dram of schnaps and darted below; shortly
afterwards the hammering recommenced.

For one hour, and more, I watched this florid tar come and breathe, and drink,
and go below, and for one hour and more I listened to the sound of tapping,
hammering, pounding, as it rose, and fell, alternately, with his coming and his
going.

Each time that he came on deck, the wet gauge‐line on his trousers‐legs stood
higher. He held excited debates with the officer on watch, from which I learned
that he was an Englishman and engineer of the ship. These talks were followed
by surreptitious visits to the schnaps decanter, by the engineer, when the officer’s
bade was turned.

The lightning showed these details in snatches; but between the glare all was
dark and still, save the roaring sea, the wind, and the roll of distant or nearer
thunder.

When at last the engineer appeared, looking as if he had stood in water to the
knees, I made bold to ask him the meaning of this sign, and of the hammering.
He answered reluctantly, and in a thick, peculiar speech, that the condenser had
given way, and that he was trying to mend it, but that water was coming into the
engine‐room faster than he could bail it out. Later, he said the machinery was
going worse and worse, the ship rolling like a log, and for his part he saw no way
but to take to the life‐boats. We were not far from land, he continued, and were
more than likely now to be drifting toward a lee shore.

Presently two of the ship’s officers began examining the life‐boats, and sailors
brought sea bread and jars of fresh water which they stowed away in them
hurriedly.

After this, the second officer came to me with the engineer, who acted as
interpreter, to say that it would be well to get such things together as we might
need if forced to put off in the ship’s life‐boats, for some near island; that they
were making everything ready, beforehand, to save confusion when the time
came for the captain to give the final order.

The officer made little of saying this, but to me there was something truly
dreadful in the prospect. It was like choosing between a hawk and a buzzard
to say which was safer at dead of night — the steamer, now disabled, which at
her best, in open day, had been nearly beaten in speed by the swimming cow, or
a little lifeboat in such a sea. The nearest coast possible was that of the island
of Borneo, which for fifty miles around is an impassable swamp, inhabited by
felines, reptiles and orang‐outangs. This island jungle was the scene of some of
the boldest and most fantastic yarns the young ensigns on the {\it Monocacy} had told
us before the typhoon blow, and it was no comfort to remember now that they
said the cannibals of Borneo dwell in the interior of the island, reached only by
rivers and streams, for they also told us of the reckless character of the plain
pirates who lurk on the immediate coast.

What preparations should one make for such a voyage to such a country? My
friends still slept, my sister close beside me, and I determined not to rouse her
sooner than need be, and went alone to the hold of the ship, with one faithful
servant, to get from our trunk such articles as might be most useful in an open
boat at sea, or if we become stranded on a desert island shore.

We had one trunk full of articles which friends had made and given to us when
we started on our travels, to provide for our comfort in any emergency. I naturally
thought of this trunk in this extremity. In it were, indeed, bags, and boxes, and
cushions, and cases, and photograph‐frames and racks, brush holders, and hooks,
housewifes and sachets, and other beautiful, useful things — but nothing suited to
our present needs. I learned now, to my surprise, that these pretty things belong
only to conditions of plenty and luxury, and are only useful when people already
have every thing that they require for life and comfort, and I also learned how
many trifles, apparently most desirable, one can do without. I recalled Robinson
Crusoe’s plight after the grounding of his ship, and the list of things he found in
the hold of the wreck, and which went to make his cave so cosey and his life so
cheerful in a distant island solitude, and I closed my trunk with a sigh. Robinson
Crusoe would not have found what he needed in this hold.

It was in fact a very perplexing question to decide, for I could carry only the
barest essentials, the things one would want first, and part with last. Money? My
sister and I had saved up a few bright eagles, intending to exchange them in India
for some specially rare or beautiful thing; they mocked me now, and I passed
them by, recalling Crusoe’s apostrophe to gold under similar circumstances: —

\begin{quote}
“O, drug! what art thou good for? … I have no manner of use for
thee; even remain where thou art, and go to the bottom, as a creature
whose life is not worth saving!”
\end{quote}

But like Robinson Crusoe, who upon second thought changed his mind on the
subject of gold, I reconsidered the eagles, put them in a little chamois‐skin bag, and
tied them around my neck. Next I selected, one by one, some photographs, a few
books, some pens, ink, and paper, a roll of rugs, a small, but conveniently‐packed
dressing‐case of necessaries, and gladly turning from the stifling hold, made my
way back to the tempestuous scene of the deck.

When the officer by the life‐boat saw what I considered my very modest
collection of comforts, he said it would not do. There would be no room for rugs;
photographs, books, pens, ink and paper were out of the question; bottles of
{\it eau de cologne}, and even Pond’s Extract, could not be thought of; soap, brushes, and
sponges were not necessities I He allowed me to keep a cup, tied around my waist
by a silver chain, and also two little utensils chosen from the dressing‐case abundance, which I could carry in my pocket, and which even the Dutch subordinate
admitted were useful at all times.

An umbrella was now placed in the life‐boat, and it was pronounced provisioned and ready to be launched at a moment’s notice.

Meantime the heavy gale blew onward from the northwest, and the hammering
rang louder from the engine‐room. The boatswain’s whistle seemed never tired
of shrieking challenges to the wind; pale lightning flashes showed the faces of
officers and, crew blanched and anxious; and then again, amid thunder‐rolls and
the ship’s deep lurches, the trackless sea merged black around us.

My sister still slept, unconscious of the ordeal through which I was passing.
O, home seemed very far away, and my eyes, weary with the vigils of the wild
night, strained to find a reality among the phantom coasts which loomed beyond
the darkness.

Suddenly some streaks of crimson light shot upward to the clouded sky, from
the horizon, east, toward America, like a greeting to my anxious heart from that
dear land.

And truly a message of hope and good cheer did come with the crimson light,
for the sun in his daily round had passed our continent, and was now bringing to
the other side of the earth the good‐morning which our friends were sure to have
wished for us when his last shining rays sank below their western horizon, to
beam again in daylight for distant Asia.

The captain was on deck now, and sharp, brief commands, carried from bow
to stem and back again, were executed in haste and good order.

The hammering ceased, and word came from the hold that the condenser was
“all right.”

My sister, awake now, learned with amazement, in the early morning hours,
how distracting the night had been to me. I showed her the life‐boat sparingly
provisioned with hardtack and fresh water, pointed out the solitary umbrella, and
described my visit to the hold, and how, by the light of the ship’s lantern, I had
painfully chosen such things as I thought we would have cared most to take to a
desert island.

“What did you choose?” asked my sister eagerly.

I told her of all I had chosen, and showed the two little things which the officer
had allowed me to keep.

While not disputing the good sense of my selection, my sister thought she
would have chosen a little differently had she been in my place. She rather regretted, than was pleased, that I had not called her to share my memorable watch;
for now, when it was passed, she found it hard to realize how near those experiences had seemed which fancy always clothes with romantic interest — the perils
of the sea — and could hardly bring herself to believe that for hours, shipwreck
and starvation, cannibals or pirates, had in imagination successively stared me in
the face.

At breakfast the captain, all smiles and compliments now, explained what had
seemed a shocking want of discipline the night before.

The ship’s regular engineer, a Hollander, had fallen suddenly ill as we were
about leaving Batavia, and this had detained the ship. The only substitute to be
found on such short notice was the harum‐scarum, raw Englishman, who was
out of a place and glad to engage for this one passage. His first act was in flagrant
disobedience of the rule of this and every other good service — that no man on
duty shall touch spirits; and he not only indulged too freely himself in the potent
schnaps, but induced the quartermaster, also a green hand, to follow his example.
The strong stimulant quickly muddled their brains, and the engineer took to
pounding on the boiler, which had doubtless bulged a little in the race with the
cow, and when he discovered that the condenser was sprung, his excitement and
incoherence caused the panic among the mates and men, which I witnessed, and
which later the captain’s few hearty rebukes put an end to.

The regular engineer, better now, straightened things out in the engine‐room,
where he found that the pumps had not been used at all.

There had been, it seems, a series of blunders and misunderstandings all
around, which the captain made plain and also related what he knew about
monsoons, having sailed in them going and coming, he said, for nearly forty
years. These great winds rise and fall with the tides, and as the day waxed now,
the monsoon lulled.

The captain with the help of an interpreter told me that foreseeing a rough
night, likely to make lands folk fearful, he had taken care, just before going off the
watch, to reassure me, and he hoped I had enjoyed a good sleep. The real meaning
of his solemn‐sounding admonition was now made known, and the Dutch words:
“{\it Er is geen gevaar, mejufvrouw; ga gerust slapen, en heb geen vrees,}” translated into
English mean, “There is no danger, young lady; go to sleep and have no fear.”

It was not strange, perhaps, after my first grievous misapprehension of the
captain’s kind advice, that the spirit of confusion and discord, abroad in the
elements that wild night, seemed to me to have entered the ship’s staid company.
In the scene I witnessed, by lightning‐light, from my tented perch in the stern of
the {\it Singapore}, the pale, immovable faces of the dark‐bloused Dutch sailors were
images of dismay, while the figure of the apparent leader and chief of the tumult
and mischief, the erratic engineer, with waving arms, and hair aflame, illuminated
like the rest, now here, now there, looked not unlike a Salamander in a vision of
Furies.

I have found it easy, since that dreadful night, to understand young sailors,
who verily come to believe that they have necessarily passed through the horrors
of storms, dangers of shipwreck, or even the desolation of castaways on cannibal
and piratic shores, which their older shipmates relate as the likely experiences
and common adventures in “a life on the ocean wave.”

\chapter{Animals I Have Met.}

Storms and dangers are to travelers like bumps and bruises to base‐ball players,
pleasant experiences, when once passed. The typhoon and monsoon were only
grateful memories when we set sail again, westward.

Sailing on an orient sea, where no cloud obscures the sunshine, and no breeze
ruffles the ocean’s perfect calm, the monotony of the beauty becomes wearisome, and one grows drowsy watching the bright angel‐fish darting through the
foam‐drops in the vessel’s wake. On leaving Ceylon and while still breathing the
fragrance of that lovely island, with such a voyage of six or seven days, in the
bay of Bengal before me, it was a boon to hear that there were some lion‐whelps
on board, bound also for Calcutta.

I made haste to hunt them out, and found them on the upper deck; two
homesick, bewildered creatures, about three months old. They had come all the
way by sea from Africa, and were now huddled together at the back of the cage,
as far out of sight as possible. The sailors were poking sticks at them, and playing
at “lion‐hunter,” knowing their victims safe behind the iron bars. The lions were
resisting bravely. The more the men poked the more stubborn the whelps grew,
answering only by occasional deeply indignant and portentous growls, but never
showing their faces. Their stolid persistence gained the day at last, and the sailors
retired. One of them came back to put some pieces of raw meat between the
bars of the cage, and I hid behind a great coil of ropes to watch the poor little
prisoners.

For many minutes they were as still as lions carved in stone, and then, all
being quiet, first one Stretched out a stealthy paw, and grabbed a piece, then the
other, and both retired to the back of the cage with their booty, accompanying
their carving operations with croupy growls. They gradually came to the front
again, and I saw them plainly. They were about the size of ordinary pointers, very
scraggy, and so lean that I could count their ribs. Their coats were rough and
humpy, the hair half frizzled and brownish. There was no sign of the mane, nor
any other beauty about them, and they had at the ears that appealing, immature
look of all young animals. Their paws were big and heavy, and they had great
blinking green eyes, the iris showing only a fine black line like a cat’s at midday.
Their expression was at once wild, solemn, and babyish, their whole appearance
grotesque beyond description. The sun was blazing full in their faces, and I could
see that it blinded them, and fancied that they hated eating in the glare. They
scurried back to the dismal corner at the slightest sound, though not without
some air of dignity, even when tumbling over each other, and falling in a furry
heap of legs and paws and ears.

I felt very sorry for the young wanderers, so strong and yet so helpless, and
longed to be kind to them, but hardly knew how to begin. I named them to myself
“Jack and Jill,” for convenience only, for, as they came from the Cape, and as lions
live on the plain, it was not likely that they had ever gone up a hill, and surely
there was small chance, even if they had done so, that they would ever go up or
down hill again. I begged the sailors, at least, not to tease the whelpies any more
for that day.

I had read in one of Gordon Cumming’s books, I think, how lions prowl at
night for their food, liking to eat just before dawn, and to hide and sleep all day. I
saw to it therefore, that the little lions’ next meal was given them as the day was
breaking. Indeed, I was there to see it properly served and to give them a first
lesson in table manners, of which they stood somewhat in need. They were wide
awake and on the alert when the early food came, and I asked the keeper to put it
down quickly and go away.

Then coming near to the cage, I said, “Jack and Jill,” distinctly, several times,
pushing the food under the bars toward them. Their eyes were large and brilliant
in the dim morning light, the irises now distended until they seemed quite black.
The instant I spoke. Jack drew back, but Jill lunged forward, stuck out her fore‐paw
and sat up straight. She tossed back her head with the motion of menace and
defiance, as much as to say, “You hurt my brother if you dare!” I remained perfectly
quiet and she very likely thought her insolent bearing had subdued me, or that I
was an inefficient sort of sailor that no one need to mind; for in a few minutes they
both came forward and ate their brakfast, all things considered, harmoniously.
I had treated them to a big bowl of bread and milk, in addition to their portion
of meat, and I cannot deny that they were very growly and somewhat greedy in
their enjoyment of it.

That day I induced the sailors to move the cage into a quiet corner, improvising
a sort of den for them, among some great bales, out of the sun, and in the way
of a refreshing breeze. I begged that they might be fed again after sunset, and
presided at their supper as I had done at their breakfast. So again, at every meal
they ate, until, before we reached the end of the voyage, the baby lions seemed to
know me, and also to know their own names. They showed this by lifting their
heads and listening intently to the sound of my voice, and by a low inquiring
sort of growl. Then their eyes would dilate and bum and a little creeping quiver
would show in the tips of their tails, like a cat’s when watching for a bird or a
mouse. At the very last, they ate quietly in my presence, and ceased to start when
I approached, a mark of recognition and confidence which they never gave to the
sailors.

They showed much intelligence, but not the least disposition to playfulness,
their best mood being one of imperturbable solemnity. Jill was more excitable and
quarrelsome than Jack, to whom, however, she showed many marks of sisterly
devotion. His coat was in rather better condition than hers, and I discovered
that she stroked and licked him, much as a mother‐cat does her kitten. It was
apparently understood between them that Jack should have the first and biggest
piece of meat, though Jill seemed quite as strong and the more active of the two
animals.

They were on their way to a menagerie I was told, and one fancied, that as
they deplored their captivity, they would still more resent the indignity of being
hauled about in a gaudy wagon, if they did not pine under the calamitous fall
from roaming hereditary king and queen of the forest, to the low estate of mere
show animals as well as prisoners in a cage.

I gave a last meal, and said good‐by to my queer little fellow travellers with
some emotion; never expecting to meet or to hear of them again, and knowing
full well that many strange and even painful experiences might lie before them
in the great world. We parted when the ship landed at Calcutta, and their grim,
wistful young faces haunted me for hours.

All sad thoughts about the little lions were however diverted by the story
of another wild creature, a Bengal tiger whose romantic fate became known to
me. These beautiful and fierce creatures, native of Hindustan, infest the dense
jungle which covers whole districts of the broad plain or basin, watered by the
many mouths and outlets of the Hoogly River. The Hindus live in deadly terror of
them, but, as everybody knows, it is against the religious principles and humane
sentiments of that wise and gentle race to kill or even to ill‐treat any creature.
Wild animals are especially regarded with commiseration as well as fear, and are
treated by all with something of the tenderness and forbearance which Christians
accord to the insane. To the Hindu, human beings of unsound mind are even
regarded as peculiarly sacred: their affliction being counted the worst that can
be visited upon man, and all who are so afflicted are loved and protected as
beings specially chosen by the gods to bear the consequences of the sins and
sorrows of others on earth, and to be rewarded in the future with Heaven’s best
gifts. By the same compassion, these fierce wild creatures are allowed to roam
unmolested in their native jungle. The great floods which inundate the swamp‐like
expanse, their favorite prowling ground, destroy the small animals which are
the tiger’s legitimate prey, and when thus deprived of their natural food, they
became ravenous and terrible. No living being is safe from their ferocious attack.
Whether by an instinct of loyalty to the race which has protected them, or by
natural affinity of tastes with Fee‐fo‐fum, I do not know, but I am told they choose
“the blood of an Englishman,” whenever they can find it, in preference to that of a
Hindu. A price is put by the British Government in India on the head of every
tiger, kittens not excepted, as the only protection possible against the horrible
consequences which the extreme humanity of the Hindus, and the savage power
of the ravenous beasts produce.

To the sort of men who like to chase and kill animals, tiger‐hunting in India is
counted the rarest, as it is certainly the most heroic of all sports, and is conducted
scientifically, with much of the drill and circumstance of war. Natives are trained
to track the creatures through the jungle, and in the capacity of “beaters,” with
the aid of horses, to drive and hunt them into the open where they are met and
shot by clever marksmen from the backs of elephants.

At the Government House in Calcutta, I heard the story of a creature whose
life was spared in one of these famous battles with the tigers.

A party of young officers once organized a grand hunt, impelled by the
unusual terror and suffering of unprotected natives, after a great flood. Armed
and equipped, and following a large company of beaters, they struck into the
jungle on well‐trained elephants, and soon stampeded a notorious band of tigers.
Tigers old and young, who had the stripes and wrinkles, the claws and teeth, and
all the marks of race implying pedigree, which would entitle them to prizes in
any competition for tiger honors in the world.

This particular group moreover, was known to be uncommonly ferocious and
insatiable. The cruel use they had made of their splendid strength and freedom of
the jungle, to prowl into and devour whole villages, proved fatal to their chances of
escape now. All their power, cunning and agility, in escaping, decoying and even
attacking their foes, failed to prevail against the fierce pursuit of the tiger‐hunters,
and their stern purpose to exterminate the savage band. The infuriated creatures,
one after another, were captured and shot, but not without many a serious wound
to the pursuers, their horses and elephants. Beautiful tiger‐skins were secured as
trophies of the chase; claws preserved to be mounted in gold and so converted
into what some people fancy to be suitable ornaments for fair and gentle women.

One huge tigress was driven, during the fight, to her lair, or in the agony of
her wounds sought that retreat, and was killed in the presence of her young ones,
one of whom also perished. But one was asleep in a corner, and the hearts of the
officers were touched by the sight of so much unconscious innocence, or latent
mischief, cuddled away in a tiny yellow tiger‐skin. The little creatures were only
a few hours old, and this one was as helpless as a new‐born kitten, and very little
more scratchy.

One of the officers rescued the motherless tiger baby, and carried it away
in the howdah on his elephant It mewed and cried a great deal for its mother
before it could be induced to accept a feeding bottle substitute. But it was such a
plucky, hearty little thing, that once it got a good taste of milk, it went of! to sleep
contentedly. The officer took the tiger princeling to the Government House for a
present to the Governor‐General’s wife. The Countess of Mayo already possessed
many beautiful tropical birds, and a tame monkey or two, and made no difficulty
of accepting this strange pet, but was rather pleased to add such a rare jewel to
the list of her favorites, and shared the ownership with her son the Honorable
Terence Bourke who was about five years old. It was from this young gentleman
that I learned the tiger’s story.

It soon grew into a charming creature. Although but a tiny tiger kitten it was
as large as an ordinary house cat, and under the perfect care it received, it grew
sleek and agile. Its smooth, glossy coat was a bright yellow color, dark on the
back and growing lighter underneath; its breast as white as the whitest rabbit
skin. There were stripes like a tabby’s going around the tiger’s body, a dark stripe,
almost black, down the back, to the tip of the tail, and many shining rings like
small, painted bangles around the active little paws. He had a black nose, a very
red mouth, and a pink tongue so rough that it felt like a lemon grater scraping
over the hand when he gave a friendly lick. His eyes, big, round things, were
sometimes green, sometimes yellow, sometimes like fire, and always as bright as
diamonds, except when he was sleepy, and then, soft and hazy like all sleepy cat’s
eyes.

The little tiger was named “Ned” and proved to be an intelligent, docile pet,
with all the mysteriously enchanting ways of a tame kitten, intensified. He slept
on the veranda, in a basket, at night, and followed Lady Mayo and her little son
about the rooms, the galleries or the gardens in the day time. A perfect playfellow,
he delighted in any game of romps, running, leaping, hiding or seeking. He played
a ball, catching, rolling or hiding it; he chased flies, ate bread and milk, comporting
himself like a kitten in all ways, but was far more intelligent, affectionate and
fascinating, the honorable Terence assured me, than any tame puss could ever be.
He delighted in being stroked, and resorted to all sorts of pretty kittenish devices
to get his great head under his mistress’s fair hand. Then he would close his
eyes, sheathe and unsheathe his claws, and purr like a contented pussy‐cat; only
like a giant kitten, so loud that a stranger might fancy it the sound of a distant
policeman’s rattle. Left by himself he seemed frightened and miserable, crying
out in desolate mews and tiger calls, so articulate, that rescue was sure to reach
him soon, or else he diverted his loneliness by such desperate acts of mischief,
that his silence was more dreaded than his outcries.

But when playtime was over, this exacting creature seemed even more contented than ever, as he gathered himself up for a nap, looking a great topaz on the
train of his mistress’s velvet gown or some other luxuriously chosen resting place
near his friends. It was pretty to see this wild creature of the forest wilderness
subdued to the demurest ways of civilized life, walking with his young master,
under the palms and banana‐trees, and among the brilliant flower‐beds or the
Eden gardens which adjoin the palace, followed by tall, lithe Hindu attendants.
These men, in the glitter of their vice‐regal livery, all white and gold and scarlet,
which is purposely designed to harmonize the British colors with oriental ideas
of show and splendor, escorted the viceroy’s son, with watchful but deferential
tread. The boy all unconscious of the state which surrounded him, enjoyed the
gay companionship of his favorite playfellow; while Ned in a gold collar and
chain followed his sturdy little master in a quiet promenade, or better still, all
glee and brightness, capered and leaped, in many a mad race and frolic, at his
side. The dogs and cats in the neighborhood, for reasons of their own, got out of
the way of this gay côrtège, and provided a wide berth for its passage through
the park.

And Ned had silken cushions and delicate food as well as gentle companionship. He was a good, obedient little tiger, and very happy and contented too.
He knew very well what a little switch meant when he had chewed up boots
and bonnets, or torn a sofa to pieces, and never intimated that he thought its
application unjust, when such sad occurrences made it necessary.

He grew apace in strength and beauty. At four months he was as large as
an ordinary Saint Bernard dog and very powerful. His claws grew longer, his
white teeth stronger, and every one feared him save Lady Mayo and her son.
They remained fond and proud of him. and he returned their affection with
characteristic feline devotion. He still ate bread and milk and kept his guileless
pussy face, never having shown the slightest disposition to hurt any one. People
who knew the tiger nature well however, tried now to persuade his owner to have
his teeth filed down, and his claws cut off. They refused to hear of it, declaring
that so long as Ned was amenable to reasonable discipline and kindness, they
would not have him made to suffer. Notwithstanding this, Ned while enjoying
comparatively the freedom of the palace was strictly watched and guarded, and
alas! this proved a very wise and necessary precaution.

One day by accident, at play with Terence Bourke, he scratched the boy’s
hand quite severely. Oh! that was the turning point in Ned’s life, little as he
dreamed it in his reckless romp. Never before bad he smelt or known the taste of
blood, and to his savage senses it was irresistibly intoxicating. They who saw him
knew that he would never forget it. Instantly his whole expression changed from
a frank, fearless, contented creature’s to that of a cunning, stealthy and cruel one.
His mouth watered and he licked his chops. His tail quivered and his eyes flashed.
He crouched, clutching the carpet and prepared to spring at the friend whose
confiding affection made him the easy victim of treachery. But strong hands and
arms were near to seize the tiger and defy his ferocity.

Poor Ned! His time of freedom was over. All the confidence which his graceful,
fascinating ways had inspired was gone. He would never be again the comrade
of friends who trusted him, but evermore a creature feared by all, and subject to
the mercy of those who, knowing his real nature, would protect themselves by
imprisoning him, while they gratified his horrible thirst and pampered his cruel
appetite, to make him the better show. He seemed never so attractive as the day
after the accident, when he was again the fond and gentle cat, arching his back
and purring a drowsy tune, leaning against his sorrowful little master, whose arm
was in sling, and whom Ned coaxed vainly to join in the accustomed romp. Ned
was so used to admiration and indulgence, that having only acted on a natural
impulse of his tiger heart he had no notion that any thing was amiss. Not many
hours later he was secured and borne away.

All this happened the year before I was in India. The tiger was now caged in
the gardens of Government House at Barrackpore.

Terence Bourke could never endure the pain of looking at the pet who had
been his loved comrade, after seeing his face distorted and demonized by that
look of cruelty and fear. He also found little consolation from the fact, that Ned
now had plenty of the one thing he liked best, raw meat, to tear and growl and
grovel over. He never cared to go to Barrackpore.

One day I went on a visit with the Governor‐General and Lady Mayo, and a
party of ladies and gentlemen, to this lovely summer palace, by a great river’s
side. We passed over the same road which Ned in chains had travelled on the
first day of his banishment. It was a strange feeling to me, an American, to be
of a company, who driving through a friendly district, were escorted by lancers,
the viceroy’s guard. They were mounted on swift prancing barbs, with leopard
skins for saddle‐cloths, the riders carrying gay banners or armed with long
murderous‐looking weapons, as if they were expecting to meet the dragon whom
we are taught to believe Saint George effectually disposed of, long ago.

The gardens at Barrackpore consist of groves and avenues of tropical trees, and
acres of beautifully cultivated shrubs and flowers, from all countries and zones.
We had tea under a banyan‐tree, or rather under the shade of one extreme portion
of a tree of that wondrous growth, which stretches its branches horizontally from
the main trunk, when they turn straight down, and striking into the earth, take
root and grow again like a second tree. This one tree, with its many branches,
made a whole grove or bosquet.

At the other side of the great banyan, there was an enormous cage, which
was Ned’s house. Mr.~Ned, now grown to the size of an Alderney cow, but not
the least like an Alderney cow, either in appearance or character, was within, and
was a splendid kingly creature to look upon. He was walking aimlessly up and
down the cage, but with the sweeping tread of a conqueror, and the grace and
lightness of a Persian kitten. His coat, sleek and shiny, as if dressed by a valet
with brilliantine, was yellow, yellower even than madder or gold. His stripes and
bangles, each distinctly defined, were now black as polished jet. His eyes, green
and terrible, glared moodily into space.

Lady Mayo had directed that his cage should be made as pleasant as a cage
could be. At one end there was a tank of fresh running water, to cool his parched
throat, while the other end was shaded by the banyan branches, affording him
retreat from the sun.

Lady Mayo, who had walked with me to the cage, now went quietly to the
farthest side and called him by name. The great tiger halted and lifted his head,
wrinkling his nose and showing his cruel teeth, and this, for only an instant Then
there evidently rose in his brain an effort of attention to some far away memory
or association.

When he heard the clear gentle call again “Ned! Ned!” he walked stealthily
forward, listening with bewildered intentness, and a pitiful, partial gleam of
recognition, in his strange, wild eyes. When he saw his former friend standing
close to his prison bars, he made one rush and plunged forward, his enormous
form heaving and quivering as he crouched to the ground. I have seen the same
action in a squirrel or raccoon, when pleased or terrified. Lady Mayo would not
have been afraid to remain, recognizing only some of the well‐known pranks of
her {\it ci‐devant} pet, but she was told by the keepers that this would be rash, as the
tiger might, while meaning only to be affectionate, strike his great paw, with its
terrible claws, through the bars and wound her to death.

Walking away from Ned’s house, I came upon a large fresh looking cage, from
whose depths there issued a prolonged and familiar sound. On looking in, whom
should I see but Jack and Jill, grown sleek and plump, but as grim, grum, and
growly as ever. They were switching their tails and rolling their eyes and growing
into first‐class specimens of African Cape lions. Jack was expecting shortly to
wear a flowing, black mane, and Jill to have a fashionable tuft to her tail.

They had improved immensely and were evidently becoming accustomed
to their surroundings. I was told they evinced intense disapproval of the noisy
dissipations of the packs of jackals which scampered through the park by night,
and were highly indignant at the impertinence of the crows, who, flying in
enormous flocks, darken the sky by day in that region, and sometimes swooping
down, cast envious glances at the lion’s fresh meat. To the little jackals and
the ravens, they were only unenviable captives not even to be feared, while
to themselves and each other they were lions still. They would have nothing
whatever to say to me, and I had to pass them by, as animals with whom I had
only a very distant or casual acquaintance.

\chapter{The Elephants of an Indian Prince.}

The elephants in this story are not heroes or heroines of romance, nor are they
related to the Jumbo family whose domestic trials, travels and tragic fate have
excited so much interest in England and America, They are merely simple‐minded,
contented creatures, doing, or learning to do, their straightforward every‐day
duty in the state of life to which they have been called in their native land, for
these docile, sagacious and powerful creatures appear everywhere in Asia, as
a necessary part of the panorama which the half‐barbaric customs of human
activity present there.

At Peking you see the palace where the Sacred Elephant lives in splendor and
retirement until the coronation of an Emperor calls him to the performance of hb
only task — that of bearing the new monarch to his throne.

At Borneo they tell you how happy the people were when the telegraph
was laid through the wastes which separate one habitable part of the island from
another, until the wild elephants in the jungle spying the telegraph poles, knocked
them all down, by rubbing their great sides, and scratching their backs against
them, and destoyed the telegraph line.

Travel in the mountains of Cambodia is all done on elephant‐back; and in the
seaport towns of Ceylon elephants perform the duties of laborers and porters,
piling wood and coal, loading and unloading ships, and lifting heavy weights;
while in the country they do every sort of work, especially that of nursery maids,
carrying children and babies about safely, and gently rocking them to sleep with
their great trunks. In Hindostan no prince’s nor potentate’s stable is complete
without a troop of them.

One day Mr.~Seward, my sister and I were walking in the Eden gardens, which
adjoin the Government House, in Calcutta, where the young son of the viceroy
had frolicked with Ned the tiger, and there we met a prince. He was not like the
European princes of royal blood, who look like other gentlemen, and differ from
there only when they are more amiable or urbane. This was a real story‐book
prince, an Oriental prince, a first‐class Maharajah, frowning and promenading
in the park with a retinue of stately followers attending him. He was a tall man,
very handsome, with flashing black eyes and curling hair. He was clad in India
muslin of finest texture and whitest sheen, golden shoes, a velvet coat, a satin
shirt, a sword, a turban, and jewels “galore” — that is, more than you can count
and as many more and more again. A man among men, in fact, as a peacock is
among birds.

This superb person had come to Calcutta from his principality at the North
where he lived as other Indian princes do, a reigning despot, but at the same time
a heavy tribute‐payer to the British Crown. His family had always been loyal to
their British conquerors, and now the Maharajah had come to the British East
Indian capital to be decorated with the Star of India, the order of knighthood
which was created for Queen Victoria’s Indian officers and subjects, and for which
Prince Albert chose the beautiful motto: “Heaven’s Light our Guide.”

His Highness looked disdainfully about him, and said that he was not like the
men and officers at Calcutta; that he was an absolute ruler in his own domain, and
could kill men there if he pleased without the formalities of court and justice, and
he seemed very proud of this prerogative as a right becoming his unquestioned
importance.

He had never studied geography nor read a book, such accomplishments being
counted vulgar for a prince whose thinking was all done for him, and whose
understanding even could be supplied by others. He was fully persuaded that
the subjects of the British provinces would do well to know one another, and he
invited us most cordially to visit him, when we should pass through his country,
on our travels in the north; saying that he lived in a poor, mean way, but that
such fare as he could offer we were most welcome to.

We protested that we were not subjects of a province, and ventured to explain
the matter; he declined to be enlightened, however, saying that he already knew
all about America, and we could tell him nothing more, but that as he intended
making a tour of the provinces one fine future day he would go there and see for
himself, and meantime he wished us to see his country, and to know his people,
and exchange ideas with them. The Prince was very kind, and opportunity serving
us it came about that we went to his country, and not only met his people, but
also met the elephants of whom I am going to tell you, or rather they came out to
meet us, as you shall see.

Oriental hospitality is based on a totally different principle from that cheerful
practice which we of the West hold to be one of the highest duties and sweetest
privileges of our social life. The self‐forgetting modesty in receiving guests which
is the first requirement of good‐breeding with us, where the visitor’s happiness
and comfort are the host’s most sacred trust, is unknown to the people of the East
The Oriental host wishes to exalt himself in the eyes of his guest, and the latter’s
ease and sense of freedom are quite secondary to this desire; he is even willing to
mortify his visitor, if possible, by a show of his own superior grandeur and riches,
while he always aims to appall his own people by an extravagant display of his
power.

Our Prince was an Oriental of the Orientals, and we could not have seen the
Eastern idea of hospitality better illustrated than it was in his manner of receiving
us.

A cavalcade of horsemen, with a lot of velvet‐lined and bespangled carriages,
met us at the boundary of the province, and were joined by a company of lancers
bearing banners, who followed us for miles, with an array which might have been
mistaken for a Barnum’s circus procession, minus the bison, the calliope and the
grizzly bear.

The capital of the province was guarded by a strong fortress mounted with
cannon, and a broad avenue led from the fortress to the ruler’s palace, some miles
away.

When we come to the fortress new features were added to our procession.
Regiments of every department of military saluted and presented arms, while five
or six hundred led horses, half as many camels, and fifty or sixty elephants were
drawn up in grand array, the latter standing quietly in line, blinking their eyes,
fanning their ears, and swinging their tails in contented waiting.

The elephants were fitted out with gorgeous howdahs on their backs, housings
or flounces of silk and cloth‐of‐gold on their sides, velvet pantalettes with lace
frills on their legs, bangles on their great bungling ankles, and long shining jewels
hanging from their ears.

Presently an officer of state came to the carriage where we sat, bringing
a courteous and somewhat pompous message from the Prince to Mr.~Seward,
bidding him and his family welcome thrice and a hundred times, and asking him
to choose for his own to take away and keep forever any or all of the troops and
beasts before him.

Then the Prince appeared, dressed in dazzling white, his breast and turban
shining with emeralds worth more than the ransom of an Irish king, and after
various salutes and salaams asked with a solemnity, incomprehensible to us, considering the occasion, whether Mr.~Seward would go to the palace on horseback,
elephant back, by camel back, or carriage. Now Mr.~Seward was an old traveler,
and estimated so justly the values of experiments in locomotion that he answered
without hesitation, “By carriage, by all means,” believing that he spoke for his
inexperienced and timid companions as well as for himself.

He was taken without delay to a gilded chariot, drawn by six white horses
which were guided by postilions, and passing into the broad avenue he was soon
lost to our sight in a great cloud of dust with the Maharajah.

My sister and I, waiting and watching the novel scene, were now asked by
what conveyance we preferred going to the palace, and we also had no hesitation
in choosing a carriage, and as it was growing dark and we were very tired we
secretly hoped that we might be sent on our way as promptly as Mr.~Seward had
been.

A young British officer was traveling in our company who knew the language
and all the manners and customs of the stately Hindoos. After some talk with the
ministers of state who were conducting this ceremony of receiving the guests of
the Prince, the officer told us that it was contrary to etiquette for us to choose the
least magnificent of all the ways provided for our little journey, and that by all
their rules we must accept the compliment so munificently offered in the elephant
conveyance, as it would give serious offense if we did not. Convinced that under
the circumstances it was our duty not only to accept, but also to appear delighted
in imitation of the spirit in which the courtesy was offered, we adopted for the
time being the Oriental interpretation of one’s duty towards one’s neighbor.

The plan had been for the Prince to go on the first elephant, Mr.~Seward on
the second with suitable attendants, my sister and I to follow on the next two,
respectively, the captain and the First Hindoo Secretary of State on the next, our
servants, who were supposed to be officers of high degree, following; and the
natives again in their own fashion.

Now as the Prince and Mr.~Seward were gone in the chariot, this programme
had to be altered. The inexorable master of ceremonies said state etiquette required
that I, being the eldest, should be conveyed to the Prince’s palace on the back of
the first elephant; my sister on the second, the captain the third, the “American
Secretary” on the fourth, and so on.

All this punctilious ceremony seemed unreasonable and tiresome to us, and
as any American girl can fancy, I faced my doubly‐imposed duty, to which the
still small voice within gave but a faint sanction, with a heart wavering between
dread and impatience.

The elephants were brought nearer and preparations begun for our transfer
to their backs, from the carriage.

The first elephant was an enormous creature, of the Asiatic species, which has
small ears and a smooth grayish‐brown hide. His ivory tusks were engraved with
vermilion, and ornamented with brass knobs and rings; his trunk was about eight
feet long, and where his fine draperies did not cover his legs and body he seemed
fat and well‐cared for, though neither brush nor curry‐comb had been used in his
grooming, for there was not a hair to be seen on him save the tuft between his
ears, like the thick growth between the horns of an ox.

He looked about calmly and with much intelligence, and eyed us steadily, at
the same time with an expression of good humor and docility, touching in so
enormous and powerful a creature.

The howdah on his back was most imposing. It was about the size of an
old‐fashioned sleigh or as they call it in Boston “a booby,” with a canopy over
it instead of a hood. The elaborately‐carved frame and arms were covered with
plates of real gold, the floor, seat and sides were upholstered in crimson velvet,
the canopy of cloth‐of‐gold and crimson velvet rose to a sort of pinnacle in the
centre surmounted by a large gilt ball, and an enormous peacock fan stood in
each corner ready for use.

The howdah seemed as high again as the elephant’s body, and midway between
his head and his haunches he was harnessed with shelves, or platforms, about a
foot wide and six feet long, hung horizontally.

A driver dressed in white and crimson sat on the elephant’s head, just behind
his ears, armed with a heavy, sharp steel spike with which to guide him.

The order was given which the great sagacious creature understood to mean,
“Prepare to receive your burden.” He bent one great leg, and the howdah seemed
lifted higher in the air than before, but one‐sided; then he bent the other leg
under him, and the howdah toppled over the other way; in this manner it changed
position four times as the elephant folded down each huge leg and knelt, human
fashion, on the ground. But even now the howdah was higher than the back of
any horse, and a ladder was at hand to be placed against the side of the gigantic
steed; a silver ladder, smooth and strong, which was a part of the elephant’s
accoutrements, and was carried on one of the side shelves.

The master of ceremonies, speaking English with the incomparably beautiful
voice and accent of an educated Hindoo, now assured me that it was quite safe and
not difficult to mount the ladder; he said he would hold my hand, while attendants
standing on the shelf and on the elephant’s back would guide my unpracticed
feet to the distant howdah.

Now here was a chance for an exchange of experiences and ideas between the
Hindoo minister and the Western visitors which might have been a benefit. He at
least would have felt no concern for those strangers about to ascend the elephant’s
back, if he could have known the American method of gathering cherries, or have
been told of the clambering grape‐vines in Chautauqua County, and how the
nimble‐footed boys and girls there scale ladders to reach the topmost branches
where the most luscious purple clusters are sure to hang; but there was no chance
for parley of our home, and silently recalling these youthful accomplishments I
made no difficulty of reaching the howdah by the way of the ladder.

Once there the structure tottered and tipped under my feet, like a badly‐managed boat. The velvet‐covered, luxurious‐looking seat was a hard, narrow board,
high above the floor, and sitting exactly in the centre to balance the howdah, the
polished arms at the sides of the vehicle were so far away that I could not reach
them either. This isolated and footholdless position gave me a feeling of insecurity
and distance which may be easily fancied as not pleasant or reassuring.

Presently the elephant, directed in Hindoostanee to arise, began slowly drawing one leg into position, his motion producing the effect of an earthquake upheaval in the howdah. When the creature had assured himself by a hard stamp
(which made the howdah shiver as if struck with an electric shock) that the
ground was firm beneath that foot, he began again very deliberately to draw
another leg into place, leaning his huge body the other way and toppling the
howdah at right angles again. At this moment, and while struggling to keep my
equilibrium, not to say my dignity, unmoved, I was unexpectedly rescued from
the self‐sacrificing ordeal. My sister was not only younger than I, but she was the
youngest in our home. All older children know the peculiar prestige of “the baby”
in the nursery and school‐room, and what power of persuasion the “youngest
child” shows from the very first, by various but unmistakable signs. When our
little ruler wanted anything very much, the brown rings of her hair seemed to
curl tighter, her eyes brightened, her cheeks flushed, while certain little signal
marks appeared at the corners of her mouth, which meant “up,” or “down,” as the
decision of the presiding court seemed likely to go for or against the wishes of
this very special pleader, and a long and successful practice had fixed all these
signs and signal marks so securely that it seemed impossible they should ever
change or go down again.

But to‐day, when this little sister, now grown a tall young lady, stepped
forward, I saw all the old familiar signs in her hair, her eyes, and about her mouth;
and when she said, very distinctly, to those Hindoo panjandrums, “My sister must
not go on that elephant alone, and — I cannot go alone on an elephant. We are
strangers, and it does not seem proper or safe to us. I trust you will be so very
kind as to provide some other way for us,” I recognized a determination behind the
soft voice which made it easy for me to understand why the Grand Chamberlains
and the Marshals looked confounded. They had probably never heard a woman
express a preference or assert the slightest authority before in their lives. Our
British friend told us that they were alarmed lest this unexampled act on the part
of the young American lady meant the beginning of some exercise of that strange
power called “liberty,” which they had heard was allowed in the United States, but
which they naturally dreaded to have seen or spoken of in their country. Their
good standing at court, if not their happy existence anywhere, depended on their
making in this gaudy procession a display which would prove their master to be
a monarch worthy the name.

This young foreign girl, in their judgment, might interfere so much, with her
incomprehensible ideas and her fearlessness in expressing them, as to ruin all
their plans. They took counsel together in confusion and difficulty, and the confab
ended in the proposition for a compromise, by which we consented to go together
on one elephant’s bade to the palace.

The Minister of State was very courteous, even obsequious after this, and
my sister quickly tripped up the ladder, and we sat side by side in the crimson
howdah.

Two gaily dressed boys stood on the shelves on either side of the elephant,
and now each one seized a long peacock’s feather fan which he waved over us,
the driver stuck the steel spike into the elephant’s ear, and he began slowly to
arise again.

As the howdah swung to and fro, my sister and I held our hands tightly
together, and for a few moments we could fancy ourselves back in the Manocacy
while she labored in the typhoon; then came smoother sailing as preceded by a
detachment of the military the whole procession started, and we moved on.

It was now quite dark and we had five miles to travel before reaching the
palace where the Maharajah receives and entertains his guests.

When the procession started, a grand salute was fired from the artillery guns,
and the infantry likewise blazed away with their muskets, the cavalrymen drew
their clattering sabers, and several bands of music began playing together — but
each a different tune. “Rule, Britannia,” and “The Campbells are Coming,” “Saint
Patrick’s Day” and “The Girl I left behind Me” did not accord very well, but they
were played loudly, and in good faith, that among this wide selection the national
air might be hit off, which would be most pleasing to these subjects of Great
Britain who came from the wilds of the distant province of America, to see the
civilized world. Rockets and Bengal lights were struck and exploded, and all so
suddenly, that we looked anxiously to see how the animals bore the noise and
confusion of this astounding crisis.

The horses pranced and were frightened, the camels became restless and
rebelled, but the elephants listened quietly and then complained; showing no
alarm, but setting up a deep‐toned remonstrance which resounded above all the
din, and moaning out a sorrowful protest against the unreasoning hubbub and
storm.

When this grotesque pageant was fairly under way, we looked back from our
high position to see the strange effect it made — recalling that it was through
this very region, perhaps over this very road, that Lalla Rookh had passed in her
romantic journey from Delhi to Lahore on her way to Cashmere.

The Prince’s Prime Minister of State followed with his companion, on the
elephant directly behind us, leading the way for the third, which carried Freeman,
our good Washington manservant, who had gained his rights of American citizenship under the fourteenth and fifteenth amendments to the Constitution of
the United States. His marked color, intelligent, thoughtful face, and quiet dignity
impressed the people, and he was greeted by the throngs who crowded the streets
with all the respect and interest due to the “American Statesman,” whom they
understood him to be.

Eugenia, the maid, a tall, broad woman and a native of the Swiss republic,
followed Freeman, on the fourth elephant. She had lived long in India, and
traveled in the trains of the Governors’ families so many times that she was not at
all overcome by the grandeur of this Oriental procession. She had also some strict
ideas of her own which no argument prevailed against, and no consideration
could induce her to forego. Her wearing apparel was a subject of almost religious
importance, and she had adopted a sort of regulation uniform which was never
by any chance altered or changed. Her young ladies might think it suitable to
dress in cotton and flax in that tropical land, but her dignity never permitted
her to wear any but a heavy, tightly‐fitting black silk gown; and while we wore
“topees” and goggles for protection from the heat and glare, tied our heads up
in thick veils, and wore large, cool gauntlets, she never was known to appear in
any “head‐gear,” but a small fashionable bonnet, a lace “mask” veil, and always
encased her hands in two‐button kid gloves several sizes too small.

By the rearrangement of the procession my sister and I were so hidden,
shielded and protected in the great howdah intended for the tall Prince that the
natives passed us by unnoticed; and while they saluted Freeman respectfully, they
looked at Eugenia with wonder, as well they might, for she sat high in her howdah
alone and self‐possessed fanning herself and gazing from side to side with all the
immovable {\it aplomb} of the Goddess of Liberty herself, and whom, to those natives’
minds, she probably represented. The English courier followed Eugenia, and then
came the long, winding line of led elephants, camels and horses; the remaining
artillery, cavalry, infantry, and a company of imported bagpipers bringing up the
rear.

Elephants, when walking, get over the ground quickly by their long strides,
but their slow monotonous gait has to the rider all the discomfort of a heavy
chopping sea.

The strange and depressing sights around us, as we passed, could only be seen
in a country utterly unlike ours in every particular.

Here the Prince owned all the land, the houses, the cattle, and crops, and passed
his life in the pursuit of his own pleasure in his great, sumptuous palace; while his
subjects lived in little one‐story huts, owned nothing, and had no higher object in
life than ministering to their master’s caprices and the occasional excitement of
witnessing a parade wending along the broad avenue, leading from the fortress
to the palace, when the Prince received his guests and displayed his power.

Looking down from the howdah to the roofs of the huts and the throngs of
humble natives, we discovered that this was the only highway worthy the name
of street in the capital, and there were eighty thousand people living there. The
Maharajah has the power of life and death over his subjects, but their welfare
and happiness apparently give him little interest or concern. We saw no church,
schoolhouse, town hall, or other sign of civilized social life and intelligent human
intercourse in all our long march. There were no lights in the streets, and the
smoking flambeaux of the escorting torch‐bearers showed the crowded, dreary
highway no less dismal than the flat, uncultivated fields through which we passed
before alighting at the palace door.

The elephants behaved in a way to win our highest praise and admiration.
Instantly obedient to a sign from the driver they walked faster or stood still,
turned to the right or left, and trod carefully over the rough places, with more of
the double consciousness of a dog than the single‐eyed intelligence of a horse.

The next day these same elephants came to take part in a sort of zoölogical
entertainment which the Prince treated us to, where he showed all the animals of
his great menagerie.

Now we saw without his drapery, the elephant which had borne our howdah.
He was very large, though not so tall as Jumbo, and had been captured when
he was a wild little calf, and given to the Maharajah’s great‐grandfather, then a
boy, and the elephant had been for nearly one hundred years the pride and pet
of the stable and menagerie; no wonder that he was as much at home with the
Maharajah and his keepers, as our most docile domestic animals are with us. In
being groomed he was first lathered with soap, and then scraped and brushed by
strong‐armed men and sprayed off with a fire‐hose, enjoying his bath with all his
might, for at the end he was allowed a plunge in a deep river or pond, where he
swam about for hours under water, with only the tip end of his trunk coming up
to the surface for air, and this bit of a trunk, skimming along, looked not bigger
than a small frog on his travels, though there must have been a pretty big swirl
underneath the wave.

He came and went at will without a keeper, and after being shown to us and
taking some tea cakes very gently from our hands, he trotted off alone when he
was bidden across the fields and under the palm‐trees, to his stable three miles
distant.

An elephant fight now promised much excitement, if the natives were to be
believed.

The great creatures stripped of all trappings and made ready to fight by having
their tusks cut off short, were brought, two by two, into a wide open field. When
let go, they ran at each other, head foremost, with their trunks in the air. The fight
was very stupid, being simply a huge game of “push heads which is the toughest,”
where the stronger won and then drove the weaker off the field. They seemed
good‐natured, and also to enjoy the game.

Some ten or twenty other elephants looked on, apparently interested and
amused, until one very strong, active elephant among the fighters ran after his
vanquished antagonist with the evident intention of striking at his trunk, when
all the other elephants became excited, and constituted themselves a company of
umpires and set up such indignant moaning that the keeper interfered.

The Hindoos, who know nature’s deepest secrets so well, and who care for all
dumb creatures so tenderly, told us many things about the wild elephants and
their ways. They always live in herds, choosing a leader whom they all follow and
obey. It seems that living so, in families as it were, in the jungles and plains, they
have a code of honor among themselves, and a standard of ethics most creditable
to their intelligence. An elephant who fails to live up to this standard, or who
disregards the code, is punished by the leader and disciplined by the rest of the
herd.

It completely disables an elephant to have his trunk wounded or bruised, and
in fighting with tigers and lions he holds it high out of harm’s way and attacks
with his tusks. It is judged unfair among the herds for one elephant to strike at
another one’s trunk, and one who commits this base act is put into Coventry, and
forced to live by himself. So even in the wilderness cowards are shunned by their
fellow creatures, and treated like pariahs. The animals thus disgraced, become
morose and moody in their lonely vagabond life, and are almost untamable.

Elephant hunters, who have given them the name of “rogues,” avoid them
when they can, for they are always fierce, desperate creatures. We saw one
beautiful elephant who was born in the Prince’s province and had never seen a
jungle, but had grown up in the pasture and stable, like any other cow. When
she came to see us her own little calf paced by her side. The calf was the little
counterpart of the cow, and was a very pretty creature, whom one would like for
a pet. She wore draperies and frills and gold lace like her mother, with ear‐rings
which nearly reached to the ground, and gold bangles. She walked jauntily along,
“toeing out,” stiffening her knees, and holding her chin down in the most approved
manner. When we offered her a bit of sponge cake, she sidled nearer, like a pet
lamb, lifted the little finger at the end of her trunk, and examined the cake daintily
before taking it; and apparently never having seen any like it before, she turned
toward her mother with a questioning look. The mother elephant seemed puzzled.
She walked toward us with an expression of hesitating, anxious curiosity in her
small, knowing eyes, as one may see any cow do. She held out her trunk for the
cake, and the little one dutifully gave it to her, whereupon the mother turned
it over carefully, then held it up and looked at us as if for an explanation. We
motioned toward the little one, to whom she promptly returned it, and then
looked on contentedly while the calf enjoyed the tidbit

This indulgence on our part seemed to gain the mother elephant’s confidence,
for she began showing off her offspring with unmistakable pride. She pushed the
little one toward us, and turned it round and round with her great trunk. When
the calf demurred she coaxed and cat reused her. The cow was evidently vain of
the calf’s finery, and encouraged that spoiled elephantling to flaunt her furbelows
and tinkle her ear‐rings. She drew our attention to the big, little fat legs of the
beauty, and finally wound her trunk affectionately round the small neck, lifted
the little head, and showed us the beginning of her baby’s first tusk.

We thought this gentleness and intelligence were due to training, as neither
of these creatures had ever known the wild life of the jungle; but we had a chance
later to learn that this was not altogether the case.

A wild elephant cow and calf lately captured were driven up to show us the
difference. They were both darker and very rough, lean and hungry‐looking in
comparison with the domesticated ones. A strong chain tied the wild mother‐elephant’s fore‐legs together, and she was also fastened with a strong lariat to a tame
elephant. The wild baby‐elephant kept close to its mother and stumbled along
like a shy, awkward hobbledehoy. When the keepers tried to turn the wild calf
toward us, and away from its mother, the little one threw back its head, stuck
up its chin, and cried out loud and piteously. The poor mother struggled toward
her terrified calf and managed to get her own ponderous body between the calf
and the strange‐looking people. The little one refused to be comforted, and the
mother’s ways of protecting and soothing it were so tender and knowing as to
seem almost human. She stroked it with her big trunk and shoved it lovingly
behind her, and finally persuaded the little one to take some nourishment when
it threw back its small trunk dexterously, and drew the milk, smacking like any
satisfied, hungry calf.

They became quieter when they saw that no harm was intended them, and
then the little one was more amusing than ever, running under the mother and
hiding behind her great legs, occasionally darting a shy frightened peep from
behind the shelter. If we looked, or went toward her, she dodged back and hid
her face, and if we took no notice she came nearer, and even stepped one foot
forward in a testing, gingerly fashion. Meantime the bold town‐bred elephant
youngster looked on with great interest, waggling her tail, jingling her ear‐rings,
and tossing her trunk in high glee, apparently much amused at her countrified
sister’s awkwardness and discomfiture.

We had other journeys by elephant‐back before we left the Maharajah’s country; and while we learned to accommodate ourselves to the hard jolting motion of
their heavy gait, and found more comfortable and simple accommodations than
those of the high, royal howdah, we never learned to like elephant‐riding. We had
little opportunity for conversation with the Prince, and we discovered among his
people few ideas that we cared to exchange our own for. We learned a great deal
more about elephants than we could have done anywhere else, and we were fully
prepared to tell the Prince, if he had asked us for an opinion, that for a comfortable
and convenient mode of carrying passengers, the primitive “buckboard” line of
North America is infinitely superior to the gorgeous elephant‐train of Hindoostan.

\chapter{Adventures of an Abyssinian Monkey.}

This pretty mischievous little creature became a member of our party in Egypt,
and proved a most diverting traveling companion.

We had but lately come from India, the land of strange beliefs and fantastic
customs, where the peculiar veneration which the Hindoos bestow upon monkeys
had given us a new interest in the monkey race.

And the attention of all thinking people had recently been directed to the
nature and habits of the monkey everywhere because of certain new philosophical
theories concerning his origin.

We were greatly struck while driving one day through the wonderful streets
of Benares, the sacred city of the Ganges, when we passed a broad Hindoo temple
towering above a high inclosing wall, for the temple and the court surrounding it
were crowded with monkeys of all ages, sizes and colors. They climbed up the
carved pillars, raced round the architrave, scampered over the roof and hung by
the tail from all sorts of impossible and dangerous‐looking places; while as for the
streets and by‐ways, in the vicinity of that temple, they were as closely inhabited
by monkeys, little and big, as the alley‐ways in New York are by cats.

This heathen temple was dedicated to the worship of these active little creatures who are believed by the Hindoos to be no less than “gods and goddesses.”

One meek and gentle native told us why monkeys are held to be divine. Said
the Hindoo: “They are the most trying of all creatures to the \emph{patience} of men, and
they are worshiped and guarded by devoted Hindoos who seek perfection in the
practice of this noble virtue.”

It was a new idea to us that a “trying and mischievous disposition” could in
any sense be thought admirable, or that one needed to leave the ordinary path
of life to learn the practice of patience. We therefore wondered to see hundreds
of Hindoos who, coming from afar, made their homes near this temple, fed the
monkeys, cared for them, made them welcome everywhere, even blessed the
ungrateful naughty little things and left them unmolested in their cunning tricks
and scampish ways.

The “sacred monkeys” were so funny and so fascinating, my sister and I rather
regretted that the course of our lives was not likely to offer us a closer study of
the attributes so highly prized by the soft‐eyed Hindoos.

There is a belief in the East that anything much wished for is sure to come to
pass and our wish to know more about monkeys was by way of being gratified
sooner than we could have believed.

The first three hundred miles of our journey in Egypt were by railroad to the
landing‐place on the river where a steamer waited to take us far beyond on a
journey up the Nile.

We halted at Ghizeh, a station on the way, where an Arab, acting as United
States Consul, looked after the comfort and rights of such of our countrymen
as passed that way. This consul came to greet us, and brought to the railway
carriage the monkey hero of this story, who was about the size and color of a
large gray squirrel. He had a long curled tail and his tiny grayish paws were
lined with what looked like the very softest and finest of black kid. His little
tapering finger‐nails were smooth and polished. His teeth were very white and
even and closed firmly together. He had a slightly “jowlish” look, the effect of the
mouth‐pouches provided to hold the surplus when he had more goodies than he
could eat at one time. His eyes were “large, brown and expressive,” and he wore
little side‐whiskers perfectly white and as evenly trimmed and curled as those of
an English butler.

He stood in the carriage window and gave us a Turkish salute or salaam,
bowing and touching the sill with his paw, raising it to his forehead and mouth
and then crossing his hands on his little stomach, after which salutation he sprang
up to the rack where our bags, rugs and lunch basket were stowed, and watched
us curiously from over the top of the luggage.

The consul said this was a very rare kind of monkey, brought from Abyssinia;
that he carried him about in his pocket and was fond of him.

When the engine whistled and the train was about starting, we looked for
the monkey to give him to his master, and found the little thing wound into the
rack in such a way that he could not be disentangled quickly without cutting the
netting or hurting him. “Never mind,” said the good‐natured and generous Arab,
“since you like him, and he evidently wishes to stay, let me give him to you. He is
very docile and amusing, a faultless pet that I should be delighted to present to
you.”

We smiled our thanks as the train moved off, and there being no time to ask
the name of our unexpected little traveling companion we called him “Nap,” in
honor of Lord Napier of Magdala in Abyssinia, and in memory of Napoleon in
Egypt.

There was so much to think of in the ancient land through which we were
passing that for the moment we naturally gave no further attention to the ex‐pet
of the American consulate.

The voyage of the Nile is always a solemn and impressive journey, and it
was more than usually sombre in the late season of May when we made it. The
palm‐trees and green patches by the river side were already dry and browned,
and the Nile was low and turbid, while the sands of the trackless desert were
shadeless and shining under the rays of a glaring sun. Just before us, and outlined
against the burning sky, were the unbending forms of the pyramids from which
the centuries have looked down not only on Bonaparte and his army, but on the
whole civilized world of history.

There stood the Sphinx, heedless as she has ever been of the million questions which winds of ages have whispered against her stony cheeks, and our
train wended its way still further along the rush‐covered river‐banks, where the
marvelous cradle was woven hundreds of years ago, and the little Moses rocked
among the reeds growing just as we saw them now. Looking across the great
desert, we dreamed of Arabs, and tents, and caravans, and oases, and mirages,
and adventures until, reminded by hunger of the dinner hour, our attention was
brought back to the railway‐carriage. We asked for the lunch basket and our
surprise may be imagined when it was found fastened to the rack, and our dismay
when the monkey was discovered holding it fast.

Busy indeed that small Nap had been while we were engaged in harmless
reveries. So ingeniously had he wound and threaded his long tail into the rack,
his neck and limbs into the basket, that it was immovable. He was lying partly in
the basket braced by his strong legs, while his nervous arms and cunning fingers
were free to rummage among the good things he found there. Complete master of
the situation he leisurely regaled himself and had a way of rejecting the things he
did not like and of gobbling those that were to his taste, which was most “trying
indeed to the patience of men.” Our discomfort was increased to real suffering
when he managed to open the water bottle most carefully provided for our hot
dusty journey, and before we could prevent it, upset the contents over the rugs
and wraps.

There was nothing to do but grin and bear this unwelcome discipline as every
moment we became hungrier and thirstier. The expediency of choking the little
gourmand was discussed, but we rejected this as inhuman, and at the slightest
advance he showed his sharp teeth in such a way as to convince us that he could
cause us to suffer worse pangs than those of hunger if he were once roused to
anger, and we were forced to submit unwillingly to the first lesson in forbearance
of which Nap was to teach us so many. The remaining hours were passed in
a famishing silence, while the monkey feasted on ouf viands, alternating this
occupation with the pastime of pelting us with balls of bread, half‐squeezed
oranges, bitten bananas, grape‐skins and egg‐shells, taking his aim and pitching
with a precision which even Mike Kelly might be glad to imitate.

When at length his ravenous little appetite was satisfied, and the remains
of the feast scattered, the gray monkey made his way coolly down to the seat,
looked out at the window, and after holding my finger in a friendly way for awhile
curled himself up and went to sleep — the picture of docile innocence. We began
to reflect that the poor little thing must have been nearly starved and to excuse
his rudeness on this account, and furthermore determined to take such good care
of him in the future that a repetition of such behavior would be impossible.

At Rodah we changed from the railway train to a large steam yacht, the
{\it Crocodile}, on board of which we expected to pass the next fortnight exploring the
Nile.

On embarking our first care was to provide quarters for little Nap, and we
were not long deciding that the deck was the place where he would have the most
freedom and do the least mischief. He already had a strong leather belt fastened
around his body and a little chain. To this we added a long chain fastened to
the flagstaff. Later, a little house was contrived for him under one of the ship’s
benches, while the ship’s master and crew were strictly charged to be very kind
to him.

Mr.~Seward, who was an inveterate student at home, relaxed none of these
habits while traveling abroad, and he also kept us up to the mark in study, sight‐seeing and note‐taking in every country through which we passed. We knew this
voyage up the Nile would be our most serious undertaking in point of study and
we arranged our state‐rooms with special regard for this object. Guide‐books and
dictionaries were taken out, and paper and blanks, pen, ink and pencils gotten
ready for work, and a convenient place found for each separate article we might
use or need.

Coming to a stopping‐place, our first excursion was made on donkey‐back
to Abydos, the stupendous ruins of an ancient temple built between five and six
thousand years ago, and where a stone tablet on one of the interior walls shows
the carved chronological record of seventy‐six Egyptian kings, beginning with
Menes and ending with Setis. This list is rather hard to learn to say backward,
but it was probably as easy to the little Egyptian students mummied thousands of
years ago, as it is to‐day for an American schoolboy to begin with the name of
Benjamin Harrison and repeat back to George Washington without missing in
the list of presidents.

After so much serious study we were glad to get back to the steamer and to
have a merry romp with the monkey. He was full of fun and quick at learning
a game, and raced after us until we went to the cabin, leaving him on deck, and
then he went to the railing round the ship, and assumed an attentive attitude,
as if thoughtfully listening to the sound of our voices coming up from the open
port‐holes.

The next day we went ashore to visit the temple of Denderah, and here we
found something more modern than the ruins at Abydos.

The temple at Denderah belongs to the Ptolemaic period and part of it was
built at so late a date as the reign of the Roman Tiberius. It is full of names
of the Cæsars, among them Caligula, Claudius, Nero and Augustus, and on
one of the back walls is shown a veritable life‐size portrait in intaglio of the
boy‐prince Ptolemy‐Cæsarion, taking a social walk with his celebrated mamma
Queen Cleopatra.

These excursions to tombs and temples were made at night, for the sun was too
hot by day for strangers to venture out. When we returned in the early dawn from
Denderah, and “sought the repose the cabin grants,” what a scene of confusion
met our eyes! The books and papers we had so carefully arranged lay scattered
on the floor, torn in strips; odd boots and shoes were lying about everywhere (we
discovered later that the missing fellows had been thrown out at the port‐holes
into the Nile, with numerous other valuables); soapy water had been upset in
the cabin berths, and our beloved pith hats torn to shreds and trampled into the
matting, while Mr.~Nap, the author of this mischief, was gleefully chattering and
gesticulating on the hand‐rail above, where the day before he had demurely laid
his destructive plan of campaign.

Of course we “went for him,” but before we reached the deck he had disappeared over the railing and through the port‐hole into the state‐room; and when
we were in the state‐room it was impossible to catch him there, for he bounded
at one jump out at the port‐hole over the railing on to the deck again. Lesson
number two in patience and forbearance from Nap — but those lessons shortly
became numberless.

He was one of those creatures who are never conscious of doing wrong, or of
making any one uncomfortable, and had an air about him as much as to say, “I am
so fine and rare a monkey that you must admire even the worst of my monkey
tricks.” But he was so droll and gentle that we grew more and more to like him,
notwithstanding all his pranks and apish vanity.

Our dining‐table was spread on the deck of the {\it Crocodile} which was the
Khédive’s private yacht, and dinner was served with the same observance of
Turkish etiquette, including the ceremonious passing of coffee and chibouks, as if
the Khédive himself were on board.

Nap was always at hand at dinner‐time to catch the tidbits given him, or get
such other dainties as he could prig unseen. His greedy ways were endlessly
funny and his liking for sweetmeats prodigious. He had a special taste for sugared
almonds, and watched eagerly for these hard brittle bonbons. He stowed them
away like a miser, four at a time, when they were given to him, gripping them
testily with his feet or hind paws, packing them in his cheek‐pouches and his
mouth, and clutching them in his hands. Only when he was persuaded that he
would get no more until these twenty‐eight were eaten would he sit upon his
haunches and begin crunching and munching, one by one, in the reverse order in
which he had packed them away — beginning with the almonds in his mouth and
eating those held in his hind paws last.

Nap was also a thirsty fellow. A contrivance for freshening the deck and
also for cooling purposes, so indispensable in a climate where the thermometer
ranged from 130° to 140°, and ice unknown, was very convenient for Nap also. A
trough about ten inches wide and a foot deep ran around the edge of the deck,
and Nile water pumped by machinery from the river was kept running rapidly
in the trough day and night. Here Nap stooped and drank from his hand like a
schoolboy, or lapped the cool stream like a dog. Here wine bottles were kept, and
here Mr.~Nap learned some very bad tricks; how he learned to pick and pull off
the metal covering and then to nibble out the cork of the claret bottle we never
knew — but more than once we found him grasping the neck of an upset bottle,
and holding his tiny mouth to catch the crimson stream. We fancied he longed
for the grape‐juice of his native land, but this charitable view was not taken by
the less indulgent steward when his best bottle of sweet rich Château Yquem
was found upset and empty and Nap staggering about the deck in a high state of
merriment, dropping off at last in the senseless stupor of intoxication.

By the time we reached the cataracts of the Nile we had often wished Nap at
Benares among those monkey “gods and goddesses,” teaching patience to Hindoo
devotees. We had tried different modes of discipline in vain, and the rogue had
indulged in every species of prank, caprice and mischief which monkeyhood could
suggest. He had at last gnawed and picked our only remaining toothbrushes to
pieces, mimicking their proper use meanwhile from his perch in the port‐hole,
and had spilled our last bottle of ink, carrying it to the top of the flag‐staff and
letting its contents trickle down among the stars and stripes, giving our beloved
banner the character of a “black flag” before he was discovered, for the top of the
flag‐staff was his favorite resting‐place and last retreat.

The officers on the steamer, staid, silent Moslems, despised the Abyssinian
monkey heartily, and he showed them as little respect in return.

All Turks and Egyptians from the Sultan on his throne at Stamboul to the
fellah in the desert of Sahara, wear one and the same style of head‐covering — the
red cloth fez or tarboosh, and are never seen without it. This fez has superseded
the turban of the time of Sindbad, and is so revered in the eyes of all Mussulmans
as an emblem, that it is invariably carved on their tombstones. Every day after
our dinner a squad of officers in fez and military uniform came marching around
the deck, each carrying a pipe several feet long, having an amber mouth‐piece
inlaid with gold and precious stones, which they set in silver trays at the feet of
each guest. These were the Khédive’s chibouk‐bearers, and they offered the pipes
of peace and good‐will with great stateliness and decorum. Nap regarded this
ceremony with fierce contempt and disfavor. Perhaps the smoke made him sick,
for he always scampered away when the pipe‐bearers appeared and chattered at
them from the flag‐staff. One evening, however, he seemed impelled to a madder
prank than usual, for he swung himself down noiselessly and sprang without
warning to the back of one of these officers, seized his fez by the black tassel and
threw it into the river. The agile little rascal was at the top of the flag‐staff again
before the officer could realize that his insignia of dignity was gone, and what an
absurd figure he made standing pipe in hand with only the yellow pasteboard rim
around his head which, “dude” that he was, he wore concealed under the clumsy
fez to keep it in shape. The officer was furious and it was plain to see that he
inwardly vowed to have his revenge on Nap.

His chance came the very next day when the steward, who had not forgotten
the loss of the Château Yquem, caught the monkey slyly picking the metal casing
from a wine bottle, and, seizing him by the throat, shortened his chain, and tied
the poor fellow up to a stanchion. The injured officer of the stolen fez passing at
that moment gave the helpless mite a cruel cut with a rattan, and simultaneously
my ears were greeted by his piteous cries. Not long coming to his rescue I seized
the rattan which the officer dropped at my approach and before even throwing it
overboard I unfastened the howling little prisoner.

We had never seen him other than gentle and good‐natured with all his
mischievous cunning; but now he was shaking with rage. He lashed his tail about
savagely, struck at the air with tiny clinched fists, while his whole body trembled
and his eyes were glazed and gray with anger. His great pain and fury must
have turned his small brain, for spying the instrument of torture in my hand,
he sprang at me with a vengeance. In one second his four sinewy limbs were
clutched around my arm protected only by a lace sleeve, and his double row of
sharp perfect teeth were buried in my flesh. The steward caught a large knife from
the dining‐table and raised it to cut Nap’s head off, but our good Freeman was
luckily at hand to choke the clinging animal instead, and he dropped senseless to
the floor. Not before he had left a mark on my arm however, a miniature copy of
that indented circle which boys like to make in a ripe pound sweeting, and all
hands gathered round to care for my hurt.

Meanwhile Nap recovered and scrambled out of the way, with his belt loosened
and a part of the chain gone. The knife, nearly twice as long as the monkey’s whole
body, lay forgotten on the floor. He grabbed it stealthily, and tucking it under
one arm, scaled the flag‐staff waving his wicked weapon aloft, and chattering
his grievances and his triumphant escape long and loud from the top of that
liberty‐pole.

The spirit of revenge is often mistaken for manliness and courage, but if any
one who cherishes it could see an enraged monkey, sullenly nursing his stripes
and waiting a secret chance to strike back with a big carving knife, he would
banish such ideas of that ignoble passion. Even little savage Nap grew tired of
resentment as the day darkened and a storm came on, and dropping the knife,
began slowly sliding down the pole, and started humbly on all‐fours for his
friends.

Not so forgiving that Mohammedan pipe‐bearer whose dignity Nap had overturned. It was his time now, and more sly than the monkey he waited until the
little fellow was near, then grabbed him by the remaining bit of chain hanging to
his strained belt.

The monkey, as if knowing his hour had come unless he could escape altogether, made one desperate squirming struggle, and the belt snapping in two he
sprang head‐first into the dark fast‐flowing river.

We were grieved to lose the poor pet, and as the storm gathered pulled up
and fastened to a landing to offer rewards for Nap in the Arab villages, hoping
his irrepressible monkeyship had managed to swim ashore; but no trace of him
could be found, and giving him up for lost we proceeded on the voyage.

At dawn the next morning my sister and I were pacing the deck, recalling
Nap’s sad fate and pretty ways when we suddenly saw the humped‐up figure of
our miserable little fellow‐traveler, on the middle seat of the large empty life‐boat
following the steamer’s wake at the distance of a long rope’s length. He was
drenched and shivering, but when nearly perished from cold and fright, had
had cleverness enough in the moment of his dire extremity to strike out for the
life‐boat and rescue himself from a watery grave.

After this he was given in charge of a good‐natured sailor whose sole duty
it was to look after him, and Mr.~Seward was greatly disappointed to learn that
our downward trip would not permit us to stop at Ghizeh for a call on the Arab
American consul who thought the Abyssinian monkey a faultless and desirable
pet.

When we arrived at Cairo Nap was so far restored to favor that we hoped
Mr.~Seward would be willing to let us carry him to America; but he was of the
opinion that we wasted valuable time playing with Nap, that he was a great care,
and would prove an intolerable nuisance in ordinary travel. We tried to give him
away at Cairo, but could find no one who appreciated him, and he followed us
in the baggage car to Alexandria. Fearing to leave him unprotected in the hotel
there, we invariably took him on the carriage box in our drives, and in this way
Nap visited all the sights of Alexandria, and saw Pompey’s pillar and Cleopatra’s
Needle even as he had before seen many of the ruins and obelisks of the Nile, in
our company.

Mr.~Seward was fond of pets and very considerate of others in their friendships
for the dumb creatures; but he became stern about the Abyssinian monkey, and
at Alexandria he said we should not take him with us to the Holy Land, to Greece
and to the other countries we hoped to see before going home, and made an
agreement with the master of the hotel by which the latter promised to take the
little fellow off our hands and be kind to him. And so we bade farewell to our
Abyssinian pet, and set sail without him from the continent of Africa.

Our steamer was lying at anchor in the bay off Alexandria, and as we stood
on her deck gazing our last at the ancient city with the Pharos light in full view,
we saw a steam launch carrying the American flag put off from shore and make
directly for our vessel. As it came near we recognized a gay familiar little figure,
perched at the top of the flag‐staff. In the innkeeper’s absence the innkeeper’s
wife, finding Nap moaning alone in our deserted apartment, not knowing of the
contract which her husband had made, had taken pity on the forsaken monkey,
and in the kindness of her heart forwarded him to us by special conveyance.

His moans were changed to such chirrups of delight at recognizing his traveling‐companions that we thought Mr.~Seward would surely relent. But he was
obdurate, and Nap was returned with a kind message of thanks to the innkeeper’s
wife, and the explanation that as it would be at least three months before we
sailed from England for America we could not possibly take the monkey with us.

When we came at last to England and were about starting on the final voyage
of our long journey, many things had diverted our thoughts from Nap; but the
sight of hand‐organ monkeys and the little apes in the British Zoo had never
failed to produce with us pangs of regretful memory. On the day we were to leave
London for Liverpool a sailor‐like looking man appeared at our hotel, carrying a
monkey, whom at a glance we could not fail to recognize as at least a relative of
Nap. The same size, pretty gray color, natty white whiskers, big restless eyes and
indeed, as we soon found out, the very same Nap.

Those over‐kind hotel people at Alexandria had kept our address and the date
when we were to be in England; and at a loss what else to do with the valuable
charge we had left forwarded the Abyssinian monkey by the hands of a sailor
who now claimed his hardly‐won reward, and Mr.~Nap was thereupon ensconced
entirely to his own satisfaction at least in first‐class quarters at Fenton’s, St.~James.

Mr.~Seward was now confounded and my sister and I perplexed, for much
as we liked him we were forced to admit that we could manage this persistent
little creature even less well now than when we left him in Egypt. Mr.~Seward’s
cherished friend, Mr.~Evarts, who happened to be in London at the time, came
to the rescue. He had promised to take a nice little animal of some sort from
England to his youngest son at home in America, and after hearing as much of
the monkey’s history as we could tell in a few moments, saw him, liked him, and
was pleased to accept him as a present, and a rare find for his purpose.

Nap’s next home was therefore on Second Avenue, New York, varied possibly
with journeys to Windsor, Vermont, and to Washington, and a history of his
American experience would doubtless be hailed with interest by the Nap family
in Abyssinia to whom it would seem a barbarous romance.

It is now known that before many years had passed Master Evarts presented
a “rare specimen” of Abyssinian monkey to the Zoölogical collection in Central
Park, New York. Here Nap was greatly sought and admired among others of
his race less distinguished than our little traveler, and here he may have had the
pleasure a little later of examining again the obelisk of Cleopatra’s Needle which
he had seen from our carriage‐box in far‐away Egypt and which was brought,
like Nap, all the way from Africa to adorn the great pleasure park.

It has been intimated to me that Master Evarts’ mother had something to
do with suggesting the propriety of this precious gift to Central Park on the
part of her son; and in a recent conversation with Senator Evarts about Hindoo
monkey‐worship and Darwinian theories that grave gentleman told me, with a
humorous twinkle in his eyes, that whatever peculiar importance Indian fanatics
and certain disciples of modern thought might give to the race of apes, he was very
sure that for Mr.~Seward and himself they had learned, in their experience with
one Abyssinian monkey, the profound truth of at least one great and universal
precept of moral philosophy — the precept which teaches that “it is better to give
than to receive.”

\chapter{A Dogocracy.}

There is a city by a sea which all writers have agreed in calling the most beautiful
place in the world. The sea, eighteen miles long and a mile and a half wide, is
hardly more than a toy sea, it is so small; but it is deep and blue as the ocean, it
winds and flows like a river, and is a channel, an inlet and a harbor, all in one. As
a channel it separates two continents; as an inlet it unites two seas; its harbor
would anchor all the war‐vessels in the world, but it sparkles with the burden
of the frailest pleasure yachts, boys launching their hand‐made sail‐boats on its
bosom, as though it were a placid pond.

The shores of this fairy sea, unlike the sandy beaches or rocky coasts of other
seas, rise high and lofty or follow undulating curves, and are clothed from water’s
edge to summit’s height in richest verdure.

The city clusters in triple beauty on these green banks, and reflects a thousand
matchless shapes of mosque and palace tower in the depths of the blue sea. Its
domes and minarets; all white and gold, gleam like jewels, and are vine‐crowned
and garlanded in bright plane‐tree groves and shadowy cypress avenues.

This sea is the Bosporus, and this city is Constantinople, the chief city of the
Ottoman Empire.

This story however is not about scenery nor about Turkey, but is about some
of the inhabitants of the great capital where natural grandeur and architectural
beauty combine to make a splendid stage for human dramas, a brilliant framework
for the changing scenes depicted in the daily life of the men, the women, the
children and the animals who have their homes there; for these pictures and
scenes tell the character of the town and what it means, and that is not perhaps
quite what it looks like.

The Emperor, who is also a priest and king, lives in the grandest of the palaces,
sails in the most gorgeous yachts, and is ruler of all the mosques. His titles are
many and imposing. He is Grand‐seignor, Khan, and Hukian, Sultan and Padishah,
and he is the absolute monarch of one of the most powerful despotic governments
in the world.

But this story is not about the Sultan either, only about the life, the ways and
doings, sad and jolly, of the humblest of his dependents — the dogs of Constantinople — who have formed a sort of free republic under the very walls of the Sultan’s
palace and in the shadow of the mosques.

All the dogs in Turkey seem to have gathered here; but when or why they
came, no one seems to know. Once upon a time an energetic ruler, Sultan Abdul
Medjid, ordered them all carried away to an island in the sea, but they made their
way back again in greater numbers than before, and ever since that time they
have held their independent place, such as it is, in the city of the Caliphs, and
now there are many thousands of them.

Dogs are found wherever the foot of man has trod, but it is only in Christian
countries that they become the loved and loving companions of man’s joys and
sorrows. In all Mohammedan lands, the dog race is the one most despised, and
by Turks especially a dog is counted a pariah and “unclean.”

This degradation, which seems to have reached its lowest limit in Constantinople, has forced the dogs there into a singular state of independence, and as this
condition is the result of some of the poorest and most ignoble traits of human
character, so some of the noblest instincts of dog‐nature here rise up and assert
themselves.

It is not only a principle of the Mohammedan religion to despise dogs, but
the Turks also have a superstition about “getting the ill‐will” of one of these
dejected creatures, which is only equalled in absurdity by another belief which no
Mohammedan is ashamed to own — that “bad luck” will surely and mysteriously
follow any one who kills a dog, though good luck will not desert one, in the Turk’s
opinion, who makes a dog’s life miserable. A Turk therefore who aims a blow, a
kick or a stone at a dog will be careful not to hit a vital spot, and while the air is
rent continually night and day in Constantinople with their howls of pain and
anger, and dogs are often mained and bruised as well as half starved, they are as
a rule long‐lived and healthy.

How can it be possible for creatures so abhorred and downtrodden, dumb
creatures, too, and the most dependent of all on the kindness of man, to have
banded together and to have formed a union which claims and one that obtains
respect even from the Turks, in the face of this bitter hatred and prejudice? Yet
this is the simple fact about the dogs of Constantinople.

Permitted to live by the superstition ef the Turks, they have made a dogful
struggle to maintain their rights to “liberty and the pursuit of” such “happiness”
as they can secure for themselves, and this struggle has resulted in a Dogocracy
for mutual offence and defence, to guard against the cruelty and injustice of that
race they were born to serve with loving fidelity.

Deprived of their natural allegiance, and still true to their dumb instinct of
devotion, they have banded together, not only to defend their own rights, but have
united in a forlorn brotherhood of suffering and endurance which still enables
them to guard and protect the race they love better than their own.

The citizens of this dog‐government are dogs of all sizes and colors, chiefly
wolf‐dogs, though abound and mastiff blood and a large proportion of the pert
Pomeranian may be traced among them. This last is the wild “pariah” of the
desert, which when jet‐black, bleached white, or yellow and refined, we call the
“Spitz.”

Nameless, homeless and masterless, these dogs live, eat and sleep in weather
foul or fair, in the crooked winding streets of the beautiful city. They make their
homes in old boxes, behind walls or under curb‐stones, or these failing, stretch
themselves shelterless along the pavement, crowding the highways, blocking
the by‐ways, threatening every path. It is no uncommon sight to see a litter of
puppies nestling with their vigilant mother on a few wisps of straw, scratched
together on the rough cobblestones of a sunny thoroughfare, where with the
help of an even more savage father the family turns the tide of traffic and travel,
disputing the right of way with every man or beast who comes along.

But there is law and order in this apparently heterogeneous mass of canines.
They form a strong and trustworthy under‐police guard, and are organized in
bands as scavengers and grubbers. They have ordered among themselves a regular
patrol of the slanting irregular town, which they have divided into sections and
wards, and the dog authorities have allotted each division to the care of a certain
tribe or breed of dogs.

Each section, street and square has its apportioned number; each division of
dogs chooses or accepts its leader‐dog, the most aggressive or sagacious of the
tribe, and this chief has his lieutenants of different grades, and they train squads,
marshal companies and seem to hold themselves responsible for a good dog‐drill.

These troopers guard goods and premises, drive away all strange or suspicious‐looking people from their precincts, protect litters of puppies and their
mothers, gather and clear away garbage, hunt “mice and rats and such small
deer,” look after the old and maimed dogs of their streets, dressing and curing
their wounds, dog‐fashion, and are often seen carrying food to the sick ones in
their neighborhoods. They settle dog‐disputes too, usually by a general all‐round
shaking of the quarrelers, and above all keep vagrant or sneak dogs of other
sections from invading premises not their own.

The people know the dogs of their neighborhoods and trust so entirely to their
sagacity and energy that bolts and burglars are alike unknown in Constantinople,
and garbage carts and “soap‐fat” are unhonored and unsung.

A Stamboul boy knows instantly when a strange dog appears in his street If
he is kindly disposed he will give that dog a sign of warning or get him out of
the way. A different sort of fellow will wait to “see the fun,” when the dog police
captain of that precinct scents the intruder, and leads his dog‐patrol to hunt him
out. For woe betides the long‐haired brown dog who is found poaching on the
manor of the short‐eared yellow dog, or any other who is found where he does
not belong. If he has the temerity to stand and give fight worse woe, for they
are too many against him and the row ends with a splash in the Bosporus. If a
wretch turns tail and takes to his heels, the pack will catch and worry him, for
a caution not to come again. But if the dog is clever, and they oftenest are, he
rolls over on his back, makes a “rocker of his spine,” lays back bis ears, hangs out
his tongue — in short gives in and the fight is over. The dog‐police will not hit a
fellow who is down, but will quietly allow him to pull himself together and hustle
off as fast as his trembling legs will carry him.

But if the chase carries the pursuer near the culprit’s own quarters, the whole
canine population of that district turns out in defence of its hunted member, and
thereupon a mêlée ensues which loyalty to dogs compels me to cover with a veil
of charity.

It is believed that these wandering dogs are idle or greedy drones, driven
out into the world by their fellows to learn better. It is also noticed that they
often return cured of certain envious and roaming dispositions. If a dog returns
with a few scars, they are apparently accounted honorable; the dog is reinstated
and given a new chance. But if he once plainly deserts his own pack or betrays
his friends he is hunted without mercy from among their ranks; is welcomed
nowhere, and from that moment that dog is known by all to have had “his day.”

Our hotel was on a hillside in Constantinople and faced a short narrow street;
on the day of our arrival, we counted thirty‐seven dogs lying on the pavement, or
wandering up and down, all gaunt, hungry‐looking and forlorn. When we went
near them they showed none of that anxious expectancy or frank joyousness with
which all other dogs naturally recognize their friends, though they were alert and
watchful.

When we left the hotel the next day to see the sights of the town, not yet
knowing the rules of the dog republic, we were surprised to find the same dog
company, none missing and not one added to their number, in the same street, as
hungry and surly as before. We called and whistled to them, but they hung their
heads, dropped their tails and made off, casting suspicious looks back over their
lean shoulders.

We now learned that these same dogs lived in this particular street, and that
travelers, English and American, sometimes took pity on them and fed them there.
They were never without a hope therefore when strangers appeared that a “piece”
might be thrown their way. They knew nothing to speak of about meat, and bread
was their staff of life, a broken staff, poor fellows.

We got some loaves and offered them to the dogs, and in the names of our
dear pampered ones at home — “Leo” the Newfoundland who always preferred
the tenderloin, “Pearl” the Spitz who doted on jumbles, and “Piccina” the King
Charles who would scent a difference between two Strasburg pâtés and choose the
best — we tried to make their vagabond cousins, these poor pariahs, understand
that we were their friends, too, but it was useless. Their ears made no response,
their tails hung speechless.

Thinking them shy, we left the bread on the nearest curb‐stone, but with no
better result.

To take it from there, and unbroken, was clearly against their rules. The big
leader‐dogs, patriarchs of the pack, waited sullen and motionless; but the little
Pomeranians standing in the rear showed, by the nervous movement of their
slim paws, as they shifted from one slender leg to the other, and twitched their
ears, that they at least were impatient. One who had learned to understand the
laws of the dogocracy, by watching the ways of its citizens, told us that they only
feel at liberty to eat what is thrown away as refuse, and so it proved with our
patrol‐guard. When the loaves were broken and flung roughly into the middle of
the street — and only then — could the dogs believe that the food was really theirs.
Then with one growling rush it disappeared, and only the crumbs were left for
the little dogs to squabble over.

When we returned from our walk, the same scene was enacted by the same
thirty‐seven dogs, a few of whom were tall starved‐looking fellows with unkempt
coats; the pack ranging in size down to a crooked‐legged yellow pigmy of no
name or pedigree, whose ears, nearly as long as his thick tail, stuck straight up
like a jackrabbit’s and never could come down, but whose appealing eyes quite
made up for his general lack of beauty.

We fed this pack all they could eat several times a day for three weeks, and
summoned all the powers of persuasion which a long acquaintance with dogs
and an intimate knowledge of their ways and weaknesses could suggest, to make
friends with them. At first we met with little encouragement. Strange to say,
these Constantinople dogs were always longing and eager to be friendly with
Turks, but are distrustful of foreigners. The best dogs were slowest to respond.
They finally condescended to a slight letting up of their reserve. The hounds
became more natural and cringed, the wolf‐dogs fawned and smiled, the mastiffs
unbent, and the tender‐hearted little yellow, brown or brindled creatures of the
pack came quite near, laid back their eats deprecatingly, showed the whites of
their eyes, and wagged their little foxy tails as only Pomeranians can wag.

By the end of the first week however we had gained the good will of all. When
we went out on foot or in carriage, the whole lean motley crowd of dogs followed
to the farthest corner of the street, but never a step beyond. When we returned,
some of them were, sure to be there to greet us affectionately and would then trot
off to let the others know that the bread‐people were coming.

Meantime three that we did not see the first day had been added to the
pack — one big lout of a dog whose wounds from a recent fight were not yet
healed, and who had probably been in hospital, one lank care‐worn mother whose
young ones had been too little, to leave, and one small blind invalid who had to
be coaxed to the feast, making in all a company, of forty dogs, who never, by any
chance, left that street, and never, so far as we saw, fought among themselves.
No one spoke to or noticed them, though we found them exceedingly intelligent
and as capable of understanding and reciprocating friendship as any dogs we had
ever known and whose kind grateful eyes, following us sadly as we drove away
at last, can never be forgotten. All days are sad and dismal enough for these poor
dogs in Constantinople, but one day in the week is worse than the others — Friday,
when the Sultan makes the grand pageant of his royal state and goes in all the
pomp of his double title of Emperor and Caliph to pray in the great mosque. The
Sultan is resplendent in his gold‐embroidered uniform, riding a tall white Arab
steed, preceded by dazzling troops and trumpeters. Long lines of cavalry and
detachments of artillery and all manner of gorgeous military follow in his train.
Guns are fired and bells are rung. The whole town turns out to witness the parade.
All the streets through which the monarch passes are crowded by the faithful
followers of Mohammed and the loyal subjects of the King — that is, all but the
dogs. Long before the hour for the procession to pass they disappear somewhere,
as far away from the sound of the guns, drums and trumpets as they can get; and
on this occasion the dogs into whose sections they retire never drive them out
It is evidently understood among the citizens of the Dogocracy that they must
sustain each other in discountenancing all pretentious and autocratic customs.

Any mutual interest will unite these Constantinople dogs. A common sentiment will band the scattered tribes in one. They have no knowledge of a common
friend, but a common enemy rouses their loyalty to their own race, and they unite
to defend each other in their sorry homes.

We had halted in our carriage one day to count the brazen rings which bind
the stones of an ancient pillar — the oldest relic in Constantinople, bearing the
date of the Emperor Constantine, which stands in the centre of a broad paved
square at the intersection of several streets. The usual fifty to a hundred dogs
were prowling around or lying in the sun, when two large rough‐looking hounds
came yapping into the square, their haggard eyes and dripping tongues telling
their story of a long thirsty race. They began a series of sharp imperious yelps
and were quickly surrounded by all the dogs in the square, who listened snarling,
and barking by turn.

The tumult was so great that we tried to drive out of it, but all the streets
leading to the square were crowded with dogs coming from every quarter, running
so furiously in and out among the horses’ heels that the carriage could not move.
We marveled at the savage scene, but the natives, detained as we were in the
square, seemed more vexed than surprised, and resigned themselves to wait with
a bad grace.

One unconscious man who was making his way in from the country, expecting
to turn an honest penny by an exhibition of trained animals in this very square, had
good reason to prick up \emph{his} ears and to inquire into the cause of this commotion.
The key to the enigma was in our hands when there suddenly appeared from over
the brow of the hill the tall athletic form of this same festive Turk, arrayed in a
red fez, red Turkish trousers, and with bare legs Of about the same shade. This
clown was followed by his son and prototype. The father led a bear by a ring in
the nose, and the son drove a wolf in front muzzled and held by a chain.

As soon as the man and the bear, the wolf and the boy, appeared over the
hill there was a hurried consultation among the dogs, followed by a prolonged
and general howl, and war to the teeth was plainly declared. The two dogs
who had raced in from the outskirts to herald the news of the coming invasion,
seemed chosen leaders of the first sortie, and a furious pack, uttering horrid yells,
dashed up the hill, and we thought shudderingly of the bare legs of those unlucky
showmen. The other dogs stayed behind to guard the square perhaps, or to be
ready to reinforce the attacking party if the wolf and the bear gave fight la a
few moments the square and all the adjacent streets were solidly packed with a
howling excited mob of dogs — not however fighting among themselves.

A subtle instinct had told them that a mortal enemy was near, and had roused
the despised and downtrodden race; all its hostile tribes and clans, its unknown
packs and divisions far and near, from the most domineering mastiffs down to
the smallest mongrel puppies, had gathered to repel a common foe.

Fortunately their manœuvres were understood by the real police and by the
human inhabitants. The Turks think nothing so fascinating as a dancing bear
with a pole in his paws and a ring in his nose, and next to this amusement their
delight is in watching the gambols of a well‐muzzled wolf. But a bear and a wolf
together is a pastime not often enjoyed. The country showmen, unprepared for
a dog reception, were safely smuggled into a coffee house, and the bear and the
wolf were rescued and locked up in the walled garden of a fort near by, to wait
until a secure and suitable theater could be found for this classic entertainment.

The dog picket guard returned to the square and evidently reported that the
enemy had mysteriously disappeared before the battle was begun. The gaunt and
disappointed army dejectedly disbanded, the different packs dispersed, and we
were able at last to make our way through their scattering ranks to the hotel.

There was no sleep in Constantinople for any one that night. The dogs were
pattering up and down the streets, either celebrating their escape, or warning
each other that the foe was still in ambush somewhere near. All through the long
hours the streets of the beautiful city resounded with the barks and howls, the
yaps and inarticulate yelps of the canine chorus which were echoed back from
the depths of the flowing sea.

As all travelers unite in admitting the incomparable beauty of Constantinople,
so do they agree that the like of a Constantinople dog‐serenade can never be
heard again.

But I was once reminded of this serenade and here in America too. It was at
New London, at the time of a college boat race, between two freshmen crews.
There was a mighty victory and a great defeat on hand, in the eyes of the college
boys, and naturally there could be but little sleep for any one. A vivid, half‐waking
dream carried me away from the banks of the Connecticut to the distant Bosporus
shores, and the dogs of Constantinople seemed baying about me again, but
melodiously, and the chorus, clear and articulate now, had lost all cadence of
hunger and chagrin.

“There’s a new coon in town,” were the words of the chorus. “Over the garden
wall,” rang the warning from many hoarse throats, and “Climbing up the golden
stair,” in the exuberant tones of youth and health and fun were all mingled with
college cheers, joyous barks of dogs, and the sound of hurrying feet.

One voice was missing from the uproar the night after the race, which should
have been the loudest and the merriest of them all — the voice of the captain and
stroke of the winning crew. He had left the carnival early to drive several miles
alone through a pelting rain to look after the pet of the crew, a dear fox‐terrier
puppy who had been left at the boat‐house ill before the race came off, and there
he was now found quite safe by his loving young master, the freshman captain,
wrapped up in his own best coat and sound asleep.

Some one has said that the beauty of a town is like the character of a man — only
great in so far as its worst phase need not be hidden and its best is not paraded
for a boast.

In this respect New London on the bleak North Atlantic coast, may be more
beautiful than Constantinople and its sapphire sea.

\chapter{The Fourth of July at Robert College.}

Glad sounds, like the college boys’ rejoicings at New London, are never heard
in Constantinople, and an exhibition of gentle devotion to a dog, such as the
freshman captain showed there, would be impossible among the Turks; but there
are other sounds, which are never absent for a whole day in Constantinople; the
shrill note of the “muezzin,” or priest in the Mussulmans’ call to prayers, and the
dull monotonous hum‐hum of schoolboys at their lessons. This dreary monotone
is the same in all Oriental schools, but there is one variation in Turkey not heard
elsewhere; girls’ voices mingle here with the boys.

The first schoolday, usually the sixth birthday, is a great event in a Turkish
child’s life. If a boy, he has a new suit of clothes; ugly little trousers, a frock coat,
boots and fez, in exact imitation of his father’s, and if he is the son of a rich man
he has his first pony. It is the same with the girls, who dress on this first schoolday
like Turkish grown ladies, or in some special costume, “{\it per fantasie}” as they say,
and all have new books, and gifts of toys, and beads and money, from their friends.
The brothers and sisters, playmates and cousins, escort the little new students to
school, and the older pupils, who are always told of their coming, go out to meet
them half‐way, all joining in a merry procession, the new pupil often riding the
new pony which is gaily decked out with bells, ribbons and flowers.

But all pleasure and brightness ends at the schoolroom door. This room is
invariably plain and square and is the same for all classes; it shows no pictures,
maps, book‐shelves or globes.

It is furnished on three sides with long divans or benches, more or less hard,
where the pupils sit cross‐legged and often crowded closely together The master
sits alone on a divan at the opposite side of the room behind a small low table,
a book in his hand for reference, and a bunch of twigs, tied together, on the
table, the use of which is not ornamental — this being the much esteemed “rod” of
the Orientals. The dismal hum‐hum begins at the moment school opens in the
morning, and sometimes louder, sometimes fainter, never ceases, save for one
short interval, until its close.

Turkish boys and girls are of the race which has given the alphabet and the
sciences of numbers, navigation and astronomy to the world; but they study only
one book now and learn only one science. They study the Koran, from which
they learn to read, and the science of Mohammed’s religion, as soon as they can
commit sentences to memory, either by having it read to them, or by reading it to
themselves.

They study aloud as hard as ever they can, each beginning with a different
sentence, rocking to and fro, “weaving trouble,” meantime. If they falter in their
shrill repetitions the master’s duty is first to admonish, and if this is unheeded to
spare not the rod. There is a lull when the “muezzin’s” call is beard at noon from
the mosque minaret near by, and then the master and pupils, with faces turned
toward Mecca, drop to their knees and say a prayer.

When the priest’s call ceases and the prayers are over, the voice of the artful
candy‐man is often opportunely heard near the school, for candy is peddled about
on trays there and not sold at shops as with us. The new scholar is permitted
to “treat all round” on the first day and there are no better sweets than “Turkish
delights” — pasty, creamy, crackly things made up from rose‐leaves, violets and
poppies, nuts, dates, grapes and pomegranates, delicately mixed with honey,
sugar, syrup and spice. Pure cold water after sweets is known by all Turks, young
and old, to be the most delicious of luxuries, and this the school‐children often
enjoy, for the waterman is cunning enough to follow closely in the wake of the
candy‐vendor anxious to lighten his burden and draw a profit, as well as spring
water, from the tanned skin of a pig, which he carries strapped to his shoulders
like a bagpipe — the Turkish water bucket.

There are no other incidents to break the monotonous school hours, no recess recreations, no games, no fun, no holidays. The Koran chorus ends with
the “muezzin’s” cry at four o’clock when more prayers are said with hands outstretched toward Mecca, and school is over after courteous adieux to the master
and one another.

At six years of age, Turkish children, boys and girls, are expected to be
accomplished in all that belongs to the utmost courtesy of polite Mohammedan
manners. At the age of ten, they are supposed to have committed the Koran to
memory, which is about as easy a task as it would be to learn Bancroft’s history
by heart, and having acquired this one problematic accomplishment the girls
stop going to school altogether, and the boys begin learning a little arithmetic
and to write in Turkish and Arabic. How they manage to write aloud, I do not
know, but I have no doubt they invent a way to be noisy, even while tracing the
graceful Arabic characters, still sitting cross‐legged on the divans and holding
the copy‐books on their knees.

There are Military, Naval, and Medical schools for a few older students belonging to the wealthy classes, who have had French or English tutors at home, and
who are destined for the Government service; but apart from these, all the learning
of the Turkish youth is attained by means of this limited primary instruction.

The dearth of schools in Constantinople for many years made it very dreary
for the families of such American missionaries, merchants and sea‐captains as
were forced to pass their lives there, for their children had to be sent home or else
to grow up with scanty education.

Years before we went to Constantinople, an American philanthropist, Mr.~Christopher
R. Robert of New York, conceived the idea of founding on the shores
of the Bosporus an American school, or college, for boys, where they could study
and be taught exactly as at home, to meet the serious need of his countrymen
living there. As Secretary of State Mr.~Seward had helped to manage the very
delicate transaction on the American side, of securing a site for this college, or
rather, of inducing the Sultan to grant the firman necessary to guarantee a good
title to Mr.~Robert and the college.

Dr.~Cyrus Hamlin, the right man to carry out this generous and patriotic
plan, had lived in Turkey for forty years as missionary, and was; early chosen
president. Under his direction and personal superintendence a fine structure, rose
on the highest and most prominent point on the Bosporus, which is known far
and near throughout the East as the American “Robert College,” and its privileges
have proved as welcome to the aspiring Oriental youth as to the banished young
Americans.

When we arrived at Constantinople our great national holiday was approaching, and Doctor Hamlin invited us to Robert College to keep the feast.

My sister and I had never passed “the Fourth” away from our native land
before, and we fancied no very lively spirit of independence could show itself
in this stronghold of despotism; but nevertheless, when driving through the
picturesque and crowded thoroughfares, in the morning, our unsophisticated
surprise was hardly concealed as we discovered that the pervading atmosphere
{\it anywhere} could be wholly destitute of that “element of freedom” which the very
sun’s rays seem to blaze anew in America, on the Fourth of July, that the day
should really be, even in Turkey, of no more importance than the third, which had
also shown a painful lack of that spasmodic air of anticipation which it always
carries, at home. No, the Fourth of July in Constantinople was glorious, as a
day, but not one bit more glorious because it was “Independence Day” and the
anniversary of one of the grandest achievements in the history of man.

After a drive along the shores by the waters edge, and over the undulating
slopes, we at last spied the Stars and Stripes waving from the cupola of a tall,
square building, showing a perfect and simple model of American academical
architecture, which we knew must be our stopping‐place — and what a site for a
school! What a spot for object‐lessons, in mythology, history, geography!

When a student at Robert “faces the rising sun” to “bound Turkey,” his vision
is arrested long before it reaches the horizon; across the ocean stream he can look
into the houses and gardens of people in Asia, his own feet meanwhile being
planted on the soil of Europe. On his right hand, rolls the Sea of Marmora which
sweeps away to the shores of Asia Minor and whose thousand islands glittering
in the sun seem laughing defiance at the snowy head of old Mount Olympus,
denting the southeastern sky. On the student’s left, and equally in full view, rolls
the Black Sea, the Euxine of the ancients and the mouth of devouring Russia.

Across that narrow bend in the Bosporus, Io passed from Asia to Europe on
the back of the swimming white bull; and lower down, just far enough from the
college porch for a good sledging coast, is the rocky nook into which Jason’s
sailors forced the Argos and rested in their quest of the Golden Fleece. Within a
stone’s throw from the college playgrounds the European end of that bridge of
boats was planted which King Darius caused a native of Samos to build and which
his great Persian army crossed and recrossed in triumph years before the battle
of Marathon. On that distant point Byzas founded the city of Byzantium which
was in turn stormed by Philip of Macedon, and on the same point a thousand
years later, the Roman Emperor Constantine re‐built the city and re‐named it
Constantinople, and New Rome, the Rome of the Eastern empire, which was
afterwards effaced in the torrent of northern barbarism which swept away the
empires both of old Rome and of new. Centuries later at this famous point on
the right, the Saracens gave battle to the invading and irresistible Crusaders; and
later, on the left, the armies of the Prophet grown invincible, won it back to the
Moslem leaders and their followers, who have held it ever since, notwithstanding
the never‐sleeping eyes of two powerful nations are vigilantly fastened on it;
Great Britain’s with jealous care and Russia’s with envious longing.

I do not believe that the words and sentences in Murray’s grammar and
Webster’s spelling‐book combined would tell all that one may see and think of
from the porch of Robert College, for the world in its marvelous past is there
spread out before one so clearly, that its mysterious future seems almost to be
made plain.

But strange to say we thought of none of these things when we came to the
gates and passed into the grounds of the college, though we were met there by an
army bearing fadeless emblems of victory, and the signs of conquering courage
and aspiration. The emblems were the Stars and Stripes and the signs of conquest
and desire were in the joyous faces of the peaceful and order‐loving army of
faculty and students of the American college, headed by their loved and honored
president. Dr.~Hamlin.

There were perhaps one hundred and fifty students ranging between the ages
of twelve and twenty years, two thirds of them Americans, and all vigorous,
restless fellows, so eager with the celebrating motives of the Fourth of July that
we seemed to be at home again, and driving off to “independence” following the
inspiriting measure of
\begin{quote}
“‘Columbia, the gem of the ocean,’ \\
The home of the brave and the free, \\
The shrine of each patriot’s devotion, \\
A world offers homage to thee. \\
‘Thy mandates make heroes assemble \\
When Liberty’s form stands in view; \\
Thy banners make tyranny tremble \\
When borne by the red, white and blue.”
\end{quote}

Inside the college walls a scene met our gaze, in the great cheerful corridors
and library, even more characteristic of our national ways.

All the Americans of that part of the world seemed to have come together, all
the natives and foreigners who have secured American protection, all the fathers
and mothers and friends of the Robert College students and of the students from
the corresponding “Home School” for girls at Scutari on the Asiatic side of the
Bosporus.

Yes, truly, though so far from home we were come to Independence. The boy
students m dress‐parade white suits and straw hats, the girls in white gowns,
flowers and pretty sashes, left us in no manner of doubt of it, all was so bright,
so gay, with them. How lovely they all looked — and those American girls — how
generous and sincere in greeting their travel‐worn and somewhat homesick
compatriots. We had journeyed a whole year without seeing so many of our own
countrywomen and were much later from America than some of them; many had
home‐news to tell us of mutual friends, while others had never been in the United
States at all, and were now delighted to hear how nearly this glad festival was
like a Fourth of July school‐celebration at home.

When all these young people came up to welcome Mr.~Seward and his friends,
it will surprise no one to read that many of “us girls” embraced and kissed one
another, on the spot, and that the visitors’ arms ached for days from the sincerity
of the greetings which the Robert College boys expressed through the muscles
of their bard, warm hands, hands which left in ours the odor oi mingled pine
wood, sulphur and saltpetre, the very Yankee Doodle essence of the Fourth of
July dear to the senses of every American, true‐born, in fact or in sentiment; for
Dr.~Hamlin whose labors and devotion had brought the College to completion,
was quite equal to this occasion, and not even fire‐crackers and fizzing things,
“torpedoes” and “big guns,” were lacking to make the day more vividly “the day
we celebrate,” in our own peculiar and national fashion.

The assembled boys looked not unlike a company of young Americans as one
might see them coming together in any large American town, with about the same
proportion of English, German and Irish faces among them, all animated by one
sentiment and stirred by a common purpose. But here the variety was increased
by the addition of long‐eyed Greek, dashing Albanian, wistful Armenian faces,
and the inevitably‐heavy Turkish type.

A group of straight‐browed, square‐faced, strong‐shouldered fellows, was
most attractive, by reason of a certain dignity in speaking and freedom of motion
among them, such as mountaineers might have. These sturdy youths, who soon
engaged in a pitched game of base‐ball with some American boys, were natives
of the Balkan ranges and rugged plains of Bulgaria, a then poor and unnoticed
province of Turkey, but whose people have always shown an unquenched thirst
for enlightenment and education. The leader among the Bulgarian students named
Storloff, a serious lad but a first‐class pitcher, seemed all that the others claimed
him to be — “a splendid fellow” — and as proud of his friends Camburoff, Tomoff,
Panerfètoff, the devoted brothers Slovifèkoff and the rest, as they were fond of
him and of each other. These boys became more interesting even as we knew
them better.

After the reception, “the spread” was a new page in the catalogue of Independence wonders for the young Orientals to ponder over, for not only were
American delicacies served to all, but even the more specially home‐like substantial mere forthcoming. The banquet‐room was festooned with the American Stars
and Stripes, and canopied with the Star and Crescent of the Turkish banners,
while wreaths and bouquets were seen everywhere — the work doubtless of the
Robert boys, aided and directed by their sister students from Scutari. The bountiful tables were supplied at regular intervals with pots of baked beans, roasted
turkeys and boiled hams, while rolls of American gingerbread, pumpkin pies
and doughnuts were introduced and devoured everywhere with true national
disdain of foreign example or intervention. Nothing indeed was wanting to make
the illusion complete; we could easily fancy ourselves celebrating the Fourth in
Lafayette Square at Washington, or on Boston Common.

A young American was appointed to read the Declaration of Independence,
and although there was no “oration,” there was much speaking, for it was a great
day for the two old friends Dr.~Hamlin and Mr, Seward, who now witnessed a
grateful consummation of their many hopes and labors, and they did not fail to
impress on their young hearers that the special object and aim of the college was
the conquest of learning, a motive so much higher and nobler than the purposes
which had in times past been concentrated there.

The boys and girls listened attentively, the Americans with natural pride, and
the Orientals with a surprise something like consternation, for it was to many of
them the first public exercise of free speech they had ever heard. Mr.~Seward told
them some things about America which we all know but which these far‐away
students were glad to learn, and then he spoke to the Orientals of their different
countries, and of the necessity for the same neighborly good‐will among nations
as among men. “It used to be thought,” said he, “that all great ideas must go from
the East westward, but men have already begun to see good coming, from the
West to the East; Robert College is an illustration of this great truth,” and the boys
cheered and applauded him royally.

Mr.~Seward continued: “Our national anniversary assumes a different character of celebration here from that to which we are accustomed at home, where we
so universally adhere to John Adams’ injunction in celebrating it with ‘guns, bells,
bonfires and illuminations.’ But celebration of it everywhere, at home and abroad,
requires gratitude to God, reflection upon our duties, and a study of our grave
responsibilities as citizens. It is not enough that the corporate existence of our
country is maintained, if its national spirit is not also preserved and developed;
for countries, like men, need {\it Sana mens in corpore sano}’;” here the Robert College
boys looked significantly toward the Scutari Seminary girls, who we had the
satisfaction of knowing later understood the Latin quotation quite as well as their
clever rivals.

There were other speeches in the Greek, Armenian, Syrian and Turkish
tongues, and the enthusiasm of the boys who belonged to those races who are
most susceptible to the influence of eloquence was only restrained from bursting bounds by the persuasive authority of Dr.~Hamlin and his corps of eleven
professors. After some appropriate declamations, all united in the loyal old chorus,
\begin{quote}
“My country, ’tis of thee,”
\end{quote}
and the ever soul‐stirring anthem,
\begin{quote}
“Glory, glory, hallelujah!”
\end{quote}
The immortal words,
\begin{quote}
“In the beauty of the lilies, Christ was born across the sea, \\
With a glory in his bosom that transfigures you and me, \\
As he died to make men holy, \\
Let us die to make men free,”
\end{quote}
sung far away from our beloved country, in the capital of that unchristian
people who now rule the land of our Saviour’s birth, acquired a pathos and power
never revealed to us before.

A series of games and amusements, in the halls, and on the lawns and playgrounds, followed.

Mr.~Seward was watching a base‐ball game with intense interest, when a
rather shy — but manly young fellow, one of the Bulgarians, came up to him and
said in a straightforward way that he had read some of Mr.~Seward’s speeches and
wished to ask him a few questions. Now of all things delightful to Mr.~Seward,
girls and boys who asked questions were the most delightful, and it was his
greatest of pleasures to answer them. “Go on, my boy, and ask your question,”
said he, “I’ll answer it if I can.”

“Well, sir,” said the blue‐eyed Bulgarian, “I wish to know if, in your opinion,
the law which you call higher authority than the Constitution of the United States
is still in existence, or if the ‘Irrepressible conflict’ only meant the conflict between
slavery and freedom in America?”

Mr.~Seward had been asked this question many times before, but it was always
a satisfaction to discourse on his favorite theme and be soon had a knot of fellows
round him eagerly listening while he explained in his own earnest way that the
United States Constitution, or any other work of man, is only strong and lasting
in so far as it is guided by a sense of that undying, higher law, and abides by all
other laws which govern man, society, nature and life, and that true manliness
consists in making war against and resisting any principles which seek to dodge
or subvert those laws.

In such talk as this Mr.~Seward and the young Bulgarian missed count of the
ball game; but I do not think that either regarded the time lost, and in the light of
much that has happened since, and in which some of those very boys have been
concerned, their talk with Mr.~Seward has something of prophetic interest.

The day had grown hazy and cool before, the festivities ended, we thought of
returning to the capital, and the students re‐formed in procession with music in
the air and banners flying, to escort us on our way, Willie Hamlin five years old,
bringing up the rear, carrying his “own dear flag,” as he called it, in both hands,
and waving it bravely.

The college, standing on the brow of a high hill, is surrounded by the castellated
walls and towers of the ancient Saracenic stronghold and prison, Roumeli Hissan.

When we reached the outer wall and were passing one of the great towers
of the fortress, the procession deflected, marched round and formed a hollow
square with one opening toward Pera for the carriage to pass through. There was
nothing now to prevent the students from letting off the superfluous patriotic
fervor which had been repressed by the discipline of etiquette at the banquet,
and rounds and rounds of cheers rent the air and echoed along the walls which
had inclosed the dungeons of refractory Janizaries of old, resounding now with a
legend which had never saluted their ramparts before.

There were cheers for Dr.~Hamlin, and Mr.~Seward, for “Old Robert” and the
American flag; but when it came to the cheer which is always the loudest of true
American huzzas, “three cheers for the ladies,” it was no wonder that our Turkish
horses, unused to such sounds, took their bits in their teeth, and ran away.

But “who is afraid on the Fourth of July”? Our pride and patriotism stood the
test, and before our young hosts’ voices had died away the horses were under
control and soon after we arrived safely at the hotel in Constantinople, where
our Democratic friends the dogs howled a kindly welcome.

The Fourth of July celebration has become with us at home so much a commemoration of what is past, that we ignore its present significance, and are too
apt to forget that the central idea in our high national festival is one which must
find a deep response of gratitude, wherever the love of liberty has become a living
consciousness, as it did that day among the Bulgarian students at Robert College.

Bulgaria, whose history in the distant past was one of heroism and glory, but
who had been for centuries under the dominion of Turkey, her people bound in
slavery to repugnant ideas and a cruel and tyrannical foreign government, was
suddenly released from this depressing condition and brought into line with the
free and civilized nations of the earth ten years ago. Throughout the long years of
their servitude and depression, however, the Bulgarians miraculously preserved
their inherent standard of bravery, a pure and lofty love of liberty, and never lost
faith in their own high future destiny.

At the close of the great war between Russia and Turkey in 1877–78 Russia’s
success released Bulgaria from Turkish rule, but the Balkan region under the
decision of the great treaty‐making powers of Europe was divided into two states,
Bulgaria and Senda, and the region south of the Balkans, called Roumelia, was
given to Servia. The Roumelians who were not in sympathy with Servia declared
their natural wish to belong to Bulgaria, and this claim of brotherhood being
quickly responded to by the people north of the mountains, they were all soon
joined in one feeling against Servia. The Bulgarians thus united were not however
strong enough to elect one of themselves for a leader, but with the consent of the
treaty‐powers they chose the gallant young Prince Alexander of Battenberg for
their ruler, and under his command gave battle to and defeated the Servians, a
much stronger nation and backed by Russia.

In this way the brave Bulgarians became a nation by themselves and possessors
of their own hereditary country.

The whole world looked on in amazement as this hitherto obscure and oppressed people manifested not only a rare character for coolness and self‐restraint,
but a marked capacity for independence and self‐government. Their sovereign
prince, Alexander, is first cousin to the Czar of Russia, but owing to a grudge
between them dating back it is said to some boyish quarrel, they are not quite
good friends. and the Czar was not pleased when the hitherto helpless Bulgarians,
now led by his handsome kinsman, became victorious masters of those rocky
heights of the Balkan range which form an impassable barricade across the direct
path from Russia into Europe, and there has been much confusion and ill‐feeling
in Bulgaria in consequence. But the Bulgarians have not only held their own, but
have advanced and now command high respect and confidence everywhere.

It is a pleasing sequel to our Fourth of July celebration at Robert College to
know that among those boys who feasted with us, played base‐ball, and cheered
so heartily, the Bulgarians whom we noticed afterwards became prominent men
in their own country. First of all, Storloff has been Chief Justice of the state, and
all the others distinguished soldiers or leaders in some department of their native
government

They all as boys studied the Constitution of the United States at Robert under
Dr.~Hamlin, and when Bulgaria at last became strong enough to demand a
constitution of her own, no less than twelve Robert College boys, now grown
to manhood, were elected members of the committee which framed that new
constitution. It was a matter of great astonishment to the European world to
see that the new Bulgarian constitution was formed as nearly as possible on the
model of the United States instrument. The patient Bulgarians long hoped and
prayed for the chance of liberal learning for their youth and Robert College would
seem to have been sent at the most opportune period.

This American academy stands to‐day on the heights of the distant Bosporus,
crowded with students, and protected by the grand and solemn principles which
it represents, carrying back to the old Eastern world the message of equality in
the rights of man from the far new West, and speaking of a love for freedom,
which American schools teach the way to obtain and enjoy.

The place given to dumb and helpless creatures proves the humanity of a
people, and the schools of a nation as truly indicate its standard of intelligence.

Perhaps the day will come when the careless, pleasure‐loving inhabitants of
Constantinople will accept Robert College and the “Home School” as models for
a new “school system” of their own, and when Turkey shall have advanced to a
comprehension of Christian methods and practices, the hum‐humming schooldays
will be over for the Constantinople boys and the citizens of the “dogocracy” will
cease to excite the pity of travelers who will then unite only in admiration for
their pluck and loyalty.

\chapter{In the Meadows at Trianon.}

The France of history, poetry and song, so abounds in romance and chivalry, so
sparkles with gay wit and bright humor among the more somber nations, that
the fond and glowing epithets Frenchmen apply to the land of their birth, never
seem far‐fetched or extravagant. “{\it La belle France},” “{\it la douce France},” “{\it France je
t’adore},” are at all times but natural and spontaneous expressions of the ardent
pride and admiration they feel for their dear and smiling country.

When we read Mrs.~Frémont’s recollections of the St.~Louis of her childhood
and enjoy in fancy its brightness and gayety with her, it is difficult to realize that
the French “ville” of her vivid description was a frontier town on the Missouri
River; but recollecting how much of the territory which now constitutes the
United States once belonged to France, we may wonder that so little remains in
our country to‐day of that French spirit which animated so large a section of the
early colonies.

The natural attraction France has for us, however, is plainly seen in the way
American pleasure‐seekers resort to Paris, the center of French ideas; and the
interest we feel in the ways of the French people shows itself by the aptness with
which we adopt French modes, and the appreciation we accord the grace and
elegance of true French manners.

Paris was only another name for all that was most fashionable and brilliant
in the world when my sister and I were children. Louis Napoleon, an exile after
years of intrigue and deferred hopes, had become Emperor of the French, and
was at the height of his power.

The military prestige of his name had diffused new life into the martial spirit of
the nation, and his armies in the pride of superior drill and glittering equipments
believed themselves invincible.

In the great city of Paris, boulevards were stretched out, monuments erected
and bridges built, until Napoleon’s imperial capital was beyond compare the most
splendid in Europe.

The new Emperor opened the royal palaces, added to the already magnificent
art‐collections, and furthermore drew round him a gay and brilliant court, amid
which the beautiful Empress Eugénie was the brightest star of the enchanted
galaxy.

Pictures of their son, the pretty Prince Imperial, in military uniform from his
babyhood, made his curly hair and sweet eyes familiar everywhere, and gave a
tender interest to the wide‐spread stories of the bravery, gentleness and talent
which proved him truly a princely boy.

Friends made by the Emperor in the United States in his days of exile were not
forgotten when he came to the throne; and as the governments of France and the
United States were supposed to stand in peculiarly close and friendly relations,
many Americans who went to Paris found welcome at the Palace of the Tuileries.

It came about in this way that young people at home heard a great deal about
the gay doings of the Court of Napoleon and Eugénie. Accounts of the balls at
the Tuileries, fêtes at Compiègne, and poetry, music, dancing and art everywhere
in France, bounded like dazzling romances, and it was not the unreasonable hope
of many young Americans to sometime witness, or participate in, this gayety and
delight.

Nor was this brilliant court the only attraction for travelers there. No American, however serious, going from his home to study the nations, would think of
leaving France out of his tour, and it was of course a part of our plan on leaving
home to visit the beautiful country; but when at last we reached Paris it was no
longer a gay and brilliant capital, but a city scared and desolated by a great and
destructive war.

When we left home war between France and Germany was but just declared:
Napoleon was telegraphing news of his first success in arms to the Empress‐regent,
and the mother’s heart had thrilled to the word that her gentle son, the young
Louis, who rode to battle by his father’s side, had gallantly received his “baptism
of fire.”

But even before we set sail from San Francisco, this had all changed, and
news came only of French reverses, so that on our arrival in Japan we were not
unprepared for the tidings which met us there, that the Emperor of the French had
been taken prisoner by the Germans and the Empress and Prince Imperial were
refugees in England. From time to time as we neared Europe accounts reached us
of more defeats to the French arms until at last we heard of the final catastrophe;
the Germans had marched on Versailles and France was forced to a capitulation.

Then came a temporary government in France, threatened with destruction
by the wild fanaticism of anarchists until quiet was restored and peace declared
under a new régime, of which Mr.~Thiers was chosen the provisional president.

We arrived in Paris just at the time when the destinies of France were at last
in the hands of the people, represented by a popular Assembly or Congress which
met to deliberate at Versailles twelve miles distant from Paris, whose ferocious
mobs the Government had every reason to dread and fear.

The frontier of France once passed, the sad condition of the country was but
too manifest. We saw everywhere the gloom and sorrow of a mourning nation.
The men were downcast and dejected, the women, with tear‐stained faces, were
clad in the black robes of grief.

In Paris little remained but mocking traces of the once splendid capital. The
great public buildings surrounding the Place de la Concorde were a smoking
mass of ruins, the streets, often barricaded by their own paving stones torn up
in the fury of the commune massacre, were deserted by all save proud Prussian
soldiers and such few citizens, wearing crêpe badges, as the actual necessities of
life forced abroad to meet the bitter scowls of their victors with looks more black
and haughty.

Bridges were destroyed, monuments overturned, parks empty, their beautiful
trees lying dead on the ground. The Opera House was closed, the Tuileries a
blackened pile, Compiègne completely demolished, ravaged and burned by the
Communists, the imperial family in exile. The very churches we found patroled
by strong military guards.

Does History present a more dismal picture?

Could anything be more disappointing to two American girl travelers, than
such a realization of Paris, the city of their dreams?

There were not even shops to go to, and the world‐famed, tempting windows
of the Palais Royal and the Rue Rivoli were curtained or empty.

Eugenia, the Swiss maid, who had made such a “stunning” appearance in the
elephant parade in India, had been eagerly looking forward to Paris as the goal
where she might replenish our travel‐worn wardrobes, and she now had some
difficulty in finding even a dressmaker; but when she was at last successful, the
French {\it modiste} proved worthy the reputation of her class.

Before many hours our rooms were strewn with the delicate and fascinating
“{\it chiffons}” which a Paris dressmaker can apparently weave out of blades of grass
and sunbeams, as it is said a French cook can make an excellent soup with a few
nails and a piece of shoe leather. Here were silks and ribbons, laces and gloves (of
the “latest fashion” and “{\it so very cheap}”), enough to stock a village at home, which
all through the terrible days of the Commune had been hidden and saved by the
clever and deft little woman who brought them “just for the ladies to look at” as
she most politely assured us. Eugenia was in raptures. We were also glad to make
a choice among these dainty fabrics and especially pleased to have new gowns
for an entertainment to which we were bidden, for French people will always
make occasions for hospitality, and we had been asked to dine with President
Thiers at Versailles.

The time appointed happened to fall on the very day which had been set apart
for a general debate in the Chamber of Deputies, in which all the opposing parties
were to meet and decide the great question whether there should be a permanent
republican government in France, a President, or no.

It was a moment deeply interesting to Mr.~Seward, and he directed an early
start for our drive to Versailles, where we were to pass the day, before going to
the dinner in the evening.

Eugenia went to Versailles by railway train, equipped with two large trunks
which she had carefully packed, evidently looking forward with great pride to
the exercise of her special talents. Eugenia’s ideal young lady was one who could
only be content “{\it en grande tenue},” that is, with banged and frizzled hair, feet in
high‐heeled, pointed‐toed shoes, waist in a vise, and all surrounded by a mass of
perfumed trains, flounces and furbelows.

The only occupation which she deemed worthy this gorgeous but uncomfortable creature besides “dressing in the fashion,” was, like the fop’s opinion in
the play that “a lady” might also “play the piano.” Poor Eugenia was naturally
amazed by the ideas and practices of her country‐bred American mistresses who,
while traveling, seldom thought any costume at all approaching her standard of
elegance, either suitable or becoming.

Arrived at Versailles, we went with Mr.~Seward to the small theater “Pas Perdu”
attached to the ancient royal palace where the National Assembly was held. The
great debate had already begun between Orleanists, Legitimists, Imperialists and
Republicans, the last of whom, I need not say, had our profoundest sympathy. M.
Grévy was presiding with great firmness and dignity, but the Chamber of Deputies
otherwise presented a scene of wildest uproar and confusion, notwithstanding
the presence of many of the most distinguished men of France — Monsieur de
Rémusat, a grandson of La Fayette, the Duke de Broglie, Monsieur Gambetta, and
many others whose names are familiar everywhere.

Seated in one of the close theater boxes, my sister and I found the air hot and
oppressive, the noise and confusion intolerable, when Mr.~Seward was greeted
by an old friend whom he was rejoiced to see and than whom no one in France
could have been more welcome to us.

This gentleman was a member of the Chamber of Deputies, and he came into
the box holding his hat, gloves and stick in one hand and bowing low, with that air
of self‐deprecating dignity which marks the peculiar grace of a French gentleman.
He was none other than Edouard Réné Lefebre Laboulaye, the poet, statesman and
philosopher, who is known to children as one of the dearest of “old story tellers.”
Mr.~Laboulaye had a rather grave face, but it beamed nevertheless with the light
of the calm mind and genial spirit within which likewise animated his gentle
voice and pervaded his graceful, expressive gestures. He had been Mr.~Seward’s
constant correspondent throughout the terrible years of our Civil War, and the
meeting between the friends at this critical moment in the affairs of France was
full of deep and significant interest.

My sister and I had been taught to admire the noble character of Mr.~Laboulaye
and to feel peculiar gratitude for his wise and timely views of our national questions. We had read and loved so many of his delightful fairy tales that he seemed
like a dear and familiar friend. It did not surprise us therefore that one who knew
fairies and their ways so well should read and interpret the thoughts of his own
friends when he said, “You are very tired of this, Mesdemoiselles, and long to be
away from it,” meaning of course the uproar of the Assembly Chamber.

In vain we protested that it was deeply interesting as an historical event and
that we were glad to be there — he saw through the veil of politeness which we
tried to draw over our ennui.

“Come, come,” he said cheerfully, “let me take you to something very different;”
and following him we passed from the stuffy galleries and entered the grand palace
of the kings of France.

This palace at Versailles is, beyond all others in Europe, stately and magnificent,
with its “crystal hall,” ostentatiously dedicated to “All the Glories of France.” Silent
now, those past glories were obscured by the cloud of recent memories, the very
saddest and most humiliating for the French people to bear. Less than three
months before the King of Prussia had made his bivouac in those resplendent
chambers, and the mosaic‐paved galleries had been turned into hospitals for his
victorious soldiers; there, too, he had assumed the crown and title of Emperor
of that Germany whose iron heel of power was now pressing so sorely on the
freshly‐wounded heart of “{\it la belle patrie}.”

Our sympathetic guide quickly felt that the cold grandeur of this royal palace
made sorrowful and gloomy by association of such scenes, was as oppressive to
his companions as the noisy Assembly Chamber. Indeed we told him that we
had often found the real world of our travels so sad that we had longed for the
Fairyland to which he seemed to possess the magic key.

“Fairyland is always nearer than you think,” said Mr.~Laboulaye. “Come, I
have a little time before the great vote is taken in the Chamber. We will go as
near to Fairyland as we can in Versailles.”

The palace is full of associations of Queen Marie Antoinette, and of her life
there, always brilliant and regal, but so formal and severe as to seem like prison
life to the joyous little Austrian princess.

To find, with Mr.~Laboulaye, a French Fairyland, we eagerly ran down the
broad marble stairs which the fascinating girl first trod, when she came, leaving
her mother and her happy home, at the age of fifteen, to be married to the Dauphin
of France, scarcely older than herself.

We drove quickly through the modem‐looking streets, and Mr.~Laboulaye
talked so charmingly we had forgotten to ask him where he was taking us, when
we suddenly found ourselves transported to the midst of a quiet, miniature forest,
the paved streets with their strange cries, and hubbub, the theater and palace, far
behind.

The deep‐shaded forest would have seemed primeval in its solitude, except
that the trees were of all species and climes, from the spruce of Norway to a
Javanese palm. Flower‐beds, fountains and lawns peeped through the shadows,
and cunningly‐contrived vistas showed us the roofs and gables of a tiny Swiss
village. In the midst of all arose a small, white structure, a very bijou of a royal
palace, fit residence for a fairy queen.

Mr.~Laboulaye was not far from the truth when he called this lovely place a
Fairyland, for it was “Le Petit Trianon,” once the playground and palace of the
boy and girl king and queen.

The Princess and the Dauphin had been married but a short time when upon
the death of the old king, Louis xv., the Dauphin ascended the throne. As soon as
Marie Antoinette became Queen she begged her youthful husband to give her a
little home of her own far away from the sumptuous palace and the stiff, stately
ways of court life, and the King gave her this exquisite little estate.

Their Majesties were still In their “teens” and here they contrived to live at
intervals a simple, rural life with all the glad freedom of the children they were at
heart. The King had a little mill, the Queen a vegetable garden and poultry yard.
The King delighted to dress like a miller and grind his own wheat, and the Queen
was proudest when he came to her {\it petits soupers} and praised her eggs, her cream
and strawberries.

Their pretty chateau stands now exactly as they left it. On entering the white
and gilded drawing‐room we found the walls still decorated with the paintings
which Marie Antoinette caused to be made in Vienna, portraits illustrating scenes
in her happy childhood, where she is seen at play or dancing merrily with her
brothers and sisters at the palace of Schönbrunn.

Here were the books she studied, the poems of her tutor Metastasio daintily
bound in vellum, still showing her favorite passages marked with her own hand,
here the little harpsichord (quite like that of our own Nelly Custis, at Mount
Vernon) which the winsome queen delighted to play on, and the music‐stand
still held a scroll from the hand of her beloved and great master, Glück. It was
impossible to realize that this lovely little chateau, this girlish fancy of a “play
home,” should have been uninhabited for nearly a hundred years!

The silence was lonesome and uncanny. We begged Mr.~Laboulaye to summon
Thumbkin, Yvon and Finette, Zerbino and Aleli, Piff‐Paff or some of the other
fascinating people he knew to dispel the gloom, but he said “No, fairy folk shun
palaces and live in the open air.” He then led us away from the chateau, beyond
the miniature mill, the dairy, the lake, the rockwork, all perfect and dainty as if
just built, through the grove to the meadows where the sheep and cows of these
romantic young sovereigns used to graze and wander, and where the King and
Queen, with a chosen few of their favorite friends, and people of the court, in the
costumes of shepherds and shepherdesses, once picnicked and frolicked, fancying
it, for the hour, their real life, and not a dream.

The meadows were enlivened even now by the presence of a company of gay
young haymakers in white caps, blue dresses and wooden shoes, who were raking
up the clover into fragrant mounds.

We were soon seated on the grass in the shade of an overspreading oak,
leaning against one of those clover beds, taking tea and cakes and honey which
we surely thought some of Mr.~Laboulaye’s fairy folk had provided, for we found
the refreshment beside us, and most daintily served, by whose orders we had no
knowledge.

It was a beautiful, soft August day. The hay‐makers, boys and girls, were
drowsily humming the plaintive French martial melody, “{\it Partant pour la Syrie},”
swaying back and forth, in lines, to the rhythmic motion of their busy hand rakes.

Wholly pleased at last we begged Mr.~Laboulaye for just one story, to suit
our mood and the sweet surroundings, but he answered: “No, this quiet hour has
been like an oasis in the desert of white‐heated politics; now I must return, to
cast my vote for the Republic and President Thiers,” and he left us, saying that the
carriage would return as we desired to remain longer.

An interval of rest in this enchanted spot, between the morning’s feverish
scenes and the formal dinner which awaited our evening, was too tempting to
be refused, and so it came about on that pleasant midsummer’s day in sunny
France that my sister and I passed from Fairyland into dreamland where, our
heads resting on the soft haycock, we breathed in the fragrance of sweet clover,
and wandered so long that Eugenia came very near to losing the chance of attiring
“her young ladies,” and they the state dinner at the President’s.

We were recalled from this pleasant land, where our spirits had fled, by a
breathless servant who ventured to tell us that it was nearly dark and the carriage
had waited two hours.

Making our way to it as fast as possible, we were whirled post‐haste to the
hotel where we found Mr.~Seward too jubilant over the turn the debate had taken
to even wonder at our absence.

But had Mr.~Seward wondered at our long absence, the thought could never
have occurred to him while he watched the tumultuous scene in the Chamber
of Deputies, that his charges were unconsciously dreaming in the quiet Trianon
meadows.

The provisional government had passed away, a permanent one had been
established, with Louis Adolph Thiers for President, and the government then
formed is the French Republic of to‐day.

Thiers was the son of a locksmith, and, educated at the public school of his
native town, was a true man of the people. His talents and energy (for in his
youth he is described as being possessed “by a very demon of restlessness” and as
having “a tongue which wagged like a bell”) had made him a splendid career, and
great was our pleasure that we were to pass the evening in social converse with a
man so distinguished for genius and achievement on the day of his election to
the proudest office his country could bestow.

But our delicious and romantic escapade had left only a few moments for
the toilettes which had formed the schemes and meditations of the day for our
faithful Eugenia, and to her great mortification we dressed as quickly as possible,
refusing to submit to the curling‐tongs and the various perfumes and powders
with which she had armed herself, and reluctantly putting away her {\it batterie de
toilette}, Eugenia told us, with an air of injured dignity, that she had never expected
to be asked to “brush grass seed out of the hair of young ladies who were going
to dine with the ruler of France.”

The palace where President Thiers lived was built for the Prefect of the district
by Napoleon {\sc iii}., and was fine enough for any king; indeed, the King of Prussia
had made it his residence during his sojourn at Versailles.

This evening the corridors and salons were filled with the first men and
women of the nation, cardinals, generals and admirals, lords and ladies, the
Princess Metternich wearing her famous pearls, Prince De Chigi blazing with
decorations, all anxious to be first to do honor to the man who had done most to
save France in the dark hours of her defeat and humiliation.

President Thiers had a strong personality and most attractive charm of manner.
Though short of stature he had a military bearing accounted for perhaps by
soldierly tastes which he possessed, in common with most small men as it is said.

He was now seventy‐four years old, and his snow‐white hair, cut with regulation precision, was brushed straight off his fair, smooth brow. His dark eyes
sparkled with the fire of wit and vigor. He wore his dress coat buttoned close to
his chin, stepped quickly and spoke in a sharp, imperative tone which instantly
commanded attention.

At dinner he surprised and startled us not a little by turning suddenly and
saying, “I hope you slept well and enjoyed your dreams!” The idea that our secret
of the afternoon, which we supposed unnoticed by any one, was known to the
President himself, was embarrassing to say the least, but it was characteristic of
Mr.~Thiers to know, as if by magic, every detail in the ways, almost the thoughts,
of persons in whom he felt any interest. Regarding us as his guests for the day he
had been informed of all our movements and it was to him, instead of to some
unknown Fairy Prince, that we were indebted for the refreshing cup of tea of
which we had partaken so thankfully under the great oak in the meadows.

Mr.~Thiers went on to say that just at the hour when the vote was being taken
which made him President of the French Republic, he heard how his young guests
were occupying their time and hailed it as an augury of peace and good fortune
to his administration that after all the horrors of the past few months in France
two Republican American girls had been in that intense hour securely sleeping in
the clover meadows of “Le Petit Trianon.”

\chapter{A Visit to Kensington Palace.}

An American’s first impression of England should be taken at London, and to have
it clear, one should land there straightway. Travelers mostly sail for Liverpool or
Southhampton and have a five or six hours’ glimpse of the Angel’s land, from the
windows of a fast‐flying railway carriage, before they arrive at the great town.

Seen in this way England is not very unlike many parts of New England, New
Jersey, New York or Pennsylvania. The traveler arrives late, is hurried off in a
close cab, through crowded streets dim with fog, to a commonplace hotel. Weary
from a sea voyage he dines early and goes to bed, hardly realizing that he is far
from home. In the morning papers one reads the American news, telegraphed the
night before. In the corridors of his London hotel an American meets so many of
his countrymen that he easily fancies himself at the Fifth Avenue or the Revere
House. When at last one starts out to see the sights of London, it is with much
the same sort of feeling that he had in driving to Central Park, to see the Egyptian
Obelisk, or to the East River to see an English‐built yacht. In these ways the
impression is blurred and the traveler who has been in London for only a few
days, says he is “disappointed,” thereby astonishing all Englishmen as much as
we were offended and amused when it was reported of a conspicuous English
traveler in America, that he had kindly expressed himself as “not disappointed
with Niagara.”

I first saw London on a bright September morning, and had the good fortune
to see it from the river side. Mr.~Seward, my sister and myself had come by
steamer across the German Ocean from Hamburg, and fifty‐four miles up die
Thames; the broad river, teeming with all the shipping in the world, seemed like
a stately avenue approach to the great Babylon. We passed so many historical
points and places of interest, and became so imbued with national associations,
that it would have seemed not extraordinary to have anchored opposite the great,
gray, square Tower, to have been pulled across to the landing, marched over moat
and drawbridge and through the portcullis in charge of yeomen of the guard, so
vivid grew the impression that we were in {\it old} England.

Apart from such fantastic associations of past history and London Tower,
how grand and beautiful seemed the real London Town of today; how different
our feeling for it, our pride in it, from any sentiment that we could summon for
any of the other great towns we had visited in our journey around the world.
London for so many hundred years the great metropolis and center of our race,
standing to‐day by the Thames, vast, solemn and grand, shrouded in a nimbus of
that gray mist‐cloud which never leaves it, and holding, as no other center holds,
the living elements of past civilization, illustrating the best of the present, and
giving assurance of future Anglo‐Saxon power.

We landed at a great wharf, and were quickly in the midst of a crowd of
porters, passengers, and sailors. There was no time to ask ourselves why all sights
and sounds were so familiar, nor to wonder why all the men we saw seemed
like brothers. Though the broad Atlantic rolled between us and our native land,
we felt vaguely but truly in our hearts that we were at home. Driving along the
street we spied a bareheaded, long‐skirted “Blue‐coat boy,” whom we seemed to
recognize, a queer sign, so well known to us of the painted wooden figure of a
midshipman, that when a broad sailor rolled along, in wide trousers and a glazed
hat, we had the impulse to shout out a greeting to our old friend “Captain Cuttle,”
and to ask how “the little instrument maker was.” We soon came to understand
why an American must feel at home in London and in England. Not so much at
home as in his own house, nor even as in his father’s house, perhaps, but at home
in the fond, proud sense that one feels at home in one’s grandfather’s house.

For months we had been traveling, far away from home. We had sailed from
San Francisco, and had come by a westward course, through many lands of Asia
and Africa, which are the far East of European history and geography. We had
heard no familiar words, nor seen a familiar face, save for the British residents of
India, and the occasional English and American travelers whom we had chanced
to meet amid the millions of strange peoples whom we had seen on our way. But
at last our feet touched the soil of our mother country, and we were listening,
after many months of longing, to our own native tongue. No peculiarity of accent,
nor difference of cadence, struck our glad ears, for we were not critical. All the
people around us, men, women, and children, spoke “English,” and we understood
their hearts, pronounced with an h or without, and drank in the welcome sounds
as eagerly as that parched and tired boy in the poem drank deep draughts of pure
spring water from the familiar “old oaken bucket that hung in the well.”

Our first drive in London, from the wharf, up the Strand, past the Bank of
England, by Temple Bar, across Trafalgar Square, under the shadow, of Charing
Cross, through Pall Mall, to Fenton’s Hotel, in St.~James, was like a day dream,
in the study chair, over an English History lesson, or a loved volume of Dickens,
in the library at home. At Fenton’s, the rosy, fat landlady met us with a smiling
courtesy, and the landlord like a host.

It was the dullest month of the year for seeing people, and although to us
the streets seemed crowded, we were told that “everybody was out of town.”
Mayfair was desolate, doors barred, and shutters closed, for the Court and all
the fine people were away in the country, as Bostonians and New Yorkers are in
the mountains and at the seashore in August We had no visits to pay, no dinner
parties nor balls to divert us from our cherished purposed of seeing London.
London sights and London sounds, London parks and London bridges, from
Hampstead Heath to Camberwell, from Woolwich to Putney; the Crown Jewels
in the Tower, the Abbey and the tombs, all must be explored. It might not be the
“London season,” but the prospect for us was not dull. Ouf guide was Mr.~Seward’s
old friend. Sir Henry Holland, who was a celebrated traveler, a distinguished
physician and courtier. He married Sydney Smith’s daughter, who wrote the
Memoirs of her famous father. Lady Holland was not now living, but Sir Henry
Still occupied Sydney Smith’s house, which had always been their home. It was
an old‐fashioned house, with a cheerful breakfast room at the end of the hall, as
you entered from the street, and there smiled down from over the chimney‐piece,
a speaking portrait of the witty and reverend old master of the house.

This house was around the corner from Fenton’s, and here we often began the
days, breakfasting with Sir Henry, while he made our plans for us, knowing well
from his long acquaintance with American thoughts and ways, what would most
interest two young American women. Under his kind care we saw the British
Museum, the National Gallery, and all the well‐known sights. The Elgin Marbles
in the Museum interested us more than anything else there, for we were lately
from Greece, where we had seen the wonderful architrave of the Parthenon at
Athens, looking as if just stripped of these its own matchless ornaments, although
fifty years and more had passed since they were removed from there to England.
Among all the pictures the beautiful, strange Turner’s awed us as none others
could do, and we saw among the beloved Landseer’s the original of the “Shepherd’s
Last Mourner,” the old hall, the rough coffin, the dear colly, leaning his paw and
head against it, in broken‐hearted sorrow, our tears being an honest tribute to the
picture’s pathos, often roused before, even by the copy which hung in the little
ferry boat crossing from Garrison’s Landing to West Point, a British picture, from
which West Point boys have learned new depths of tenderness, and which has
been a part of the thoughts of some of the bravest young men of our country, too.
Would frontier warfare leave them, the gay cadets, so much of friendship at the
last, as this colly’s grief proved for the loss of his poor master?

Sir Henry took us also to see many rare private collections and libraries, to
sumptuous club houses and stately palaces, now all draped in linen dusters, and
coldly silent. To Apsley House, where the first Duke of Wellington lived for
many years, and where the successor to his title still lives. To Stafford House, the
splendid palace of the Duke of Sutherland, to Holland House, where Walpole and
other famous people have lived, and to many others. But the most interesting
palace to us was the old palace at Kensington, where Queen Victoria was born,
and where she passed her childhood and youth, with her mother, the Duchess
of Kent, and in whose halls she was still living, and studying in quiet seclusion,
when King William died at Windsor, and the grave Ministers of State came to hail
her “Queen,” and “she wept to wear a crown.”

Now my sister and I humored, if not cherished, the superficial American prejudice against all royalty, aristocracy and their ways, as the modem expression of
that arrogance and the other specially British attributes which our Revolutionary
ancestors, as we fully believed, fought against, and as far as our national character
was concerned, destroyed. Our own grandfather was wounded in the battle of
Lake Erie and when we were little things we had heard him peremptorily direct
that “God Save the Queen” be torn from our musical instruction books. This
stern command made a lasting impression on our minds, and in our inmost hearts
we felt rather proud of our scorn of state and royalty. But the story of the girl
Queen is always fascinating, even to republican girls, and we delighted to picture
the little princess with her dear half‐sister, her pony and her pet dogs, playing
in those shaded gardens, the very site, tradition tells, of the ancient capital of
the early English fairies, where little King Oberon, ages ago, had his palace and
held his elfin court. Or, we could fancy that “maiden, heir of Kings,” treading
the galleries of the old palace, smiled down upon by the quaint, early German
portraits which adorn the walls, and to fancy her modest, quiet life there, such a
contrast to all the care and grandeur which was to come to her so soon. It is a
somber old palace, built oi brown bricks, and was made originally for the country
house of a private gentleman, but King William bought it — the King William who
reigned with his wife Queen Mary — and ever since the early eighteenth century
it has been a royal palace. The gardens which were once extensive, have been
curtailed by the growth of the town about them, and now the place is rather a
contracted court, though still having some fine trees and box hedges.

Different members of the Royal family have lived in the old palace since the
Duchess of Kent, the Queen’s mother, left it after Victoria Regina ascended the
throne. Now we were told the Queen’s favorite aunt by marriage, the Duchess of
Inverness, was living there, and by that chance it came about that we were to have
some personal experience of the real ways of British Royalty and palaces. Mr.
Seward visited London in the spring of ’57, in the height of the London season;
he went to court and saw many great people and fine palaces, which seemed to
have left very little impression on his memory. But he never had forgotten the
Duchess of Inverness, for whom he had formed a strong friendship. She was
own great‐aunt to his friend Lady Napier, and it was by her and Lord Napier, the
latter who as British Minister at Washington had been so highly esteemed, and
Lady Napier so well beloved, that Mr.~Seward was presented to the Duchess in
1857, as one of their best American friends. All these pleasant recollections had
been renewed in Mr.~Seward’s recent visit to India, where we had passed most
delightful weeks with the Napiers at Government House, Madras, where Lord
Napier was now Governor. We had then promised Lady Napier that we would
not fail to visit her aunt.

So it happened when passing Kensington Palace one day, that Mr.~Seward
accidentally heard the Duchess was there, and drove in to inscribe his name on
her book, which is the manner of leaving a visiting card on royalty. The assassin’s
knife had disabled Mr.~Seward’s right arm, and it was often my privilege to write
his name for him. I now wrote it in the big register‐looking visitor’s book, and
my attendant and I passed sentinels on duty and officers in scarlet before we
came to the inclosure in a corridor, where a tall liveried custodian sat guarding
the precious record. I was greatly struck by the really royal care thus taken by
royalty, to preserve in a book the autographs of all the people who are so good,
they say, as to come to inquire for them. That is not quite the feeling with which
ordinary people regard their friends’ visiting cards. This little incident occurred
just at the end of our stay in London. Our trunks had already been forwarded to
the Cunard steamer at Liverpool, in which we were to sail the next day but one,
and we had no expectation of hearing more from the simple act of courtesy of
writing Mr.~Seward’s name in the visitor’s book at Kensington Palace. We were
greatly surprised, therefore, the next morning, when a footman appeared at our
door, in the royal livery, bearing a note, sealed with the royal coat of arms and
addressed to Mr.~Seward. It was a cordial and informal invitation to dinner, from
the Duchess of Inverness, for that very evening.

Now a royal invitation is regarded as binding on the recipient, as a royal
command would, be, and this one Mr.~Seward was very glad to accept. I wrote
his note of acceptance, and was about sealing it, when Sir Henry Holland came
in, the picture of amiability and kind interest. As I had never written to a royal
Duchess before, I opened the note to ask him if it were {\it en règle}, for it had seemed
to me a little odd to do as Mr.~Seward had told me and write “My dear Duchess,”
just as I would have written to any other lady, “My dear Mrs.~So and So.” Sir
Henry said it was correct and added, “Of course you are going?” We told him
we were not invited, and that it was lucky the Duchess did not know we were
there, as our evening dresses were gone in the trunks to Liverpool. We were
going that day to pay some visits with Sir Henry, and wore our prettiest walking
suits, Which we had brought from Paris. “Why not dine in those gowns?” said
Sir Henry. But telling him that was out of the question, we invented a pleasant
plan of our own for the evening. Sir Henry, after paying a part of the visits with
us, excused himself, and we went alone for a last drive in Hyde Park.

On returning to the hotel, we found to our surprise, another royal footman bearing two invitations, and an informal note, expressing very simply the
Duchess’s cordial desire to have us come with Mr.~Seward. This note was from
the Duchess’s grand‐niece, and she now wrote me that her aunt had learned from
Sir Henry Holland (who by the way, was the Duchess’s old friend and physician,
and had now taken all the trouble, without telling us, to drive out to Kensington
to explain this matter to her), of the embarrassment we might feel in coming in
our traveling gowns, but that she would regard it a special favor if we would
show ourselves to her in that very guise, as since she lived so quietly at home
herself, nothing pleased her more than to know how people looked and felt in
the busy world outside. Indeed the noble young lady convinced us that we would
be conferring a favor on her great‐aunt by coming just as we were, and this was
our second lesson in true royal ways. We accepted the invitation, of course, and
countermanded the little plans we had made for the evening. We could, even now,
have gone to a great shop and bought dresses and made toilets for the occasion,
but no such precipitate and extravagant action would have harmonized with the
dignity and good sense, the sincere and gentle courtesy, which spoke in every
line of the Duchess’s invitation. We dressed with what care we might and were
also particular to start fully in season to reach the Palace calmly and promptly
at eight o’clock, to show ourselves at least conversant with that etiquette which
makes punctuality the courtesy not only of kings.

We fancied the rows of tall, powdered footmen, in scarlet plush breeches and
white silk stockings, stuck out their chins a little more than was necessary, as
we passed up the long corridors where they stood on duty, and that the waiting
women in the chamber where our wraps were removed, gave subdued sniffs of
amazement when they found no trains to let down, and saw that we were going
to the drawing‐room in those simple, short dresses.

The drawing‐room seemed a very picture; a long, gracefully proportioned
room, with low ceilings, softly lighted by candles and furnished in all delicate,
harmonious shapes and tints, and pure old decorations, strictly speaking in Queen
Anne’s style, for Queen Anne had really lived there herself, and the drawing‐room
was full of royal, literary and artistic associations of the past. We could only think
of it, however, as the favorite drawing‐room of the wise and beautiful Duchess of
Kent, where she and her gentle daughter passed so many happy, tranquil hours.

Here the Queen and Prince Albert first came to know each other, perhaps
indeed first met, in that three weeks’ visit which the graceful cousin prince, with
his father and his brother, Prince Ernest, paid to the Duchess of Kent, when the
Princess Victoria was just seventeen. In this room too, I believe, all the chief
ministers, high officers, princes, dukes and peers Assembled, the day after King
William died, to kiss the young Queen’s hand and swear their fealty. Her two
uncles, the Duke of Cumberland and the Duke of Sussex were the first to kneel to
her, which confused her not a little, though she kissed them sweetly and dutifully
in return, and it was the widow of one of them, the Duke of Sussex, the best
beloved brother of the Queen’s father, with whom we had now come to dine. But
she was called Duchess of Inverness and this because of one of those ideas and
principles of royalty, which no American can like or understand. She was not
of royal blood, and was only a subject. Lady Cecilia Underwood, and therefore,
although legally married to the Duke of Sussex, by the ritual of the Church of
England, and acknowledged as his wife, the provisions of the Royal Marriage
Act made it impossible for her to ever receive the name and title by which her
husband was known, and another name and title, that of Duchess of “Inverness,”
was bestowed upon her. The Duke of Sussex married Lady Cecilia Underwood
while the Queen was still a little girl, and she was always regarded with great
affection by her niece and sovereign.

When we entered the drawing‐room, about twenty ladies and gentlemen were
standing at the other end of the room, talking in subdued voices. The young lady
who had written the pretty note to us, came forward at once. She proved to be a
charming person, and greeting us cordially, presented us to some of the ladies.
There were duchesses and marchionesses, and peeresses and earl’s daughters,
and an equal number of titled gentlemen, some of whom were admirals and
generals in uniform, whose breasts glittered with decorations. There was no lack
of brocaded trains, point lace, low corsages, and diamonds, so many indeed that
we soon forgot that we also were not {\it en grand tenue}.

Presently some one said, in a low tone, “the Duchess,” and in came a stately,
smiling, dear little old lady, in lavender satin, with much beautiful lace, and some
lovely jewels. Her grand‐niece went forward at once, and gave her arm, as though
she loved her, and took the Duchess to her own special seat, a chair made rather
higher than the rest.

The Duchess had not met Mr.~Seward for many eventful
years, and there was much for them to talk of together, but the elder ladies present
went first to speak to her, after which she called for Mr.~Seward, and then said,
“Now bring me those dear girls who have consented to give me the pleasure of
seeing them dressed for their great journey.” When we were presented to her,
she placed us on either side of her and nothing could be gentler and kinder than
her talk. She examined the material and fashion of our gowns with feminine
particularity, even requested us to “turn around,” and asked “if ladies always
traveled in such pretty things in America?” Now truth compels me to say that
those dresses were rather nicer than any we had ever worn traveling, but we did
not go into these small details with the Duchess, for she seemed well pleased to
find her nephew, Lord Napier’s, account confirmed, that “American ladies were
always becomingly as well as appropriately dressed,” which is not so invariable a
rule in England. It is pleasant now to recall those gowns, for they were very pretty,
and what the French call {\it la mode inédite}. They were of silver‐gray poplin, with
full {\it crêpe lisse} ruffles in neck and sleeves, and these put in with soft rose‐colored
silk facings, ending in pretty bows. Our boots and gloves had been made in Paris
to match, and we wore no ornaments, save bunches of fresh pink roses, fastened
in our belts. Our good French maid took special pains dressing our hair that
evening, and as my sister and I were attired exactly alike, and were of precisely
the same height, we had the feeling of supporting each other during this amusing
ordeal. The Duchess asked us many questions about our travels; and in return
told us some anecdotes which her “nephew Alfred,” the Duke of Edinburgh, had
related to her, of his experiences in the East, and also how the Prince of Wales
had told her of Mr.~Seward’s kindness in arranging a shooting excursion for him
“in the wilderness” (I think in Illinois), when as a lad he visited America. But
best of all was an animated little account of her own experiences, for she too,
she said, had also traveled and understood too well a traveler’s delights, dangers
and fatigues, for said the Duchess, with great spirit and satisfaction, “once, many
years ago, I crossed the Channel to France.”

When dinner was announced, the Duchess preceded everybody, with Mr.
Seward, and this was the only difference in etiquette from any other dinner party
I have ever seen. The table was circular and the service very handsome, all of
silver or gold, and the {\it épergne} in the center, was altogether of gold; some design
of the national crest, the lion, the unicorn and the crown. The conversation was
all in low tones, rather serious, and not general. The dinner was served in the
regulation English style, as it might be in old‐fashioned houses in New England
or Virginia to‐day; that is, the soup, fish, joints, the fowls, and the pudding were
put on the table before being removed and served, not {\it à la Russe}, as all grand
dinners on the Continent and in America are now more gracefully managed.

The Duchess talked earnestly with Mr.~Seward of the changes in the world,
and among their mutual friends, which had come with the years; but she soon got
back to the interesting topic of our long voyage. The Duchess seemed to think
that crossing the Atlantic in September must be full of peril. She was amiably
solicitous for our comfort, and asked Mr.~Seward many questions about our
preparations, which he, with manlike unconsciousness of feminine details, found
it difficult to answer. She asked, with keenest interest, if he had provided us with
plenty of rugs, hoods, tippets and muffs, and whether we had hot water bottles;
finally, when there was a lull in the hum of talk, all heard her ask in her clear,
thin voice, “Have they any down petticoats? They are the most comfortable of
all comforts.” The Duchess urged the necessity of these garments so strenuously,
saying that her friends assured her they had never known comfort until they
wore them, that Mr.~Seward had a confused sense of having been remiss toward
his wards, in not providing them. “They really \emph{must} have them,” persisted the
Duchess, “you must see to it,” and greatly perplexed, Mr.~Seward was constrained
to ask, in his measured tones, “My dears, have you any down petticoats to wear at
sea?” With great honesty and some reluctance, we answered “No.” Whereupon,
a detailed description of these light, warm articles became necessary, and the
subject was becoming tiresome, not to say embarrassing, when the gentle voice
of the niece was heard, saying, “Dear aunt, I will take the young ladies myself
to Marshall \& Sellgroves, and we will get them in the morning,” and so turned
attention away from us, and the swan’s‐down skirts.

The Duchess asked Mr.~Seward how old he was, knowing him quite well
enough for that friendly familiarity, and when he said “seventy‐one,” replied,
“Oh! I am one year better than you, I am just seventy.” The lovely niece smiled
pleasantly, and told me in an undertone, that this matter of age was an idiosyncrasy
of her aunt, who was really past ninety, but who for many years had called herself
seventy. This little indulgence any royal Duchess can allow herself without
misconstruction, for the true record of the birth is gazetted in the Peerage, where
any one can read it.

After dinner, the Duchess insisted that we should sit beside her, and “talk to
her,” which meant better still, for it was really to listen to her. She talked very
prettily about her royal grand‐niece, the Princess Louise, whom she said was a
“clever girl,” and her special favorite. This Princess had but recently married the
young Marquis of Lome, and was at that moment at Inverary, where the Duchess
said she was having “a very good time indeed.” She had already brought her
handsome husband to see her invalid great‐aunt, and left photographs of him,
taken together with herself, by which he appeared a very Scottish chieftain, which
the Duchess showed with great pride and satisfaction. (We did not tell her that
we had seen thousands of them in the shop windows already.) She said to Mr.
Seward, “Oh! Louise has great talent; she can do anything she likes. She is sure to
have a happy life, for she is never idle.” And she showed us a very pretty picture
which the Princess had painted for her. Finally she said, quite pathetically to Mr.
Seward, “Now that the Queen has permitted her own beautiful daughter to marry
a subject, I have a strong hope that she will contrive a way to let me have my
dear husband’s name and title before I die.”

But this never came to pass. The Duchess lived two years longer, and died
Duchess of Inverness, at the great age of ninety‐four, and the thirty‐first year of
her widowhood.

We parted as from a dear, old friend, and our own great‐aunt Caroline could not
have shown more tender, simple solicitude for our coming voyage, nor expressed
kinder wishes for our future, than did this venerable Duchess. We felt that we
had seen a little bit into the real home life of the British royalty, and as that was
before the Diaries of the Queen, and other Memoirs had been given to the world,
by which “Vicky’s” mishaps, and “Bertie’s” attractions, and many other domestic
details have been made familiar to all the world, it was a picture which dispelled
many of our prejudices, and modified our views. We had learned that there
could not be a better illustration of the greatness of gentleness, the sweetness of
sincerity, nor the dignity of direct simplicity, than the customs of this royal palace,
and the manners and ways of this royal lady, and her titled and decorated friends
afforded. We gained a new insight into the cause of the rare personal influence
which the royal family exercise, and an understanding of the loyal sentiment
which impels every British subject, in whatever quarter of the globe he may be,
to spring to his feet at the first strain of “God Save the Queen.”

This was the end of our first visit to London. We did not regret that a swift
ship was to bear us so soon to our own beloved land, but we felt that these last
days of our long journey had enriched us with new and strong sentiments of
affection and pride in our great mother country.


\end{document}
